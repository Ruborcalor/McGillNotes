
\documentclass{article}

\usepackage{amsmath}
\usepackage{amssymb}
\usepackage{enumerate}
\usepackage{parskip}
\usepackage{siunitx}
\usepackage{tikz}
\usepackage{upgreek}
\usepackage[margin=0.5in]{geometry}
\usepackage{graphicx}
\usepackage{esint}
\usepackage{nicematrix}
\NiceMatrixOptions{transparent}

\begin{document}

\title{Notes 2019-09-20}
\author{Cole Killian}

\maketitle

\section{Day 7}

\subsection{hashCode}

hashCode() returns a 32 bit integer associated to an object. When invoked on the same object during execution of Java, hasCode must return the same integer consistently.

o_1.equals(o_2) is true means that o_1.hashCode() == o_2.hasCode()

Converse not necessarily true. Think same hash, same bucket, but not same object.

\end{document}
