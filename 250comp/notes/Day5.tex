
\documentclass{article}


\usepackage{amsmath}
\usepackage{amssymb}
\usepackage{enumerate}
\usepackage{parskip}
\usepackage{siunitx}
\usepackage{tikz}
\usepackage{upgreek}
\usepackage[margin=0.5in]{geometry}
\usepackage{graphicx}
\usepackage{esint}
\usepackage{nicematrix}
\NiceMatrixOptions{transparent}


\usepackage{listings}
\usepackage{color}


\definecolor{dkgreen}{rgb}{0,0.6,0}
\definecolor{gray}{rgb}{0.5,0.5,0.5}
\definecolor{mauve}{rgb}{0.58,0,0.82}


\lstset{frame=tb,
  language=Java,
  aboveskip=3mm,
  belowskip=3mm,
  showstringspaces=false,
  columns=flexible,
  basicstyle={\small\ttfamily},
  numbers=none,
  numberstyle=\tiny\color{gray},
  keywordstyle=\color{blue},
  commentstyle=\color{dkgreen},
  stringstyle=\color{mauve},
  breaklines=true,
  breakatwhitespace=true,
  tabsize=3
}

\begin{document}

\title{Notes 2019-09-16}
\author{Cole Killian}

\maketitle

define package by writting "package packageName;" at the top of the class

i.e. 
\begin{lstlisting}
package nba.annoytingTeams;

public class MiamiHead {

}
\end{lstlisting}

Review: What happens if package names conflict.

Two main rules: (1) name of class must match name of file (with .java added) (e.g. MiamiHead.java). (2) Folder path must match exactly package name 

\subsection*{Using package in program}
Specifying entire path
\begin{lstlisting}
animals.Dog myDog = new animals.Dog();
\end{lstlisting}
Import package member:
\begin{lstlisting}
import animals.Dog;
\end{lstlisting}

Java compiler automatically imports two entire packages: java.lang and the current package

\subsection*{Classes}

Each time we define a class we create a new object type. An object is an instance of a class.

Nested class: defining a class within another class.

Reasons: To group class that are used only in one place.
When the class isn't useful in any other scenario, it makes no sense to not have it nested.

Increases encapsulation. which allows for better control

\subsection*{modifiers}

Acccess control modifiers: public, protected, default, private.

Non-access modifiers: static, final

Remember as a general rule fields should be declared private.

static method/field is associated with the entire class and called class variables.

non static method or field belongs to instance of class and also called instance variables.

Length is a non static method because its execution depends on specific string

parseInt() is a static method and does not depend on specific object of type Integer

final int cannot be changed after declaration. A final object cannot change its reference, but the object that it references can be modified.

Final field must be initialized.

\subsection*{Local variable vs. fields}

local variables are declared inside a method or block.

fields are declared inside class but outside a method

difference: 

scope: local variables only accessed within method or block

class variables accessed from any method or block inclass.

review: difference between fields and variables


\end{document}
