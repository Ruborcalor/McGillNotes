
\documentclass{article}

\usepackage{amsmath}
\usepackage{amssymb}
\usepackage{enumerate}
\usepackage{parskip}
\usepackage{siunitx}
\usepackage{tikz}
\usepackage{upgreek}
\usepackage[margin=0.5in]{geometry}
\usepackage{graphicx}
\usepackage{esint}
\usepackage{nicematrix}
\NiceMatrixOptions{transparent}


\begin{document}

\title{Notes 2019-09-09}
\author{Cole Killian}

\maketitle

\fbox{\begin{minipage}{\linewidth}
    Definition - The Cartesian Product: $A \times B \ \text{of} \ A \ \text{and} \  B = \{(a, b): a \in A, b \in B \}$
\end{minipage}}
i.e. $R \times R = R^2$ 
$[0, 1] \times [0, 2]$ 

\fbox{\begin{minipage}{\linewidth}
Definition - Functions in Calculus: Let D and E be sets; a function $f: D \to E$ is a rule that takes an input from D and assigns to it an output in E.
\end{minipage}}

\[
\begin{pmatrix}[name=A]
  1 \\
  2 \\
  3 
\end{pmatrix}
\ \ \ 
\begin{pmatrix}[name=B]
w \\
x \\
y \\
z
\end{pmatrix}
\\ \\ \\ 
\begin{pmatrix}[name=C]
  p \\
  q \\
  r
\end{pmatrix}
\]
\tikz [remember picture, overlay] \draw 
[red,->] (A-1-1) to (B-3-1) 
[red,->] (A-2-1) to (B-3-1)
[red,->] (A-3-1) to (B-4-1)
[red,->] (B-1-1) to (C-2-1) 
[red,->] (B-2-1) to (C-1-1)
[red,->] (B-3-1) to (C-3-1)
[red,->] (B-4-1) to (C-3-1); 

In modern math we define a function $f: D \to E$ as a subset $f \ \text{of} \ D \times E$ s.t. $\forall x \in D$ there exists EXACTLY ONE $y \in E$ s.t. $(x, y) \in f$.
Functions are thus just sets, there is thus just one fundamental concept (sets) we need to consider.

ex: $f: \{-1, 0, 1\} \to \{-1, 0, 1\}$ 

$x$ "maps to" $x^2$ 

image vs. codomain

$\{(-1, 1), (0, 0), (1, 1) \} = f$

\fbox{\begin{minipage}{\linewidth}
    Definition - a function $f: D \to E$ is called injective or one-to-one if $x_1 \neq x_2 \Rightarrow f(x_1) \neq f(x_2)$
    Everything gets mapped to its own unique point.
    Equivalently:  $f(x_1) = f(x_2) \Rightarrow x_1 = x_2$
\end{minipage}}

\fbox{\begin{minipage}{\linewidth}
    Definition - $f: D \to E$ is called surjective or "onto" if $\forall y \in E \ \text{"there exists"} \ x \in D : f(x) = y$
\end{minipage}}

\fbox{\begin{minipage}{\linewidth}
Definition - $f: D \to E$ is called bijective if f is both injective and surjective
\end{minipage}}

let $f: D \to E$, $A \subset D$
then $f(A) = \{f(x): x \in A\} \subset E$ is called the image of A under f

\fbox{\begin{minipage}{\linewidth}
     Definition - let f: $D \to E, B \subset E$, then $f^{-1}(B) = \{x \in D: f(x) \in B\} \subset D$ is called the inverse image of B under f
\end{minipage}}

CAUTION: The inverse image $f^{-1}(B)$ makes sense whether or not f is invertible!

ex: $f: \{-1, 0, 1\} \to \{-1, 0, 1\}$ 

$x$ "maps to" $x^2$ 

image vs. codomain

$\{(-1, 1), (0, 0), (1, 1) \} = f$

note that f is NOT injective (because $(f(1) = f(-1)$ and is thus not invertible. none the less, inv. images make sense.

$f^{1} = \{-1, 1\}$
$f^{0} = \{0\}$
$f^{-1} = \{\} = \varnothing$

ex: let $f: D \to E$ be bijective. 

then $f^{-1}(\{y_0\}) = \{x_0\}$ where $f(x_0) = y_0$

inv. function: $f^{-1}(y_0) = x_0$

inv. image:  $f^{-1}(\{y_0\}) = \{x_0\}$ 

\fbox{\begin{minipage}{\linewidth}
    Theorem (i): let $f: D \to E$, $A, B \subset D$ then
(a) $f(A \cup B) = f(A) \cup f(B)$

(b)  $f(A \cap B) \subset f(A) \cap f(B)$

(ii)  $let f: D \to E, \ \ A, B, \subset E$ then
(a) $f^{-1}(A \cup B) = f^{-1}(A) \cup f^{-1}(B)$

(b)  $f^{-1}(A \cap B) = f^{-1}(A) \cap f^{-1}(B)$
\end{minipage}}

(ii)(a) will be shown in the tutorials
(b) assign 1

we will prove (i):

(a) we have to show that the 2 sets $f(A \cup B) \ \text{and} \ f(A) \cup f(B)$ are equal

Proof:
let $y \in f(A \cup B)$ 
$\Rightarrow \ \text{"there exists"} \ x \in A \cup B: y = f(x)$
$\Rightarrow \ \text{"there exists"} \ x \in A: y = f(x) v \ \text{"there exists"} \ x \in B: y = f(x)$
$\Rightarrow y \in f(A) v y \in f(B)$ 
$\Rightarrow y \in f(A) \cup f(B) \Rightarrow f(A \cup B) \subset f(A) \cup f(B)$


proof part 2:
 \begin{align*}
   \ \text{let} \ y \in f(A) \cup f(B) \\
   & y \in f(A) v y \in f(B) \\
   & \ \text{"there exists"} \ x \in A: y = f(x) v \ \text{"There exists"} \ x \in B: y = f(x) \\
   & \ \text{"there exists"} \ x \in A \cup B: y = f(x) \\
   & \Rightarrow y \in f(A \cup B) \\
   & f(A) \cup f(B) \subset f(A \cup B) \\
   & \Rightarrow f(A \cup B) = f(A) \cup f(B)
\end{align*} 

\end{document}
