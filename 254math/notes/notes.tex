
\documentclass[11pt]{scrartcl}
\usepackage[sexy]{/home/gautierk/.config/evan}
\title{Math 235}
\date{\today}
\author{Cole Killian}
\usepackage{array}
\usepackage{pifont}
\newcommand{\cmark}{\ding{51}}%
\newcommand{\xmark}{\ding{55}}%
\newtheorem{recap}{Recap} %10-02

\begin{document}

\title{Math 254 Notes}
\author{Cole Killian}

\maketitle

\section{Lecture 10-02}

\subsection{Open and Closed Sets}

\begin{definition}
  \ul{Open} interval does not contain any of its boundary points.
  \ul{Closed} interval contains all of its boundary points.
\end{definition}

\begin{theorem}
  Every open interval is open. This is not self evident because definition of open is very specific.
  \begin{proof}Let I be an open interval. We need to show that I is always open.
    \begin{enumerate}
      \ii
      Case: $I = \left]a, \infty \right]$

      Let $x \in I$ be arbitrary. Let $\epsilon = x - a$. Then $V_\epsilon(x) = \left]x - \epsilon, x + \epsilon\right] = \left]a, 2x - a\right] \subset \left]a, \infty\right]$. i.e. $V_\epsilon(x) \subset \left]a, \infty\right] \Rightarrow I$ is open.
      \ii
      Case: $I = \left]-\infty, b\right]$. Do yourself. Let $x \in I$ be arbitrary. Let $\epsilon = b - x$. Then $V_\epsilon(x) = \left]x - \epsilon, x + \epsilon\right[ = \left]2x - b, b\right[ \subseteq \left]-\infty, b\right[$. Therefore I is open.
      \ii
      Case: $I = \left]a, b\right[$.

      Let  $\epsilon = $ min $\{x - a, b - x\} > 0$.

      Then $V_\epsilon(x) = \left]x - \epsilon, x + \epsilon\right[$.

      Note that $x + \epsilon \leq x + (b - x) = b$ and $x - \epsilon \geq x - (x - a) = a \Rightarrow \left]x - \epsilon, x + \epsilon\right] \subset \left]a, b\right[$ . Therefore I is open and therefore any open interval is open.
    \end{enumerate}
  \end{proof}
\end{theorem}

\begin{theorem}
  Every closed interval is closed.
  \begin{proof}
    Let I be a closed interval. We need to show that $\RR \setminus I$ is open.
    \begin{enumerate}
      \ii
      Case: $I = [a, \infty] \Rightarrow \RR \setminus I = \left] -\infty, a\right[$ which as an open interval is open $\Rightarrow$ I is closed.
      \ii
      Case: I = $\left[-\infty, b\right]$. do yourself. $\RR \setminus I = \left]b, \infty\right[$ which is an open interval $\Rightarrow$ I is closed.
      \ii
      Case: $I = \left[a, b\right] \Rightarrow \RR \setminus I = \left]-\infty, a\right] \cup \left]b, \infty\right]$. Union of open with open is open so I is closed. Therefore any closed interval is closed.
    \end{enumerate}
  \end{proof}
\end{theorem}

\begin{theorem}
  \begin{enumerate}[label=\alph*.]
    \ii[]
    \ii
    Let J be an index set and let $u_j$ be open for all $j \in J$. Then
    $$\bigcup_{j \in J}u_j$$ is open.
    "Arbirary unions of open sets are open.
    \begin{proof}
      Let $u = \bigcup_{j \in J}u_j$. Let $x \in u$ be arbitrary $\Rightarrow \exists j \in J$ such that $x \in u_j$ open $\Rightarrow \exists \epsilon > 0: V_\epsilon(x) \subset u_j \subset U$. Can't follow. Basically uses definition of union and definition of openness.
    \end{proof}
    \ii
    Let $u_1, \dots, u_n$ be open. Then 
    $$ \bigcap_{j = 1}^n u_j$$ is open. "Finite intersections of open sets are open.
    \ii
    Finite unions of closed sets are closed.
    Let $v_1, \dots, v_n$ be closed, then $\bigcup_{j = 1}^{n}v_j$ is closed.
    \begin{proof}
      Let $v_1, \dots, v_n$ be closed, then $\RR \setminus v_1, \dots, \RR \setminus v_n$ are all open.
      $\Rightarrow \RR \setminus v_1 \cap \dots \cap \RR \setminus v_n$ is open. By demorgans law this equals $\RR \setminus (v_1 \cup \dots \cup v_n)$ is closed.
      $\Rightarrow v_1 \cup \dots \cup v_n$ is closed. (because closed is comp lement of open)
    \end{proof}
    \ii
    Arbitrary intersections of closed sets are closed.
    Let J be an arbitraryIndex set and let $\forall j \in J v_j$ be closed, then $\cap_{j \in J}v_j$ is closed.
    \begin{proof}
      Let $v_j$ be closed for all $j \in J$.
      $\Rightarrow \RR \setminus v_j$ is open.
      $\Rightarrow \cup_{j \in J}(\RR \setminus v_j)$ is open.
      Demorgans law gives us that $\RR \setminus \cap_{j \in J} v_j$ is open.
      Therefore $\cap_{j = 1}v_j$ is closed.
    \end{proof}
  \end{enumerate}
\end{theorem}

\begin{example}
  Every finite subset of $\RR$ is closed.
  \begin{proof}
    Let's first consider $\{x\}$. For some $x \in \RR$, $\{x\} = \left]x, x\right]$ is closed.
    Finite unions of singleton sets are thus closed $\Rightarrow$ all finite subsets of $\RR$ are closed.
  \end{proof}
\end{example}
\begin{example}
  $S_1 = \{\frac{1}{n}: n \in \NN\}$ is NOT closed.
  \begin{proof}
    Assume it is closed. Then the comp lement $u$ is open. we have $0 \in u$, but every $\epsilon$ neighborhood $V_\epsilon$ intersects $S_1$ and is thus not contained in $u \Rightarrow u$ is not open.
  \end{proof}
\end{example}
\begin{example}
  $S_2 = \{\frac{1}{n}: n \in \NN\} \cup \{0\}$ is closed.
\end{example}

\begin{definition}
  Boundary.

  Let $S \subset \RR$. A point $x \in \RR$ is called a \ul{boundary point} of S if every epsilon neighborhood cenetered around x intersects both S and the comp lement of S.

  A point $x \in \RR$ is not a boundary point of S if $\exists \epsilon > 0: V_\epsilon(x) \cap S = \varnothing \vee V_\epsilon(x) \cap (\RR \setminus S) = \varnothing$.
\end{definition}

\begin{example}
  $S = \left]a, \infty\right]$ 
  Claim: 
\end{example}

\section{Lecture 10-07}

\begin{example}
\end{example}

\begin{definition}
  Let $S \subseteq \RR$. The \ul{interior} $\mathring{S}$ "S with dot on top" or int(S) is defined as:
  $$ \mathring{S} = \bigcup_{U \subset S, \ \ U open}U $$
  Note that $\mathring{S}$ is open as a union of open sets.
  It is the largest open subset of S.
\end{definition}

\begin{definition}
  Let $S \subset \RR$. The \ul{closure} $\tilde{S}$ of S is defined as:
  $$ \tilde{S} = \bigcap_{V \supset S, \ \ V closed} V $$
  Note that $\tilde{S}$ is closed as an intersection of closed sets.
\end{definition}

\begin{theorem}
  Let $S \subset \RR$. Then $\mathring{S} = S \setminus \delta S$
  \begin{proof}
    \begin{enumerate}
      \ii
      $"\subseteq"$. Let $x \in \mathring{S} \Rightarrow \exists U \ \text{open} \ , U \subset S$ with $x \in U$.

      Thus $\exists \epsilon > 0 s.t. V_{\epsilon}(x) \subset U \subset S$.
    \end{enumerate}
  \end{proof}
\end{theorem}

\begin{theorem}
  Let $S \subset \RR$. Then $\RR \setminus \tilde{S} = int(\RR \setminus S)$.
\end{theorem}
\end{document}

