\documentclass[class=scrartcl, crop=false]{standalone}

\usepackage[sexy]{evan}
\usepackage{cole}

\date{2019-10-07}


\begin{document}

\section{Lecture 10-07}

\begin{example}
\end{example}

\begin{definition}
  Let $S \subseteq \RR$. The \ul{interior} $\mathring{S}$ "S with dot on top" or int(S) is defined as:
  $$ \mathring{S} = \bigcup_{U \subset S, \ \ U open}U $$
  Note that $\mathring{S}$ is open as a union of open sets.
  It is the largest open subset of S.
\end{definition}

\begin{definition}
  Let $S \subset \RR$. The \ul{closure} $\tilde{S}$ of S is defined as:
  $$ \tilde{S} = \bigcap_{V \supset S, \ \ V closed} V $$
  Note that $\tilde{S}$ is closed as an intersection of closed sets.
\end{definition}

\begin{theorem}
  Let $S \subset \RR$. Then $\mathring{S} = S \setminus \delta S$
  \begin{proof}
    \begin{enumerate}
      \ii
      $"\subseteq"$. Let $x \in \mathring{S} \Rightarrow \exists U \ \text{open} \ , U \subset S$ with $x \in U$.

      Thus $\exists \epsilon > 0 s.t. V_{\epsilon}(x) \subset U \subset S$.
    \end{enumerate}
  \end{proof}
\end{theorem}

\begin{theorem}
  Let $S \subset \RR$. Then $\RR \setminus \tilde{S} = int(\RR \setminus S)$.
\end{theorem}


\end{document}
