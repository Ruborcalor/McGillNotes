\documentclass[class=scrartcl, crop=false]{standalone}

\usepackage[sexy]{evan}


\begin{document}

\section{10-30}

\begin{theorem}
  A bounded sequence converges if and only if it has \ul{exactly one} accumulation point.
  \begin{proof}
    \begin{itemize}
      \ii[]
      \ii[$(\Rightarrow)$ ]
      Let $(x_n)$ be convergent. $x \coloneqq \lim(x_n)$. Then every subsequence $(x_{n_k})$ of $(x_n)$ converges to $x$. Thus $x$ is the only accumulation point of $(x_n)$.
      \ii[$(\Leftarrow)$ ]
      Let $(x_n)$ be a bounded sequence which has only one accumulation point $x$. We will show that $(x_n)$ converges to $x$. Assume that this is \ul{not} the case.
      \\\\
      Convergence: $\forall \epsilon > 0, \ \exists N \in \NN: \forall n \geq N, \ |x_n - x| < \epsilon$
      \\
      Negation: $\exists \epsilon > 0 : \forall N \in \NN, \ \exists n \geq N: |x_n - x| \geq \epsilon$
      \\\\
      Thus $\exists$ infinitely many $n \in \NN$ such that $|x_n - x| \geq \epsilon_0$.
      \\
      Let $n_1 < n_2 < n_3 < \dots$ such that $\forall k \in \NN : |x_{n_k} - x| \geq \epsilon_0$.
      \\\\
      Consider the subsequence $(x_{n_k})$ of $(x_n) \Rightarrow (x_{n_k})$ is bounded because $(x_n)$ is bounded. 
      \\\\
      By Bolzano-weierstrass, $(x_{n_k})$ has a convergent subsequence $(x_{n_{k_j}})$. Let $\sim{x} \coloneqq \lim(x_{n_{k_j}})$. Since $it$ is a subsequence of $(x_n)$ which has only one accumulation point. It follows that $\sim{x} = x$.
      \\\\
      Thus $\lim(it) = x$ and $\forall j \in \NN, |it - x| \geq \epsilon_0 CONTRADICTION$
      \\
      Thus our assumption was wrong which proves that $(x_n)$ converges to x.
    \end{itemize}
  \end{proof}
\end{theorem}

\begin{theorem}
  Let $(x_n)$ be a bounded sequence and let $A$ be the set of all accumulation points of $(x_n)$. Then $A \neq \varnothing$ and $A$ is compact (i.e. $A$ is closed and bounded).
  \begin{proof}
    By BOLZANO-WEIERSTRASS, $(x_n)$ has at least one convergent subsequence. Its limit is an accumulation point of $(x_n) \Rightarrow A \neq \varnothing$.
    \\\\
    \ul{A is bounded}:  $(x_n)$ is bounded i.e. $\exists M > 0$ such that $\forall n \in \NN, \ -M \leq x_n \leq M$.
    \\\\
    Let $x \in A$ be arbitrary. Then $\exists$ subsequence $(x_{n_k})$ of $(x_n)$ with $x = \lim(x_{n_k})$.
    \\
    We have that $\forall k \in \NN : -M \leq x_{n_k} \leq M \Rightarrow -M \leq x \leq M$.
    \\
    $\Rightarrow x \in [-M, M]$ for all accumulation points $x$ of $(x_n)$.
    \\
    $\Rightarrow A \subseteq [-M, M] \Rightarrow A$ is bounded.
    \\\\
    \ul{A is closed:} Let $x \in \RR \setminus A$ i.e. $x$ is \ul{not} an accumulation point. Thus $\exists \epsilon > 0 : V_\epsilon(x)$  contains at most \ul{finitely many} terms of $(x_n)$.
    \\
    Let $t \in V_\epsilon(x)$. $V_\epsilon(x)$ is open. Thus $\exists \tilde{\epsilon} > 0: V_{\tilde{\epsilon}(t)} \subseteq V_\epsilon(x)$.
    \\
    Thus $V_{\tilde{\epsilon}(t)}$ contains at most finitely many terms of $(x_n)$. Thus $t$ is \ul{not} an accumulation point $\Rightarrow$ no point in $V_\epsilon(x)$ is an accumulation point of $(x_n) \ \Rightarrow V_\epsilon(x) \subseteq \RR \setminus A$. Thus $\RR \setminus A$ is open $\Rightarrow A$ is closed.

  \end{proof}
\end{theorem}
We've just seen that the set of all accumulation points of a bounded sequence $(x_n)$ is $\neq 0$, closed, and bounded. 
\\
\\
Since $A$ is bounded, it has a supremum and an infimum. Both $\sup$ and $\inf$ are boundary points. $A$ is closed so it contains $\sup$ and $\inf$. Therefore $\sup(A)$ is the  \ul{Maximum} of A and $\inf(A)$ is the \ul{minimum} of A. i.e. $\sup(A)$ is an accumulation point of $(x_n)$, the greatest accumulation point of $(x_n)$. Similarly $\inf(A)$ is the least accumulation point of $(x_n)$.

\begin{definition}
  \begin{enumerate}
    \ii[]
    \ii
    Let $(x_n)$ be a bounded sequence. Then the greatest accumulation point of $(x_n)$ is called the \ul{LIMES SUPERIOR} of $(x_n)$. In symbols: $\limsup(x_n)$.
    \ii
    The \ul{least} accumulation point of $(x_n)$ is called the \ul{LIMES INFERIOR} of $(x_n)$. In symbols: $\liminf(x_n)$.
  \end{enumerate}
\end{definition}

\begin{theorem}
  Let $(x_n)$ be a bounded sequence. Then $(x_n)$ is convergent if and only if 
  \[
    \liminf(x_n) = \limsup(x_n)
  \]
  \begin{proof}
    \begin{itemize}
      \ii[]
      \ii[$(\Rightarrow)$ ]
      Let $x \coloneqq \lim(x_n)$. Then every subsequence $(x_{n_k})$ of $(x_n)$ converges to $x$. 
      \[
        \Rightarrow A = \{x\} \Rightarrow \liminf(A) = x = \limsup(A)
      \]
      \ii[$(\Leftarrow)$ ]
      Assume that $\liminf(x_n) = \limsup(x_n) \coloneqq x$. 
      \[
        A = \{x\}
      \]
      i.e. $(x_n)$ has only one accumulation point. By previous theorem, $(x_n)$ converges.
    \end{itemize}
  \end{proof}
\end{theorem}

\begin{example}
  \begin{enumerate}
    \ii[]
    \ii
    \[
      x_n = (-1)^n
    \]
    Accumulation points are $-1$ and $1 \Rightarrow \liminf(x_n) = -1$ and $\limsup = 1$. Especially, $(-1)^n$ diverges because $\liminf \neq \limsup$.
    \ii
    Let $(x_n)$ be an enumeration of $\QQ \, \cap \, [a, b]$ where $a < b$. We'll show that $\liminf = a$ and that $\limsup = b$.
    \begin{proof}
      Let $x > b$. Let $\epsilon \coloneqq b - x > 0$. Then $\forall n \in \NN$, $x_n \notin V_\epsilon(x) \Rightarrow$ x is \ul{not} an accumulation point of $(x_n)$.
      \\\\
      Let $x \in [a, b]$ and let $\epsilon > 0$; consider $V_\epsilon(x) = ]x - \epsilon, x + \epsilon[$. By the density of $\QQ$ in $\RR$, $V_\epsilon(x)$ contains infinitely many rational numbers, especially, $V_\epsilon(x_n)$ contains infinitely many terms of $(x_n) \Rightarrow x$ is an accumulation point of $(x_n)$.
      \\\\
      \ul{x = a}: By density of $\QQ$ in $\RR$, $]a, a + \epsilon[$ contains infinitely many terms of $(x_n) \Rightarrow a$ is an accumulation point of $(x_n)$.
      Similarly for $x = b$.
      \\\\
      Therefore $A \coloneqq [a, b] \Rightarrow \liminf(x_n) = a \ \text{and} \ \limsup(x_n) = b$.
    \end{proof}
    \ii
    Find all accumulation points of the following sequence.
    \[
      x_n : 1, 1,  \frac{1}{2}, 1, \frac{1}{2}, \frac{1}{3}, 1, \frac{1}{2}, \frac{1}{3}, \frac{1}{4}, \dots
    \]
    Claim: $A = \{0\} \cup \{1, \frac{1}{2}, \frac{1}{3}, \frac{1}{4}, \dots\}$
    \begin{proof}
      For every $k \in \NN$, the constant sequence $\frac{1}{n}, \frac{1}{n}, \frac{1}{n}$ is a subsequence of $(x_n)$. Thus 
      \[
        \frac{1}{n} = \lim(\frac{1}{n}, \frac{1}{n}, \frac{1}{n}, \dots) \in A
      \]
      and $1, \frac{1}{2}, \frac{1}{3}, \frac{1}{4}, \dots$ is a subsequence of $(x_n)$. Thus
      \[
        0 = \lim(1, \frac{1}{2}, \frac{1}{3}, \dots) \in A
      \]
      Now let $x > 1, \ \epsilon \coloneqq x - 1 > 0$. Then $\forall n \in \NN : x_n \notin V_\epsilon(x) \Rightarrow x \notin A$.
      \\\\
      Similarly, $x \notin A$ for all $x < 0$. Let $0 < x < 1; \ x \notin A$. Then $\exists n \in \NN : \frac{1}{n + 1} < x < \frac{1}{n}$.
      \\\\
      Let $\epsilon \coloneqq \min\{x - \frac{1}{n + 1}, \frac{1}{n} - x\} > 0$. Then $\frac{1}{n + 1} \notin V_\epsilon(x) \vee \frac{1}{n} \notin V_\epsilon(x)$ 
      \begin{gather*}
        \Rightarrow x_n \notin V_\epsilon(x) \ \forall n \in \NN \\
        x \ \text{ is \ul{not} an accumulation point of} \ (x_n) \\
      \end{gather*}
      Thus $A = \{0\} \cup \{\frac{1}{n}: n \in \NN\}$
    \end{proof}
  \end{enumerate}
\end{example}

\subsection{Properties of $\limsup, \liminf$}
\begin{theorem}
  Let $(x_n)$ be a bounded sequence and let $\epsilon > 0$. Then $\exists N \in \NN \ \forall n \geq N : x_n \in ]\liminf(x_n), \limsup(x_n) + \epsilon[$. i.e. at most finitely many terms of $(x_n)$ have the property that $x_n > \limsup(x_n) + \epsilon \ \text{or} \ x_n < \liminf(x_n) - \epsilon$
  \begin{proof}
    assignment 8
  \end{proof}
\end{theorem}

\begin{theorem}
  Let $(x_n)$ be a bounded sequence. Then $\limsup(x_n) = \lim(\sup\{x_k : k \geq n \})$ and $\liminf(x_n) = \lim(\inf\{x_k: k \geq n\})$.
  \begin{remark}
    It is not clear initially whether this is well defined. We'll prove this.
  \end{remark} \noindent
  Let $y_n \coloneqq \sup\{x_k: k \geq n\}$. Then $(y_n)$ is bounded because $(x_n)$ is bounded.
  \\\\
  Let $A, B$ be bounded with $A \subseteq B$. Then $\sup(A) \leq \sup(B)$.
   \begin{note}
     $\{x_k: k \geq n + 1 \} \subseteq \{x_k : k \geq n\}$.
  \end{note} \noindent
  Therefore $\sup\{x_k: k \geq n + 1\} \leq \sup\{x_k: k \geq n\}$.
  \\\\
  Therefore $(y_n)$ is bounded and decreasing and therefore converges. 
  \\\\
  Thus $\lim(\sup\{x_k : x \geq n\})$ exists. A similar argument applies to $\lim(\inf\{x_k: k \geq n\})$.
  \begin{proof}
    Examination material. This is the cutoff for the midterm exam. Next week coshy sequences. 3.4 in the textbook. Important: This doesn't mean that you don't have to remember the stuff from before. If you don't know stuff from before you will be closed. I used open and closed todayand left it to you to know what open and closed means. It did not contain interior and closure so that is midterm 2 material. And you need to know what boundary sets are in order to make sense of these things but I won't ask a separate question on these things.
  \end{proof}
\end{theorem}

\end{document}
