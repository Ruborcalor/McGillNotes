\documentclass[class=scrartcl, crop=false]{standalone}

\usepackage[sexy]{evan}
\usepackage{cole}

\date{2019-11-06}


\begin{document}

\section{11-06}

\subsection{Divergence to infinity}

\begin{definition}
  Let $(x_n)$ be a sequence. We say that $(x_n)$ diverges to $+ \infty$ if 
  \[\forall M > 0, \ \exists N \in \NN, \ \forall n \geq N : x_n > M\]
  In symbols:
  \[\lim(x_n) = +\infty\]
  \\\\
  $(x_n)$ diverges to $- \infty$ if 
  \[\forall M > 0 (\exists N \in \NN)(\forall \geq N):x_n < -M\]
  In symbols: 
  \[\lim(x_n) = -\infty\]
\end{definition} 
\begin{remark}
  If $\lim(x_n) = + \infty$ or $\lim(x_n) = -\infty$, then the sequence \ul{diverges}. The limit laws thus do \ul{NOT} apply.
\end{remark} 
\begin{example}
  $\lim(n^2) = + \infty$. Let $M > 0$. Then $n^2 > M \Leftrightarrow n > \sqrt{M}$.
  \\\\
  Let $N > \sqrt{M}$. Then $\forall n \geq N: n^2 \geq N^2 > M \Rightarrow n^2 > M$ for all $n \geq M \Rightarrow (n^2)$ diverges to $+\infty$.
\end{example} 
\begin{example}
  Let  $a > 1$. Show that $\lim(a^n) = + \infty$.
  \\\\
  Since $a > 1$, $b \coloneqq a - 1 > 0$. Then $a = 1 + b$ and $a^n = (1 + b)^n$. Applying bernoulli's:
  \[
    (1 + b)^n \geq 1 + nb > nb > M \Leftrightarrow n > \frac{M}{b}
  \]
  Let $N > \frac{M}{b}$. Then $\forall n \geq N$, we know that $a^n > nb \geq Nb > M$. Thus $a^n$ diverges to $+\infty$.
\end{example} 

\subsection{Chapter 4: Limits of functions}

Preparatory definition:
\begin{definition}[In $A$]
  Let $A \subseteq \RR$. A sequence $(x_n)$ is said to be \ul{in A} if $\forall n \in \NN : x_n \in A$.
\end{definition} 

\begin{definition}[Cluster point]
  Let $A \subseteq \RR$. A point $x \in \RR$ is called a \ul{cluster point} of $A$ if:
  \[
    \forall \epsilon > 0 : \underbrace{V_\epsilon(x) \setminus \{x\}}_{\text{Punctured neighborhood}} \cap A \neq \varnothing
  \]
\end{definition} 

\begin{note}
  Notation for punctered neighborhoods:
  \[
    V_\epsilon^*(x) \coloneqq V_\epsilon(x) \setminus \{x\}
  \]
  i.e. $x$ is a cluster point of A if $\forall \epsilon > 0 : V_\epsilon^*(x) \cap A \neq \varnothing$.
  \begin{remark}
    Cluster points of $A$ are \ul{not} necessarily elements of $A$.
  \end{remark} 
\end{note} 

\begin{definition}[Isolated Point]
  Let $A \subseteq \RR$. $x \in A$ is called an \ul{isolated point} of $A$ if $\exists \epsilon > 0 : V_\epsilon^*(x) \cap A = \varnothing$.
  \\\\
  i.e. $x$ is the \ul{only} element of $A$ that is in  $V_\epsilon(x)$.
\end{definition} 

\begin{example}
  $S \coloneqq \{0\} \cup \{ \frac{1}{n} : n \in \NN\}$.
  \\\\
  Claim: 0 is the only cluster point of $S$. All points $\frac{1}{n}: n \in \NN$ are isolated points of $S$.
  \begin{proof}[0 \ul{is} a cluster point.]
    Let $\epsilon > 0$. Then $V_\epsilon(0)$ contains infinitely many numbers of the form $\frac{1}{n}$ because $\lim(\frac{1}{n}) = 0$. Thus $0$ is a cluster point of $S$.
    \\\\
    Let $x \neq 0$. Then $\exists \epsilon > 0 : V_\epsilon^*(x) \cap S = \varnothing$ (left as exercise). Especially, such $\epsilon > 0$ exists for all $x = \frac{1}{n}$. Thus every $\frac{1}{n}$ is an isolated point of $S$.
  \end{proof} 
\end{example} 
\begin{example}
  Let $A \coloneqq \QQ$. Then \ul{every} real number is a cluster point of $A$.
  \begin{proof}
    Let $x \in \RR$ be arbitrary and let $\epsilon > 0$. Since $\QQ$ is dense in $\RR$, $V_\epsilon(x)$ contains infinitely many rational numbers. Thus $V_\epsilon^*(x)$ contains at least one (in fact infinitely many) rational numbers. i.e.
    \[
      V_\epsilon^*(x) \cap A \neq \varnothing \Rightarrow x \ \text{is a cluster point of} \ A
    \]
  \end{proof} 
\end{example} 
\begin{exercise}
  Let $I$ be an interval. Then the set of all cluster points of I is $\overline{I}$
\end{exercise} 
\begin{theorem}
  Let $A \subseteq \RR$. Then $x \in \RR$ is a cluster point of $A$ if and only if there exists  a sequence $(x_n)$ in $A \setminus \{x\}$ with $\lim(x_n) = x$.
  \begin{proof}
    \begin{itemize}
      \ii[]
      \ii[$(\Rightarrow)$ ]
      Let $x$ be a cluster point of $A$.
      \\\\
      Let $\epsilon \coloneqq 1$. Then $V_\epsilon^*(x) \cap A \neq \varnothing$. Let $x_1 \in V_1^*(x) \cap A$.
      \\\\
      Let $\epsilon \coloneqq \frac{1}{2}$. Then $V_\epsilon^*(x) \cap A \neq \varnothing$. Let $x_2 \in V_{\frac{1}{2}}^*(x) \cap A$.
      \\\\
      We obtain a sequence $(x_n)$ in $A \setminus \{x\}$ with $\forall n\in \NN : x_n \in V_{\frac{1}{n}}^*(x) \cap A$.
      \\\\
      Let $\epsilon > 0$. Let $N > \frac{1}{\epsilon} \Leftrightarrow \frac{1}{N} < \epsilon$. Then
      \[
        \forall n \geq N : x_n \in V_{\frac{1}{n}}^*(x) \cap A \subseteq V_{\frac{1}{n}}^*(x) \subseteq V_{\frac{1}{n}}(x) \subseteq V_{\frac{1}{N}}(x) \subseteq V_\epsilon(x).
      \]
      i.e. $\forall n \geq N : x_n \in V_\epsilon(x) \Rightarrow (x_n)$ converges to $x$.
      \ii[$(\Leftarrow)$ ]
      Let $(x_n)$ be a sequence in $A \setminus \{x\}$ such that $\lim(x_n) = x$. Let $\epsilon > 0$. Then $\exists N \in \NN,\ \forall n \geq N : x_n \in V_\epsilon(x)$. But since $x_n \in A \setminus \{x\}$, $x_n \neq x$. This means that $x_n \in V_\epsilon^*(x)$ and $x_n \in A$. Thus $\forall n \geq N : x_n \in V_\epsilon^*(x) \cap A$. Thus $v_\epsilon^*(x) \cap A \neq \varnothing \Rightarrow x$ is a cluster point.
    \end{itemize} 
  \end{proof} 
\end{theorem} 
\begin{theorem}
  Let $A \subseteq \RR$. Let $x$ be a cluster point of $A$. Then $x \in \overline{A}$.
  \begin{proof}
    Let $x$ be a cluster point of $A$. By previous theorem, $\exists (x_n)$ is $A \setminus \{x\}$ such that $\lim(x_n) = 0$.
    \\\\
    Since $\forall n \in \NN : x_n \in A \setminus \{x\}$. We have that $\forall n \in \NN : x_n \in \overline{A} \supseteq A \setminus \{x\}$.
    \\\\
    Since $\overline{A}$ is closed, $\lim(x_n) \in \overline{A}$ (see assignment 6).
  \end{proof} 
\end{theorem} 

\begin{definition}[The limit of a function: \ul{Sequential Definition}] \leavevmode \\
  Let $f: A \subseteq \RR \to \RR$. Let $x_0 \in \RR$, we say that \ul{$L$ is a limit of $f$ as $x \to x_0$ }. In symbols:
  \[
    L = \lim_{x \to x_0}f(x)
  \]
  if for \ul{all} sequences $(x_n)$ in $A \setminus \{x_0\}$ with $\lim(x_n) = x_0$, we have that $\lim(f(x_n)) = L$.
\end{definition} 

\begin{example}
  Let \[f: \RR \setminus \{0\} \to \RR, x \to \frac{x^2}{|x|}\]
  \\\\
  Note that for $x \neq 0$ we have that 
  \[
    \frac{x ^2}{|x|} = |x|
  \]
  Claim: $\lim_{x \to 0}f(x) = 0$.
  \\\\
  Let $(x_n)$ be a sequence such that $x_n \neq 0$ for all $n \in \NN$ and such that $\lim(x_n) = 0$. We need to show that $(f(x_n))$ converges to $0$. Note that $f(x_n) = |x_n|$.
  \\\\
  Let $\epsilon > 0$. Since $\lim(x_n) = 0$, there exists $(N \in \NN)(\forall n \geq N):|x_n - 0| = |x_n| < \epsilon$.
  \\\\
  Thus $\forall n \geq N : ||x_n| - 0| = ||x_n|| = |x_n| < \epsilon \Rightarrow \lim(f(x_n)) = 0$. Thus:
  \[
    \lim_{x \to x_0}f(x) = 0
  \]
\end{example} 

\begin{example}
  Let $f: \RR \setminus \{0\} \to \RR$ where $x \to \frac{1}{x}$. Let $x_0 \neq 0$. Show that
  \[
    \lim_{x \to x_0}f(x) = \frac{1}{x_0}
  \]
  \begin{proof}
    Let $(x_n)$ be a sequence in $\RR \setminus \{0, x_0\}$ with $\lim(x_n) = x_0$. Then $\lim(f(x_n)) = \lim(\frac{1}{x_n}) = \frac{1}{\lim(x_n)} = \frac{1}{x_0}$.
  \end{proof} 
\end{example} 
\begin{example}
  Let $f: \ZZ \to \RR$ where $x \to 0$. Let $L \in \RR$ be \ul{arbitrary}. Then
  \[
    \lim_{x \to 0}f(x) = L
  \]
  Since $0$ is an \ul{isolated} point in $\ZZ$, there doesn't exist \ul{any} sequence in $\ZZ \setminus \{0\}$ that converges to 0. Thus $\ul{all}$ sequences $(x_n)$ in $\ZZ \setminus \{0\}$ that converge to $0$ hvae that property that
  \[
    \lim_{x \to 0}f(x_0) = L
  \]
  Thus $\lim_{x \to 0} f(x) = L$ for \ul{any} $L \in \RR$.
  \begin{remark}
    This example shows that we should avoid isolated points when considering limits.
  \end{remark} 
\end{example} 

\begin{theorem}
  Let $f: A \to \RR$ where $x_0$ is a \ul{cluster point} of $A$.
  \\\\
  Then: if $f$ has a limit as $x$ approaches $x_0$, then this limit is uniquely determined.
  \begin{proof}
    Let $L_1, L_2$ be limits of $f$ as $x$ approaches $x_0$. Then $\exists (x_n)$ is $A \setminus \{x_0\}$ with $\lim(x_n) = x_0$. Because $f$ has a limit at $x_0$, $\lim(f(x_n))$ exists and $L_1 = \lim(f(x_n)) = L_2$.
  \end{proof} 
\end{theorem} 

\end{document}
