
\documentclass{article}

\usepackage{amsmath}
\usepackage{amssymb}
\usepackage{enumerate}
\usepackage{parskip}
\usepackage{siunitx}
\usepackage{tikz}
\usepackage{upgreek}
\usepackage[margin=0.5in]{geometry}
\usepackage{graphicx}
\usepackage{esint}
\usepackage{nicematrix}
\NiceMatrixOptions{transparent}

\begin{document}

\title{Notes 2019-09-25}
\author{Cole Killian}

\maketitle

\section{Catching up because I was late}

\fbox{\begin{minipage}{\linewidth}
    Theorem Let $S \subset \mathbb{R}$, if S is bounded from above, then S has a supremum in $\mathbb{R}$. Because of this property, $\mathbb{R}$ is complete.
\end{minipage}}

Remark: This is not self evident. The set of rational numbers for example. Consider $S = \{x \in \mathbb{Q}: x^2 < 2\}$. In this case 2 is an upper bound so S is bounded from above, but S does not have a least upper bound in  $\mathbb{Q}$ (the actual supremum is $\sqrt{2}$.

Intuitively the cause of this is that $\mathbb{Q}$ has holes. This thus claims that real numbers do not have any holes i.e. that R is complete. This is too informal to be proven rigorously. Instead, in modern math we define the real numbers as the completion of $\mathbb{Q}$. Therefore real numbers are automatically complete.

Application of completeness:

\fbox{\begin{minipage}{\linewidth}
    Theorem - The archimedean property of $\mathbb{R}$.

    Let $x \in \mathbb{R}$. Then $\exists n \in \mathbb{N}$ such that $x \leq n$.

    Informal interpretation: There are no infinite real numbers.
\end{minipage}}

\fbox{\begin{minipage}{\linewidth}
    Theorem - Let $x \in \mathbb{R}$. Assuming that no such n exists such that $x > n$ for all $n \in \mathbb{N}$. This would mean that x is an upper bound of $\mathbb{N}$. This would mean that x is an upper bound for N. Hence N is bounded from above. "By completeness of $\mathbb{R}$, natural numbers have a supremum $u = supr(\mathbb{N})$.
\end{minipage}}

Then $u - 1$ is not an upper bound for natural numbers.

This would mean that $\exists n \in \mathbb{N}$ such that $u - 1 < n \Rightarrow u < n + 1$. $n + 1 \in \mathbb{N}$ so this is a contradiction. Therefore $u$ is not an upper bound for $\mathbb{N}$.

"Thus $\exists n \in \mathbb{N}$ such that $x \leq n$ square." Natural numbers go just as high as real numbers.

Remark: It is also true that every nonempty subset of $\mathbb{R}$ which is bounded from below has an infimum in $\mathbb{R}$. (See assignment 3 \#3)

\fbox{\begin{minipage}{\linewidth}
Theorem -

(a) Let S be bounded from above, $S \neq \varnothing$, then $u = supr(S)$ if and only if (IFF)

\quad (1) u is an upper bound for S

\quad (2) $\forall \epsilon > 0, \exists v_\epsilon \in S$ such that $u - \epsilon < v_\epsilon$
\end{minipage}}

Proof:

The proof makes sense but I didn't type it out because I was trying to process it instead. I was confused by $v_\epsilon$ and why this was used instead of just $y$ or something. I guess it is to distinguish different $v$ from one another based on $\epsilon$. At first I thought it was an index of some sort which confused me; I should realize that a subscript does \underline{not} always mean index.

Review: QUOD ERAT. This word was used in the proof but I don't know why.

Proof to the right.

\subsection{Example}

Let $S = \{\frac{1}{n}: n \in \mathbb{N}$. Show that inf(S) = 0

Proof: 0 is a lower bound for S.

Let $\epsilon > 0$. We need to show that $\exists v_\epsilon \in S$ with $v_\epsilon < 0 + \epsilon = \epsilon$

Let $n \in \mathbb{N}$ such that $n > \frac{1}{\epsilon}$ (This is possible by archimedean property)

$\Rightarrow \frac{1}{n} < \epsilon$ with $v_\epsilon = \frac{1}{n}$.

I don't fully understand this example. Review

\section{Density of $\mathbb{Q}$ in $\mathbb{R}$ }

\fbox{\begin{minipage}{\linewidth}
    Theorem - Let $x, y \in \mathbb{R}, x<r$. Then $\exists r \in \mathbb{R}$ such that $x < r < y$.

    Remark: This actually means that there are infinitely many rational numbers between $x$ and $y$. (Because theorem is "recursive")

    (Depending on perspective you see different density. Rational numbers are countable whil real numbers are not; this makes rational numbers seem not very dense. But there are infinite rational numbers between any two real numbers so the rational numbers are dense in real numbers.)

    We say that $\mathbb{Q}$ is dense in $\mathbb{R}$.
\end{minipage}}

Proof of theorem:

1. Case: $0 \leq x < y$

Let $b \in \mathbb{N}$ such that $\frac{1}{b} < y - x \Leftrightarrow b > \frac{1}{y - x} > 0$ (which is possible by archimedean property which means there is also a value greater than x so $y - x$ can always be greater than 0) .

Consider: $\{\frac{1}{b}, \frac{2}{b}, \frac{3}{b}\}$ 

Then $\frac{k}{b} \leq x \Leftrightarrow k \leq bx \geq 0$

\fbox{\begin{minipage}{\linewidth}
    Theorem - The set of all irrational numbers $\mathbb{R}/ \mathbb{R / Q}$ is dense in $\mathbb{R}$. (Definition dense review ?)

\end{minipage}}

Proof: Let $x, y \in \mathbb{R}, x < y$. Consider $\frac{x}{\sqrt{2}}< \frac{y}{\sqrt{2}}$ and apply previous theorem to these numbers.

Then $\exists r \in \mathbb{Q}$with $\frac{x}{\sqrt{2}} < r < \frac{y}{\sqrt{2}}$. If r = 0 pick one of the infinitely many other rational numbers. Review from textbook!

Next class: what happens when you intersect nested intervals. Second proof for uncountability of R based on this.




\end{document}
