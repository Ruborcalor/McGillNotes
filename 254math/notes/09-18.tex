
\documentclass{article}

\usepackage{amsmath}
\usepackage{amssymb}
\usepackage{enumerate}
\usepackage{parskip}
\usepackage{siunitx}
\usepackage{tikz}
\usepackage{upgreek}
\usepackage[margin=0.5in]{geometry}
\usepackage{graphicx}
\usepackage{esint}
\usepackage{nicematrix}
\NiceMatrixOptions{transparent}

\begin{document}

\title{Notes 2019-09-18}
\author{Cole Killian}

\maketitle

\section{Homework advice}

$\sqrt{2} + \sqrt{3} \Rightarrow$ irrational

$\sqrt{2}, \sqrt{3}$ are both irrational

people try to conclude that sum of irrational numbers is irrational, but this is not necessarily true.

ex. $\sqrt{2} - \sqrt{2} = 0$. 0 is rational.

\section{Beginning Class}
\fbox{\begin{minipage}{\linewidth}
  Theorem - Let $f: \mathbb{N} \to A, A$ a set st f is surjective, then A is countable.
\end{minipage}}

Proof:

Making use of the fact that f is surjective.

Let $a \in A$. Let $S_a \equiv f^{-1}(\{a\})$

Then $S_a \neq \varnothing$ because f is surjective. Also, $S_a \subset \mathbb{N}$ because f goes from N to A.

Then $S_a$ has a least element because all subsets of the natural numbers have a least element $n_a$.

Consder $S \equiv \{n_a: a \in A \} \subset \mathbb{N}$

Consider  $fl_s^{\text{"f restricted to s"}}: S \to A$.

Claim:  $fl_s: S \to A$ is a bijeciton

Prove surjectivity: $fl_s $ is surjective because $f(n_a) = a$ 

Prove injectivity: let  $f(n_{a_1}) = f(n_{a_2}) \Rightarrow a_1 = a_2 \Rightarrow n_{a_1} = n_{a_2}$

 $fl_s$ is surjective and injective so it is bijective.

 $\Rightarrow |S| = |A|$.

 S is countable as A is a subset of $\mathbb{N}$.

 $\Rightarrow$ A is countable

 Review the above proof. Don't completely understand the steps. Why does S have to be countable?


 \subsection{def}

 Def: Let A be a set. An \underline{enumeration without repetitions} of A is an ordered list $a_1, a_2, a_3, \dots$ st every element of A appears in this list exactly once.

 Def: An \underline{enumeration of A with rep} is an ordered list st every element of A appears at least once.

 ex. (1) [0, 1, -1, 2, -2, 3, -3, dots] is an enumeration of integers without repetition

 (2) 1, 1, 2, 1, 2, 3, 1, 2, 3, 4, dots is an enumeration of natural numbers with repetition

 Observations: (a) there is a one to one correspondence between bijections $f: \mathbb{N} \to A$ and enumerations of A without repetition.

 (b) there is a one-to-one correspondence between surjections $f: \mathbb{N} \to A$ and enumerations of A with repetitions


 Proof of (a): let $f: \mathbb{N} \to A$ be bijective. Let $a_1 = f(1), a_2 = f(2), \dots$. Then $a_1, a_2, \dots$ is an enumeration of A without repetition.

 Conversely, if $a_1, a_2, \dots$ is an enumeration without rep of A, then $f: \mathbb{N} \to A, n \to a_n$ is A bijection. because surjective and injective.

 Proof of (b): exactly the same with bijection replaced by surjection.

 \fbox{\begin{minipage}{\linewidth}
     Theorem - The set $\mathbb{Q}$ of rational numbers is countably infinite.
 \end{minipage}}

 Proof:

 step 1 is to prove that $\mathbb{Q}^+ \equiv \{x \in \mathbb{Q}: x > 0 \}$ is countably infinite.

 The goal is to list all rational numbers. Q is very dense so I would think that there would be way more rational numbers than natural numbers, but apparently they are on the same level of infinity.

 considering the following scheme of "rectangular" numbers.

first row   : $\frac{1}{1}, \frac{2}{1}, \frac{3}{1}, \frac{4}{1}, \dots$

second row: $\frac{1}{2}, \frac{2}{2}, \frac{3}{2}, \frac{4}{2}, \dots$

third row   : $\frac{1}{3}, \frac{2}{3}, \frac{3}{3}, \frac{4}{3}, \dots$

every positive rational number occurs in this scheme because every positive rational number can be written as one natural number divided by another natural number. It actually contains all rational numbers infinitely often because every rational number can be written infinitely many times as a quotient of natural numbers.

countain along the first row fails because you never get to the second row. Same idea with counting along columns. Brilliant idea is to count diagonally with slope $y = -x$.

Every element can be counted in finitely many steps. (Review: How does it make sense to count countably infinite set in finitely many steps?)

We now enumerate along the diagonals.

First diagonal: $\frac{1}{1}$, second: $\frac{2}{1}, \frac{1}{2}$, third: $\frac{3}{1}, \frac{2}{2}, \frac{1}{3}, \dots$. Note that the sum of the numerator and denom inator is constant for a row. This is an enumerated list with repetition.

"Every positive rational number appears on this list infinitely often. i.e. this list is an enumeration of $\mathbb{Q}^+$ \underline{with} repetition."

$\Rightarrow \mathbb{Q}^+$ is countable.

Cannot be finite because the natural numbers are infinite and the natural numbers are a subset of the rational numbers.

$\Rightarrow \mathbb{Q}^+$ is countably infinite. (Because given that the set is countable and infinite, the set must be countably infinite.)

Step number two: $\mathbb{Q}$ is countable. Let $a_1, a_2, a_3, \dots$ be an enumeration without repetition of $\mathbb{Q}^+$. We know this is possible because we proved that $\mathbb{Q}^+$ is countably infinite.

Then $0, a_1, a_{-1}, a_2, a_{-2}, \dots$ is an enumeration of $\mathbb{Q}$ without repetition and therefore $\mathbb{Q }$ is countably infinite.

\subsection{ Next: prooving that there is no bijection from natural numbers to real numbers meaning that real numbers are on another level of infinity. CANTOR came up with this theory.}

\fbox{\begin{minipage}{\linewidth}
    Theorem (CANTOR) $\mathbb{R}$ is \underline{not} countable.
\end{minipage}}

Def: A set "A" which is not countable is called \underline{uncountable}

Proof: We will prove that even the interval $[0, 1[$ is uncountable.

we'll prove this by contradiction and will use a fact about decimals (without proof): every real fraction can be written uniquely (one exception: $0.29 \ \text{repeating} \ 0.3 = 0.3000 \ \text{repeating} \ $ as a decimal fraction without repeating lines (without repeating lines in order to account for exception).

By this definition above, every $y \in [0, 1[$ has a unique representation where $y = 0. y_1 y_2 y_3 y_4 \dots: \forall n \in \mathbb{N}: y_n \in \{0, \dots, 9\}$ without repeating "something I can't read"

If we assume that $[0, 1[$ is countable, then we should be able to find an enumeration with repetition $x_1, x_2, x_3, \dots$. 

$x_1 = 0. x_{11}, x_{12}, x_{13}, \dots$

$x_2 = 0. x_{21}, x_{22}, x_{23}, \dots$

$x_3 = 0. x_{31}, x_{32}, x_{33}, \dots$

$\forall i,j, x_{ij} \in \{0, \dots, 9\}$

\subsection{Now to show the contradiction}

This is supposed to be a \underline{ complete} list of numbers in $[0, 1[$. Now to construct a number which is not on this list. 

Let $y = 0. y_1 y_2 y_3 \dots$ where 

 \[y_n \to
   \begin{cases} 
     1, & x_{nn} = 0\\
     0, & x_{nn} \neq 0\\
   \end{cases}
 \]

 Thus $\exists n \in \mathbb{N}: y = x_n$ but,

 $x_n = 0 . x_{n_1} x_{n_2}\dots x_{nn} \dots $
 $y   = 0 . y_{1} x_{2}\dots y_{n} \dots $ 

 but $x_{nn} \neq y_n$ so why is not equal to x

 This is a contradiction. $y \ni \{x_1, x_2, x_3, \dots\}$, so the assumption that $[0, 1[$ is false. this gives that  $\Rightarrow \mathbb{R}$ is uncountable. Therefore there is \underline{no} bijection from natural numbers to real numbers. Therefore real numbers cardinality must be larger than that of natural numbers. In the tutorials next week we will prove the following theorem:

 \fbox{\begin{minipage}{\linewidth}
     Theorem (CANTOR): Let A be a set and let $P(A)$, be the \underline{power set} of A (which means that it is the set of all subsets of A). Then there does not exist a bijection from A to $P(A)$. $\Rightarrow |A| \leq |P(A)|$. Consequence of this is that there are infinitely many levels of infinity.
 \end{minipage}}
 


 

\end{document}
