\documentclass[class=scrartcl, crop=false]{standalone}

\usepackage[sexy]{evan}
\usepackage{cole}


\date{2020-01-16}


\begin{document}

\section{Lecture 01-16}

\subsection{Selection Problem}

Given: $x_1, \dots, x_n \in \RR$ (pairwise distinct)
\\\\
Want: Find the $k$th smallest.
\\\\
Sorting leads to $\Theta_(n \log n)$ complexity. But we can do better.
\\\\
Algorithm of Blum et al.
\\\\
If $n \leq 5$, sort and exit. Cost $\leq 7$. Try it at home. Usually it can be done in 8, but with an extra trick you can do it in 7.
\\\\
Else
\begin{enumerate}
  \ii
  Divide all elements into groups of 5 and find the median. Cost $\leq n / 5 \times x$ where $x$ is the time to find the median of a group of 5 elements. It isn't obvious but it can be done in 6.
  \ii
  Find the median of $M$, recursively. Cost $\leq T_{n / 5}$
  \ii
  Compare all items with $m$, forming sets $L, R$. Cost $= n - 1$.
  \ii
  Case $k \leq |L|$, then find the  kth smallest in $L$. Cost is $T_{|L|} \leq T_{7n / 10}$
  \\
  Case $k = |L| + 1$, then return $m$. This has cost of 0.
  \\
  Case $k > |L| + 1$, find the $k - |L| - 1$ smallest in $R$. Cost is $T_{|R|} \leq T_{7n / 10}$
  
\end{enumerate} 
\begin{gather*}
  T_N \leq
  \begin{cases}
    7, & n \leq 5 \\
    T_{n / 5} + T_{7n / 10} + \frac{11}{5}n & n > 5 \\
  \end{cases} 
\end{gather*} 

Now to show inductively that $T_n \leq Cn$.
\begin{proof}
  Case where $n \leq 5$, then $7 \leq C$, so $C \geq 7$.
  \\\\
  Now assume that $T_n \leq Cn$ up to $n - 1$, where $n > 5$. 
  \begin{gather*}
    T_{n / 5} + T_{7n / 10} + \frac{11}{5}n \leq C(n / 5 + 7n / 10) + \frac{11}{5}n 
    = C\frac{9}{10}n + \frac{11}{5}n \leq C_n \\
    \Rightarrow \frac{11}{5} \leq \frac{C}{10} \Rightarrow C \geq 22
  \end{gather*} 
  An improvement can be made by return $L$ and $R$, so step 3 only has to check $\frac{4}{10}n - 1$ elements.
  \begin{gather*}
    C\frac{9}{10}n + \frac{8}{5}n \leq C_n \\
    \Rightarrow \frac{8}{5} \leq \frac{C}{10} \Rightarrow C \geq 16
  \end{gather*} 
\end{proof} 
There we go that is the linear time algorithm.

Tips for guessing the order of an algorithm when preparing to do induction: The subscripts on $T$ summed together are less than 1. So when building the recursion you'll see that the cost at each level decreases by $\frac{1}{10}$. So it's reasonable to assume that it might be linear.

\begin{note}
  $T_n =$ maximal cost on any input of size $\leq n$. This implies monotone.
\end{note} 

\subsection{Algorithm "FIND" (Hoare)}

Very similar to the above algorithm.
\\\\
Worst case of quick sort. $n + (n - 1) + (n - 2) + \dots + (n - (n - 1)) = \Theta(n^2)$ 
\\\\
$E\{T_n\} \leq \frac{1}{2}(ET_n + E_T{3n / 4}) + n$.
\\
So $ET_n \leq E_{8n / 4} + 2n \leq Cn \Rightarrow  \frac{3n}{4} \Rightarrow$ % revisit

\subsection{Mergesort Induction}

\begin{gather*}
  T_n \leq T_{\floor{(n / 2)}} + T_{\ceil{(n / 2)}} + n - 1
\end{gather*} 
Claim: $T_n \leq Cn\log_2(n)$
\begin{proof}
  \begin{gather*}
    \begin{cases}
      T_n \leq T_{\floor{(n / 2)}} + T_{\ceil{(n / 2)}} + n - 1 \\
      T_0 = 0, T_1 = 0
    \end{cases} 
  \end{gather*} 
  If $n = 1$, \cmark. \\\\
  If $n$ is even. $T_n \leq 2T_{n / 2} + n - 1 \leq 2C(n / 2)\log_2(\frac{n}{2}) + n - 1$
  \begin{gather*}
    = cn\log_2n - cn + n - 1 < n\log_2n \quad \forall C \geq 1
  \end{gather*} 
  If $n$ is odd. 
  \begin{gather*}
    T_n \leq T_{\frac{n + 1}{2}} + T_{\frac{n - 1}{2}} + n - 1 \\
    \leq C\frac{n + 1}{2}\log_2\frac{n + 1}{2} + C\frac{n - 1}{2}\log_2\frac{n - 1}{2} + n - 1 
    \\
    = C \frac{n}{2}\log_2\frac{n^2 - 1}{4} + \frac{C}{2}\log_2\frac{n + 1}{n - 1} + n - 1 \\
    = Cn\log_2n - Cn + \frac{C}{2} + n - 1 \\
    < n \log_2n
  \end{gather*} 
\end{proof} 

\subsection{Binary Search}

Given: Sorted array $x_1, \dots, x_n$.
\\\\
Find: $x$. Return either. 1. $x = x_i$ or 2. $x_i < x < x_{i + 1}$ 
\\\\
Model: Ternary Oracle. Two inputs, three possible outputs ($<$, $=$, $>$).
\\
Binary Oracle: $x \leq y$ or $x > y$.
\\\\
BinarySearch$(x, i, j)$
\\\\
Cases
\begin{enumerate}
  \ii 
  $i = j$. Ternary Oracle($x, x_i$). Exit with either $x = x_i$ or $x$ is not present.
  \ii
  $i > j$. This doesn't especially make sense. Exit with "$x$ not present"
  \ii
  $i < j$. Let $k = \floor{(\frac{i + j}{2})}$. Ternary oracle $(x, x_k)$.
  \begin{gather*}
    \begin{cases}
      \ \text{Return BinSearch $(x, i, k - 1)$} \ & x < x_k \\
      \ \text{Return "$x = x_k"$} \ & x = x_k \\
      \ \text{Return BinSearch $(x, k + 1, j)$} \ & x > x_k \\
    \end{cases} 
    \\\\
    T_n \leq 
    \begin{cases}
      T_{n / 2} + 1 & \ \text{n even} \ \\
      T_{(n - 1) / 2} + 1 & \ \text{n odd} \
    \end{cases}  \\
    T_1 = 1 \\
    T_0 = 0
  \end{gather*} 
\end{enumerate} 


\end{document}


