\documentclass[class=scrartcl, crop=false]{standalone}

\usepackage[sexy]{evan}
\usepackage{cole}


\date{2020-01-14}


\begin{document}

\section{Lecture 01-14}

\subsection{Recurrences}

Mergesort: $T_n = 2T_{n / 2} + n$. Roughly because emitting the floor function.

Binary Search:  $T_n = 2T_{n / 2} + 1$. Roughly because emitting the floor function.

Chip Testing: $T_n \leq T_{n / 2} + \frac{3n}{2}$.

Karatsuba Multiplication: $T_n = 3T_{n / 2} + n$. % revisit \\

Matrix Multiplication Strassen: $T_n = 7T_{n / 2} + n^2$

Halfspace Counting: $T_n = 3T_{n / 4} + 1$

\subsection{Solving Recurrences}

Methods include: 
\begin{enumerate}
  \ii
  Exact Methods
  \begin{enumerate}
    \ii
    Substitution. Summing a series.
    \ii
    Induction. Assume $T_n = f(n)$.
  \end{enumerate} 
  \ii
  Order of magnitude
  \begin{enumerate}
    \ii
    Master theorem: Gives $O(), \Theta(), \Sigma()$ result. Order of magnitude.
    \ii
    Recursion tree: to get insight.
  \end{enumerate} 
\end{enumerate} 

\begin{example}[Recursion Tree]
  \begin{gather*}
    T_n = 3T_{\frac{n}{4}} + n \\
    T_1 = 1 \ \text{or} \ 0, T_0 = 0 \\
    k \ \text{(height)} \ = \log_4(n) \\
    \frac{n}{4^k} = 1 \\
    n, \frac{3}{4}n, (\frac{3}{4})^2n, (\frac{3}{4})^3n, \dots
    Sum = n(1 + \frac{3}{4} + (\frac{3}{4})^2 + \cdots + (\frac{3}{4})^k) \\
    \approx = n(1 + \frac{3}{4} + (\frac{3}{4})^2 + \cdots) = \frac{1}{1 - \frac{3}{4}} = 4n \leq 4n \leq 4n + o(n) \\ %revisit + o(n)
    \sum_{i = 0}^{\infty}x^i = \frac{1}{1 - x} \\
    \Rightarrow T_n = \Theta(n)
  \end{gather*} 
  Most of the time is spent at the top level in this problem
  \\\\
  Changing the problem
  \begin{gather*}
    T_n = 4T_{\frac{n}{4}} + n \\
    n, n, n, n, n, \dots \\
    \Rightarrow T_n = \Theta(n\log n)
  \end{gather*} 
  The same amount of time is spent at each level in this problem \\\\
  Changing one more time
  \begin{gather*}
    T_n = 5T_{\frac{n}{4}} + n \\
    n, (\frac{5}{4})n, (\frac{5}{4})^2n, \dots \\
    Sum = n(1 + \frac{5}{4} + (\frac{5}{4})^2 + \cdots + (\frac{5}{4})^k) \\
    Sum = n(\frac{5}{4})^k(1 + \frac{4}{5} + (\frac{4}{5})^2 + \cdots + (\frac{4}{5})^k) \approx \frac{1}{(1 - \frac{4}{5})} = 5 \\
    \approx n(\frac{5}{4})^k*5 = 5 * 5^k = 5 * 5^{\log_4(n)} = 5n^{\log_4(5)} % revisit how the swap
  \end{gather*} 
  More time is spent at the bottom level in this problem
\end{example} 
Note that the coefficient in front of n wouldn't affect the magnitude.

\subsection{Master Theorem}
\begin{gather*}
  T_n = aT_{n / b} + f(n) \quad (T_n = constant, n \leq \square) \\
\end{gather*} 

Largest of $f(n), n^{\log_b(a)}$


\begin{enumerate}
  \ii
  \begin{gather*}
    \frac{n^{\log_b(a)}}{f(n)} \geq n^\epsilon \ \text{for some} \  \epsilon > 0 : T_n = \Theta(n^{\log_b(a)})
  \end{gather*} 
  \ii
  \begin{gather*}
    \frac{f(n)}{n^{\log_b(a)}} \geq n^\epsilon .... \quad T_n = \Theta(f(n))
  \end{gather*} 
  \ii
  \begin{gather*}
    f(n) = \Theta(n^{\log_b(a)}) : T_n = \Theta(f(n) \times \log(n))
  \end{gather*} 
\end{enumerate} 
Technical Condition
\begin{gather*}
  \limsup_{n \to \infty}\frac{af(\frac{n}{b})}{f(n)} < 1
\end{gather*} 

If $ T_n \leq aT_{n / b} + f(n) \quad (T_n = constant, n \leq \square) $, then $O()$ instead of $\Theta()$. If $\geq$, $\Sigma()$.

Other cases (exercises to think about):
\begin{gather*}
  T_n = T_{n / 2} + \log n \\
  T_n = T_{n / 2} + T_{n / 3} + n
\end{gather*} 

\subsection{Strassen's Matrix Multiplication}

For any $n$, there exists a power of 2 $\leq 2n$. So assume that $n$ is a power of 2 without loss of generality. Strassen gets rid of a multiplication.

\begin{gather*}
  T_n = 8T_{\frac{n}{2}} + n^2 = \Theta(n^3) \\
  \ \text{Became} \ \\
  T_n = 7T_{\frac{n}{2}} + n^2 = \Theta(n^{2.8})
\end{gather*} 

Best known to date: $\Theta(n^{2.374})$. Goal: $\Theta(n^2\log n)$.

\subsection{Halfspace Counting}

Given: $x_1, \dots, x_n \in \RR^2$. Users query: How many points on one side of $H$ (hyperplane).

Fact: There exists a partition of $x_1, \dots, x_n$ into 4 equal sets, using only 2 lines.

\begin{theorem}[Pancake Theorem]
  Take a shape. There is always a cut that cuts the pancake into two equal sets.
  \\\\
  Also works with overlapping pancakes. Can cut both in have with a single line.
\end{theorem} 

\end{document}
