\documentclass[class=scrartcl, crop=false]{standalone}

\usepackage[sexy]{evan}
\usepackage{cole}

\date{2020-01-06}


\begin{document}

\section{01-06}

http://www.math.mcgill.ca/ daimon /courses /algebra /algebra.html

Assignments: due on wednesday at burnside floor 10. Handed back on Monday

office hourse mw 10:35 - 11:45

Midterm feb 19

Friday: no lecture this week

Textbook: "Linear algebra and geometry" Kostakin \& makte

\begin{definition}[Linear Algebra]
  The study of vector spaces over a field and of the maps between them.

  Homomorphism aka linear transformation. Studying linear transformations between vector spaces.
\end{definition} 

Groups are an abstraction of the notion of symmetry.

Rings are an abstraction of the notion of numbers.

Vector spaces arose as a model of physical space.

\begin{example}
  Prototypical
  \begin{enumerate}
    \ii
    $\RR$
    \ii
    $\RR^2$ 
    \ii
    $\RR^3$
  \end{enumerate} 
\end{example} 

\subsection{Abstractions}

\begin{enumerate}
  \ii
  $\RR^n$. n-dimensional euclideon space
  \ii
  Replace $\RR$ by a general field $F \to F^n$ 
  \\
  Allow you to study some interesting and practical ideas.
\end{enumerate} 

\begin{definition}
  Fix a field $F$ (e.g. $\RR, \CC, \QQ, \FF_p = \ZZ / p \ZZ, \FF_{p^n} = \FF_p[x] / (q(x)) \deg(q) = n, q \text{ irreducible}$ 

  A vector space over $\FF$ is a set $V$ equipped with the following structures:
  \begin{enumerate}
    \ii
    A binary operation.
    \begin{gather*}
      +: V \times V \to V \\
      (v_1, v_2) \mapsto v_1 + v_2
    \end{gather*} 
    \ii
    A scalar multiplication
    \begin{gather*}
      \cdot: F \times V \to V \\
      (\lambda, v) \mapsto \lambda \cdot v
    \end{gather*} 
  \end{enumerate} 
  Subject to the following axioms.
  \begin{enumerate}
    \ii
    $(V, +)$ is an abelian group, i.e. $\exists$ a mutual (identity) element for +.
    \begin{enumerate}
      \ii Identity:
      \begin{gather*}
        0_V \ \text{such that} \ 0_V + w = w + 0_V = w \forall w \in V
      \end{gather*} 
      \ii Commutative:
      \begin{gather*}
        v_1 + v_2 = v_2 + v_1 \forall v_1, v_2 \in V
      \end{gather*} 
      \ii Inverses:
      \begin{gather*}
        \forall v \in V, \exists v' \ \text{such that} \ v + v' = v' + v = 0_V
      \end{gather*} 
      \begin{note}
        $v' = -v$
      \end{note} 
      \ii Associativity
      \begin{gather*}
        (v_1 + v_2) + v_3 = v_1 + (v_2 + v_3)
      \end{gather*} 
    \end{enumerate} 
    \ii
    Multiplication rules
    \begin{enumerate}
      \ii
      Identity
      \begin{gather*}
        1 \cdot v = v
      \end{gather*} 
      \ii
      Associativity
      \begin{gather*}
        \lambda_1, \lambda_2 \in F, v \in V \\
        \lambda_1(\lambda_2 v) = (\lambda_1\lambda_2)v
      \end{gather*} 
    \end{enumerate} 
    \ii
    Distributive Laws
    \begin{enumerate}
      \ii
      \begin{gather*}
        \lambda(v_1 + v_2) = \lambda v_1 + \lambda v_2
      \end{gather*} 
      \ii
      \begin{gather*}
        (\lambda_1 + \lambda_2)v = \lambda_1 v + \lambda_2 v
      \end{gather*} 
    \end{enumerate} 
  \end{enumerate} 
\end{definition} 

\subsection{Consequences of these axioms}

\begin{enumerate}
  \ii
  \begin{gather*}
    0 \cdot w = 0_V \\
    (0 + 0) \cdot w = 0 \cdot w + 0 \cdot w = 0 \cdot w \Rightarrow 0_V = 0 \cdot w
  \end{gather*} 
  \ii
  \begin{gather*}
    (-1) \cdot w = -w \\
    (1 + (-1) \cdot w = 1 \cdot w + (-1) \cdot w \\
    \Rightarrow 0_V = 0 \cdot w = w + (-1)w
  \end{gather*} 
\end{enumerate} 

\begin{example}
  \begin{enumerate}
    \ii
    Euclidean space $\RR^n$ is a vector space over $\RR$.
    \ii
    Solutions of linear equations
    \\
    Let $x_1, \dots, x_n, a_1, dots, a_n \in F$
    \\
    \begin{gather*}
      a_1x_1 + \cdots + a_nx_n = 0
    \end{gather*} 
    If $(x_1, \dots x_n)$ and $(x_1', \dots, x_n')$ are solutions to $(*)$, then so is $\lambda(x_1, \dots, x_n)$ and $(x_1, \dots, x_n) + (x_1' + \dots, x'_n)$
    \\
    More generally you can set up a series of these equations. Let $S$ be the set of solutions of this set of equations. It is a vector subspace of  $F^n$. Homogeneous vs non homogeneous
    \\
    Let $\sim{S}$ be solutions to (**) where replace 0 with constants.
    \\
    $\sim{S}$ is either empty, or it is a coset for $S$ in $F^n$. If $x_1^0, \dots, x_n^0) \in \sim{S}$, then $\sim{S} = (x_1^0, \dots, x_n^0) + S$
  \end{enumerate} 
\end{example} 

\subsection{Linear Differential Equations}

$a_0(x), a_1(x), \dots, a_n(x)$ functions from $\RR \to \RR$ 

\begin{gather*}
  f: \RR \to \RR \ \text{such that} \ 
  \\
  a_0(x)f(x) + a_1(x)f'(x) + \dots + a_n(x)f^{(n)}(x) = 0
  \\
  \frac{d}{dx}(f(x) + g(x)) = \frac{d}{dx}f + \frac{d}{dx}g
\end{gather*}

Note that this equation would not hold true when replacing addition with multiplication. Think of the product rule.



\end{document}
