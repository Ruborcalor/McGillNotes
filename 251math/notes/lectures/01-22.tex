\documentclass[class=scrartcl, crop=false]{standalone}

\usepackage[sexy]{evan}
\usepackage{cole}


\date{2020-01-22}


\begin{document}

\section{Lecture 01-22}

\subsection{Finite Dimensional}

$B = (v_1, \dots, v_n)$, a basis for $V$.
\\\\
If $v \in V$, then $\exists ! (x_1, \dots, x_n) \in F^n$ such that $v = x_1 v_1 + \cdots + x_n v_n$. (The exclamation points indicates uniqueness).
\\\\
The n-tuple  $(x_1, \dots, x_n)$ are called the coordinates of $v$ in $B$.
\\\\
This sets up an isomorphism between $V \simeq_B F^n$.
\\\\
Any vector space of dimension $n$ "is" $F^n$ (is non-canonically isomorphic to $F^n$). This non-canonicalness is reflected in the dependence on a basis.
\begin{note}
  If $(x_1, \dots, x_n)$ are the coordinates of $v$ relative to $B$, then 
  \begin{gather*}
    v = 
    \begin{pmatrix}
      v_1 & \cdots & v_n
    \end{pmatrix} 
    \begin{pmatrix}
      x_1 \\
      \vdots \\
      x_n
    \end{pmatrix} 
  \end{gather*} 
\end{note} 
If $T: V_1 \to V_2$ is a linear transformation, then $T$ can be described by a matrix $M_{T, B_1, B_2} \in M_{m \times n}$.
\begin{gather*}
  V_1 \to_T V_2 \\
  V_1 \simeq_{B_1} F^n_1 \\
  V_2 \simeq_{B_2} F^n_2 \\
  F^n_1 \to_{M_{T, B_1, B_2}} F^n_2
\end{gather*} 
Properties:
\begin{enumerate}
  \ii
  Let $B_1 = (v_1, \dots, v_n)$. $B_2 = (w_1, \dots, w_m)$ be bases for $V_1$ and $V_2$.
  \begin{gather*}
    T(v_1) = a{11}w_1 + a_{21}w_2 + \cdots + a_{m1}w_m \\
    T(v_2) = a{12}w_1 + a_{22}w_2 + \cdots + a_{m2}w_m \\
    \vdots \\
    T(v_n) = a{1n}w_1 + a_{2n}w_2 + \cdots + a_{mn}w_m \\
    M_{T, B_1, B_2} = (a_{ij})_{i \leq i \leq m, 1 \leq j \leq n} \\
    (T(v_1), T(v_2), \dots, T(v_n)) = (w_1, \dots, w_m) M_{T, B_1, B_2} \\
    T(B_1) = B_2 M_{T, B_1, B_2}
  \end{gather*} 
  \ii
  Effect of $T$ on coordinates
  \begin{gather*}
    v = 
    \begin{pmatrix}
      v_1 & \cdots & v_n
    \end{pmatrix} 
    \begin{pmatrix}
      x_1 \\
      \vdots \\
      x_n
    \end{pmatrix} = x_1v_1 + \cdots + x_nv_n \\
    T(v) = T(x_1v_1 + \cdots + x_nv_n) = x_1T(V_1) + \cdots + x_nT(v_n) =
    \begin{pmatrix}
      T(v_1) & \dots & T(v_n)
    \end{pmatrix} 
    \begin{pmatrix}
      x_1 \\
      \vdots \\
      x_n
    \end{pmatrix} \\ =
    \begin{pmatrix}
      (w_1, \dots, w_m) M_{T, B_1, B_2}
    \end{pmatrix} 
    \begin{pmatrix}
      x_1 \\
      \vdots \\
      x_n
    \end{pmatrix} =
    \begin{pmatrix}
      w_1, \dots, w_m
    \end{pmatrix} 
    \begin{pmatrix}
      M_{T, B_1, B_2}
    \end{pmatrix} 
    \begin{pmatrix}
      x_1 \\
      \vdots \\
      x_n
    \end{pmatrix} 
  \end{gather*} 
  \ii
  Conclusion:
  \\\\
  The column vector
  \begin{gather*}
    V_{T, B_1, B_2}
    \begin{pmatrix}
      x_1 \\
      \vdots \\
      x_n
    \end{pmatrix} 
  \end{gather*} 
  is a column vector of size $m$, and represents the coordinates of $T(v)$ in the basis $B_2$.
\end{enumerate} 

\subsection{Important Special Case of Transformation to itself}

$V_1 = V_2 = V$. $T: V \to V$. Choose $B = (v_1, \dots, v_n)$.
\\\\
$M_{T, B}$ is the matrix of $T$ relative to $B \in M_{n \times n}(F)$
\begin{gather*}
  (T(v_1), \dots, T(v_n)) = (v_1, \dots, v_n) M_{T, B}
\end{gather*} 
This gives an identification
\begin{gather*}
  \Hom_F(V, V) = \End_F(V) \simeq_B M_n(F)
\end{gather*} 

Dependency of $M_{T, B}$ on $B$. Let $B$ and $B'$  be two bases. Then there eixst unique matrices, $P, P'$ such that $B' = BP$.
\begin{gather*}
  B = (v_1, \dots, v_n) \\
  B' = (v_1', \dots, v_n') \\
  T(B) = B M_{T, B} \\
  T(B') = B' M_{T, B'} \\
  B' = BP \\
  T(BP) = B P M_{T, B} \\
  T((v_1, \dots, v_n)P) = (T(v_1), \dots, T(v_n))P \\
  T(B) P = B P M_{T, B'} \\
  B M_{T, B} P = B P M_{T, B'} \\
  M_{T, B} = P M_{T, B'}
\end{gather*} 

\begin{note}
  P is invertible
  \begin{proof}
    \begin{gather*}
      (v_1', \dots, v_n') = (v_1, \dots, v_n)P
      (v_1, \dots, v_n) = (v_1', \dots, v_n') P' \\
      \Rightarrow (v_1', \dots, v_n') = (v_1', \dots, v_n') P' P \\
      \Rightarrow P'P = E_{n \times n}
    \end{gather*} 
  \end{proof} 
\end{note} 

So
\begin{gather*}
  M_{T, B'} = P^{-1} M_{T, B} P
\end{gather*} 
\begin{definition}
  Matrices $v$ in $M_n(F)$ which are related by $M_1 = P^{-1}M_2 P$ for some $P \in M_n(F)^X$ are conjugate.
\end{definition} 

\begin{theorem}
  If $M_1$ and $M_2$ in $M_n(F)$ represent the same linear transformation $T: V \to V$ in different bases, they are conjugate.
  \\\\
  Even though the matrices are not unique, they are conjugate to one another based on the basis.
\end{theorem} 

\begin{exercise}
  What functions $\varphi M_n(F) \to F$ are invariant under conjugation.
  \begin{gather*}
    \varphi(A) = \varphi(P A P^{-1})
  \end{gather*} for all $P$ invertible.
\end{exercise} 


\end{document}
