\documentclass[class=scrartcl, crop=false]{standalone}

\usepackage[sexy]{evan}
\usepackage{cole}
\usepackage{MnSymbol}


\date{2020-02-03}


\begin{document}

\section{Lecture 02-03}

Question:

Calculate 
\begin{gather*}
  \#\{(V_1, V_2), \dim(V_1) = k_1, \dim(V_2) = k_2, \dim (V_1 \cap V_2) = d, V_1, V_2 \subseteq V\} 
\end{gather*} 
Given $k_1, k_2, d, \dim V = n, \#F = q$.
\\\\
New approach to solution. Let $d = 0$. Understand the set of linearly disjoint pairs $(V_1, V_2)$ with $\dim(V_1) = k_1, \dim(V_2) = k_2$.
\\\\
Number of possibilities for $V_1$ is
\begin{gather*}
  \begin{pmatrix}
    n \\ k_1
  \end{pmatrix}_q
  = 
  \frac{(q^n - 1)(q^{n - 1} - 1)(\cdots)(q^{n - k_1 + 1} - 1)}{(q^{k_1} - 1)(\cdots)(q - 1)}
\end{gather*} 
Next, the number of possibilities for $V_2$ once $V_1$ is chosen:
\begin{gather*}
  (q^n - q^{k_1})(q^n - q^{k_1 + 1})(\cdots)(q^n - q^{k_1 + k_2 - 1})
\end{gather*} 
Dividing by the possible bases for a single subspace of dim $k_2$.
\begin{gather*}
  \frac{(q^n - q^{k_1})(q^n - q^{k_1 + 1})(\cdots)(q^n - q^{k_1 + k_2 - 1})}{(q^{k_2} - 1)(q^{k_2} - q)(\cdots)(q^{k_2} - q^{k_2 - 1})}
  \\
  = \frac{q^{k_1 + (k_1 + 1) + (k_1 + 2) + \cdots + (k_1 + k_2 - 1)}}{q^{0 + 1 + 2 + \cdots + (k_2 - 1)}}
  \begin{pmatrix}
    n - k_1 \\
    k_2
  \end{pmatrix}_q
  \\\\
  = q^{k_1k_2}
  \begin{pmatrix}
    n - k_1 \\
    k_2
  \end{pmatrix}_q
\end{gather*} 

So
\begin{gather*}
  \#\{(V_1, V_2), \dim(V_1) = k_1, \dim(V_2) = k_2, \dim (V_1 \cap V_2) = 0, V_1, V_2 \subseteq V\} 
  \\
  =
  \begin{pmatrix}
    n \\ k_1
  \end{pmatrix}_q
  \begin{pmatrix}
    n - k_1 \\ k_2
  \end{pmatrix}_q
  q^{k_1 k_2}
\end{gather*} 

\begin{remark}
  \begin{gather*}
    \begin{pmatrix}
      n \\ k_1
    \end{pmatrix} 
    \begin{pmatrix}
      n - k_1 \\ k_2
    \end{pmatrix} 
  \end{gather*} is the number of disjoint subsets of cardinality $k_1$ and $k_2$ in a set of cardinality $n$.
\end{remark} 

Now to solve for general $d$.
\begin{lemma}
  The set $\{(V_1, V_2)$ of dim $(k_1, k_2)$ with $\dim(V_1 \cap V_2) = d$ is a natural bijection with the set of triples $\{(W, \ol{V_1}, \ol{V_2})$ where $W \subseteq V$, $\dim W = d$, 
      \\
      $\ol{V_1} \subseteq V / W$, $\dim \ol{V_1} = k_1 - d$ 
      \\
      $\ol{V_2} \subseteq V / W$, $\dim \ol{V_2} = k_2 - d$ 
      \\
      $\ol{V_3} \subseteq V / W$, $\dim \ol{V_3} = k_3 - d$ 
      \\
      $\ol{V_1}, \ol{V_2}$ are linearly disjoint.
      \begin{proof}
        \begin{gather*}
          (V_1, V_2) \mapsto (V_1 \cap V_2, V_1 \setminus W, V_2 \setminus W) \\
          (\pi^{-1}(\ol{V_1}), \pi^{-1}(\ol{V_2})) \leftmapsto (W, \ol{V_1}, \ol{V_2})
        \end{gather*} 
      \end{proof} 
\end{lemma} 

\begin{gather*}
  \#\Sigma = q^{(k_1 - d)(k_2 - d)}
  \begin{pmatrix}
    n \\ d
  \end{pmatrix}_q
  \begin{pmatrix}
    n - d \\ k_1 - d
  \end{pmatrix}_q
  \begin{pmatrix}
    n - k_1 \\ k_2 - d
  \end{pmatrix}_q
\end{gather*} 

Number of choices for $W$ = 
\begin{gather*}
  \begin{pmatrix}
    n \\ d
  \end{pmatrix}_q
\end{gather*} 
Number of choices for $(\ol{V_1}, \ol{V_2})$ given $W$ 
\begin{gather*}
  \begin{pmatrix}
    n - d \\ k_1 - d
  \end{pmatrix}_q
  \begin{pmatrix}
    n - k_1 \\ k_2 - d
  \end{pmatrix}_q
  q^{(k_1 - d)(k_2 - d)}
\end{gather*} 
Number of linearly disjoint spaces of dims $k_1, k_2$ in $\FF^n$ =
\begin{gather*}
  \begin{pmatrix}
    n \\ k_1
  \end{pmatrix}_q
  \begin{pmatrix}
    n - k_1 \\ k_2
  \end{pmatrix}_q
  q^{k_1k_2}
\end{gather*} 
Question 3 from homework.
\\\\
Show that if $T: V \to V$, $\dim V = n$, then $T$ satisfies a polynomial of degree $\leq n$.
\begin{gather*}
  p(x) = x^m + a_{m - 1}x^{n - 1} + \cdots + a_1x + a_0 \\
  p(T) = T^n + a_{m - 1}T^{m - 1} + \cdots + a_1T + a_0I \\
\end{gather*} 
This shows that the space generated by 
\begin{gather*}
  \underbrace{(1, T,  T^2, T^3, \dots)}_{\leq n} \subseteq \underbrace{\End(V)}_{\leq n^2}
\end{gather*} 

We show by induction of $n$ that if $W$ is any vector space of dimension $n$, $T: W \to W$ any endomorphism, then $\exists p(x), \deg(p(x) \leq n$, such that $p(T) = 0$.
\\\\
$n = 1$. $T: V \to V, T(v) = \lambda v, \lambda \in F$.
\\\\
Case 1.  $\exists v \in V$ such that $v_1, Tv, T^2 v, \dots, T^{n - 1}v$ span $V$.
\\\\
$-T^nv = a_0v + a_1Tv + a_2T^2v + \cdots + a_{n  -1}T^{n - 1}v$.
\\\\
$p(x) = x^n + a_{n - 1}x^{n - 1} + \cdots + a_1x + a_0$.
\\\\
$p(T)(v) = 0$. $T(p(T)(v)) = 0$.




\end{document}
