\documentclass[class=scrartcl, crop=false]{standalone}

\usepackage[sexy]{evan}
\usepackage{cole}


\date{2020-02-05}


\begin{document}

\section{Lecture 02-05}

Minimal polynomial. If $T: V \to V$, $\dim V = n$, then there exists polynomial $p(x)$ such that 
\begin{gather*}
  p(T) = 0 \\
  \deg(p(x)) \leq n
\end{gather*} 
Clear for $n = 1$.
\\\\
If $ \exists v \in V$ such that $(v, Tv, T^2v, \dots, T^{n - 1}v)$ are linearly independent, then it is fine.
\begin{gather*}
  \exists p(x) \ \text{such that} \ p(T)(v) = 0 \\
  \Rightarrow p(T)(Tv) = 0 \\
  p(T)(T^2v) = 0 \\
  \vdots \\
  p(T)(T^{n - 1}v) = 0 \\
  \Rightarrow p(T) = 0
\end{gather*} 

Suppose that there is no cyclic vector. Let $v \neq 0$ in $V$. Then 
\begin{gather*}
  \spn(v, T(v), \dots, T^{n - 1}(v)) = W \subsetneq V
\end{gather*} 
$W$ is preserved by $T$. 
\begin{gather*}
  T_W: W \to W
  \\
  \dim W = d < n
\end{gather*} 
The induction hypothesis implies that there is a polynomial $p_W(x)$ such that $p_W(T_W) = 0$.
\\\\
It would be nice if we could write $V = W \oplus W'$. Problem is that there need not be a $T$-stable complementary space $W'$.
\\\\
Solution: Define $W' = V / W$. $W'$ is naturally equipped with
\begin{gather*}
  \ol{T}: W' \to W' \\
  \ol{T}(v + W) = T(v) + W
\end{gather*} 
Checking well-defined:
\begin{gather*}
  v_1 + W = v_2 + W \Rightarrow v_1 - v_2 \in W \Rightarrow T(v_1 - v_2) \in T(W) \subseteq W \\
  \Rightarrow T(v_1) - T(v_2) \in W
\end{gather*} 
There is a polynomial $p_{W'}(x)$ such that $p_{W'}(\ol{T}) = 0$ where $\deg p_{W'}(x) \leq n - d$.
\\\\
Claim: $p(x) \coloneqq p_W(x) \cdot p_{W'}(x)$ satisfies  $p(T) = 0$.
\begin{proof}
  $p_{W'}(\ol{T}) = 0$. Image $p_{W'}(T) \subseteq W$.
  \begin{gather*}
    p_{W'}(\ol{T}) = 0 \Rightarrow p_{W'}(\ol{T})(v + W) = 0 \\
    \Rightarrow
    p_{W'}(T)(v) + W = 0 + W \ \forall x \in V \\
    \Rightarrow p_{W'}(T)(v) \in W \ \forall v \in V \\
    p_W(T)(W) = 0 \\
    p_W(T)p_{W'}(T) = p_W(T) \circ p_{W'}(T) = 0
  \end{gather*} 
\end{proof} 


Goal of linear algebra:
\begin{enumerate}
  \ii
  Given $T: V \to V$, classify all possible $T$.
  \ii
  Find bases for $V$ which are "convenient" to study $T$.
\end{enumerate} 
Structural invariants attached to $T$.
\begin{enumerate}
  \ii
  Minimal polynomial $p_T(x)$ 
  \ii
  Characteristic polynomial $f_T(x) = \det(xI - T)$. $\deg(f_T(x)) = n = \dim V$.
\end{enumerate} 

\begin{definition}[Eigenvalue]
  An element $\lambda \in F$ is an \ul{eigenvalue} for $T$ if $\exists$ a \ul{non-zero} $v \in V$ such that $T(v) = \lambda v$. A vector $v$ with this property is called an eigenvector of  $T$, with eigenvalue $\lambda$. Note that the zero vector is never considered an eigenvector.
\end{definition} 

\begin{definition}[Eigenspace]
  The set $V_\lambda = \{v \in V$ such that $T(v) = \lambda v\}$ is called the eigenspace for $T$.
\end{definition} 

\begin{definition}[Spectrum]
  The spectrum of $T$ is the collection of eigenvalues of $T$. $spec(T) \subseteq F$.
\end{definition} 

\begin{proposition}
  If $\lambda_1 \neq \lambda_2 \in spec(T)$, then $V_{\lambda_1}$ and $V_{\lambda_2}$ are linearly disjoint, i.e. $V_{\lambda_1}\cap V_{\lambda_2} = (0)$.
  \\
  \begin{proof}
    If $v \in V_{\lambda_1}\cap V_{\lambda_2}$, then $T(v) = \lambda_1v, T(v) = \lambda_2v \Rightarrow (\lambda_1 - \lambda_2)v = 0$. $\lambda_1 - \lambda_2 \neq 0 \Rightarrow v = 0$.
  \end{proof} 
\end{proposition} 

\begin{definition}[Diagonalizable]
  \begin{gather*}
    \ \text{If} \ \oplus_{\lambda \in spec(T)}V_\lambda = V
  \end{gather*} 
  then $T$ is \ul{diagonalizable}.
  \\\\
  Equivalently, $T$ is diagonalizable if $V$ has a basis of eigenvectors for $T$.
\end{definition} 

\begin{example}
  \begin{enumerate}
    \ii
    $V = \RR^2$, where $T$ is a rotation by $\pi / 2$.
  \end{enumerate} 
\end{example} 

\end{document}
