\documentclass[class=scrartcl, crop=false]{standalone}

\usepackage[sexy]{evan}
\usepackage{cole}


\date{2020-01-17}


\begin{document}

\section{Lecture 01-17}

\subsection{Constructions of Vector Spaces}

Given vector spaces $V_1, V_2$, we can construct new vector spaces.
\begin{enumerate}
  \ii
  Direct sum, or cartesian product. $V_1 \times V_2 = V_1 \oplus V_2 = \{(v_1, v_2) \ : \ v_1 \in V_1, v_2 \in V_2\}$.
  \begin{gather*}
    (v_1, v_2) + (v_1', v_2') = (v_1 + v_1', v_2 + v_2') \\
    \lambda(v_1, v_2) = (\lambda v_1, \lambda v_2)
  \end{gather*} 
  Note that $\dim(V_1 \oplus V_2) = \dim(V_1) + \dim(V_2)$
  \ii
  Subspace. If $V$ is a vector space over $F$, then $W \subseteq V$ is a subspace if it is closed under addition and scalar multiplication.
   \begin{gather*}
    w_1, w_2 \in W \Rightarrow w_1 + w_2 \in W \\
    \lambda \in F,\  w \in W \Rightarrow \lambda w \in W
  \end{gather*} 
  Conclusion is that $W$ is a vector space. Other properties are inherited from $V$.
  \ii
  Homs. $\hom_F(V_1, V_2)$ is a vector space. If $V_1$ and $V_2$ are finite dimensional, dimensions $n_1$ and $n_2$, then $\dim_F(\hom_F(V_1, V_2)) = n_1n_2$. % revisit better understand hom from one vector space to another.
  \\\\
  Let $(e_1, \dots, e_n)$ be a basis for $V_1$.
  \\\\
  Key remark: A linear transformation $T: V_1 \to V_2$ is completely determined by $(T(e_1), \dots, T(E_{n_1})$.
  \\\\
  Why? If $v \in V_1$, then $v = \lambda_1 e_1 + \cdots + \lambda_{n_1} e_{n_1}$. So $T(v) = T(\lambda_1 e_1 + \cdots + \lambda_{n_1}e_{n_1} = \lambda_1 T(e_1) + \cdots + \lambda_{n_1} T(e_{n_1})$
  \\\\
  \begin{gather*}
    \hom(V_1, V_2) = \underbrace{V_2 \oplus \cdots \oplus V_2}_{n_1 \ \text{times} \ } % revisit what is the image and kernel?
  \end{gather*} 
  \ii
  Dual space: $V^* = \hom_F(V, F)$. If $B$ is a basis for $V$, then $V \simeq F_0(B, F)$ . $V^* \simeq F(B, F)$.
  \\\\
  The choice of $B$ determines an injection $V \hookrightarrow V^*$. When $B$ is finite, i.e. $\dim(V) = n < \infty$, then $V \simeq V^*$.
  \\\\
  There is a canonical inclusion of $V \hookrightarrow V^{**}$.
  \begin{gather*}
    V \to V^{**} \\
    v \mapsto v^{**}(l) = l(v) \\
    l \in V^*, \ l: V \to F
  \end{gather*} 
  $v^{**}$ is a linear functional on $V^*$. I.e. a linear transformation from $F^* \to F$. The rule $v \mapsto v^{**}$ is itself linear. i.e. $(v_1 + v_2)^{**} = v_1^{**} + v_2^{**}$.
  \ii
  The tensor product of $V_1$ and $V_2$. $V_1 \otimes V_2 = \hom_F(V_1^*, V_2)$ % revisit. Homomorphism is a linear transformation? What is an isomorphism?
  \\\\
  If $V_1$ and $V_2$ are finite dimensional, then $\dim(V_1 \otimes V_2) = \dim(V_1)\dim(V_2)$.
  \\\\
  The dimension of the dual space.
  \begin{gather*}
    V \simeq F_0(B, F) \\
    V^* \simeq F(B, F)
  \end{gather*} 
  If $\#B = n < \infty$, then $\dim(V^*) = n = \dim(V)$.
  \ii
  Quotients. If $W \subseteq V$ is a subspace, then $W$ is also a subgroup.
  \begin{gather*}
    V / W = \{v + W, v \in V\}
  \end{gather*} 
  $V / w$ has a natural scalor multiplication.
  \begin{gather*}
    \lambda \in F, \ \lambda(v + W) = \lambda v + W
  \end{gather*} 
  \begin{exercise}
    With this definition of scalar multiplication, the group $V / W$ satisfies all the axioms of a vector space over $F$.
  \end{exercise} 
\end{enumerate} 

\subsection{Isomorphism Theorem}

If $T: V \to V'$ is a linear transformation.
\begin{definition}[Kernel]
  \begin{gather*}
    \ker(T) \coloneqq \{v \in V \ \text{s.t.} \ T(v) = 0\}
  \end{gather*} 
\end{definition} 
\begin{definition}[Image]
  \begin{gather*}
    \Ima(T) = \{v \in V' \ \text{s.t.} \ \exists \tilde{v} \in V \ : \ T(\tilde{v}) = v\}
  \end{gather*} 
\end{definition} 
Claim: $\ker(T)$ and $\Ima(T)$ are vector spaces.

\begin{proof}
  Show $v \in \ker(T), \ \lambda \in F$, \ $\lambda w \in \ker(T)$. $T(\lambda v) = \lambda T(v) = \lambda \cdot 0 = 0$. Likewise for image.
\end{proof}  % revisit canonical
If $T: V \to V'$ is linear, $W = \ker(T)$, then $T$ induces an isomorphism $V / \ker(T)$ to $\Ima(T)$.
\begin{gather*}
  \varphi: V / \ker(T) \to \Ima(T) \\
  v + \ker(T) \mapsto T(v)
\end{gather*} 
Need to check:
\begin{enumerate}
  \ii
  That $\varphi$ is well-defined. $W = \ker(T)$. If $v + w = v' + W$, then $T(v) = T(v')$.
  \ii
  Injection
  \ii
  Surjection
\end{enumerate} 


\end{document}
