\documentclass[class=scrartcl, crop=false]{standalone}

\usepackage[sexy]{evan}
\usepackage{cole}


\date{2020-03-09}


\begin{document}

\section{Lecture 03-09}

A quadratic space is a pair $(V, <, >)$ where $V$ is a vector space, and $<, >: V \times V \to F$ which is bilinear.

The pairing $(<, >)$ is non-degenerate if it induces an injection

\begin{gather*}
  V \to V^* \\
  v \mapsto (w \mapsto <v, w>) \\
  (\text{when dim V } < \infty, \ \text{then} \  V \simeq V^*)
\end{gather*} 

The adjoint of $T: V \to V$ is the map satisfying
\begin{gather*}
  T*: V \to V  \\
  <v, Tw> = <T^*(v), w>
\end{gather*} 

Question: Where do non-degenerate bilinear forms arise "in nature"?

Answer: Geometry, distance.

From now on, $F = \RR$ or $\CC$.

\begin{definition}
  A \ul{real inner product} on $V$ is a bilinear form satisfying:
  \begin{enumerate}
    \ii
    \begin{gather*}
      <v, w> = <w, v>, \quad \forall v, w \in V
    \end{gather*} 
    \ii
    \begin{gather*}
      <v, v> \geq 0, \quad <v, v> = 0 \ \text{iff} \ v = 0
    \end{gather*} 
  \end{enumerate} 
\end{definition} 

\begin{example}
  \begin{enumerate}
    \ii
    $V = \RR^n$
     \begin{gather*}
      <(x_1, \dots, x_n), (y_1, \dots, y_n)> = x_1y_1 + x_2y_2 + \cdots + x_ny_n \\
      <(x_1, \dots, x_n), (x_1, \dots, x_n)> = x_1^2 + x_2^2 + \cdots + x_n^2 \\
    \end{gather*} 
    \ii
    $V = e([0, 1])$ represents continuous real-valued functions on $[0, 1]$.
    \begin{gather*}
      <f, g> = \int_0^1f(t)g(t)dt, \quad <f, f> = \int_0^1f(t)^2dt
    \end{gather*} 
  \end{enumerate} 
\end{example} 

\begin{definition}[Complex Inner Product]
  A \ul{complex inner product} on $V$ is a hermition-bilinear form satisfying
  \begin{note}
    It would become problamatic to try and declare it as a standard bilinear form
  \end{note} 
  \begin{enumerate}
    \ii
    \begin{gather*}
      <v, \lambda w_1 + w_2> = \ol{\lambda}<v, w_1> + <v, w_2>
    \end{gather*} 
    \ii
    \begin{gather*}
      <v, w> = \ol{<w, v>}
    \end{gather*} 
    \ii
    \begin{gather*}
      <v, v> \in \RR \geq 0, \quad <v, v> = 0 \Leftrightarrow v = 0
    \end{gather*} 
  \end{enumerate} 
  
\end{definition} 

\begin{example}
  Reviewing the previous examples with the new complex inner product
  \begin{enumerate}
    \ii
    $V = \CC^n$
    \begin{gather*}
      <(x_1, \dots, x_n), (y_1, \dots, y_n)> = x_1\ol{y_1} + x_2\ol{y_2} + \cdots + x_n\ol{y_n} \\
      <(x_1, \dots, x_n), (x_1, \dots, x_n)> = |x_1|^2 + |x_2|^2 + \cdots + |x_n|^2 \\
    \end{gather*} 
    \ii
    $V = e([0, 1])$ represents continuous complex-valued functions on $[0, 1]$.
    \begin{gather*}
      <f, g> = \int_0^1f(t)\ol{g(t)}dt, \quad <f, f> = \int_0^1|f(t)|^2dt
    \end{gather*} 
  \end{enumerate} 
\end{example} 

\begin{note}
  Caveat: A complex inner product space is not (quite) a quadratic space as defined before.

  We define the norm of $v$ to be $||v|| = \sqrt{<v, v>}$. "Length of $v$".
\end{note} 

\begin{example}
  \begin{enumerate}
    \ii
    $V = \RR^n$.
    \begin{gather*}
      ||(x_1, \dots, x_n)|| = \sqrt{x_1^2 + \cdots + x_n^2}
    \end{gather*} 
    \ii
    $V = \CC^n$.
    \begin{gather*}
      |(z_1, \dots, z_n)|| = \sqrt{|z_1|^2 + |z_2|^2 + \cdots + |z_n|^2}
    \end{gather*} 
  \end{enumerate} 
\end{example} 

\begin{definition}[Properties of $||\ ||$]
  Always easier to think about the square of the norm.
  \begin{enumerate}
    \ii
    \begin{gather*}
      ||v + w||^2 = <v + w, v + w> = <v, v> + <v, w> + <w, v> + <w, w> \\
      = ||v||^2 + 2 \ \text{Real part of} \ <v, w> + ||w||^2
    \end{gather*} 
  \end{enumerate} 
\end{definition} 

\begin{definition}
  Two vectors $v, w$ are \ul{orthogonal} if $<v, w> = 0$.
\end{definition}

\begin{theorem}[Pythagorean Theorem]
  \begin{gather*}
    ||v + w||^2 + ||v||^2 + ||w||^2
  \end{gather*} 
\end{theorem} 


\begin{theorem}[Parallelogram Law]
  \begin{gather*}
    ||v + w||^2 + ||v - w||^2 = 2(||v||^2 + ||w||^2) \\
  \end{gather*} 
  \begin{proof}
    \begin{gather*}
      ||v + w||^2 + ||v - w||^2 = ||v||^2 + 2Re<v, w> + ||w||^2 + ||v||^2 - 2Re(<v, w>) + ||w||^2 \\
      = 2(||v||^2 + ||w||^2)
    \end{gather*} 
  \end{proof} 
\end{theorem} 

\begin{theorem}[Polarization Formula]
  The function $v \mapsto <v, v>$ is enough to recover $(v, w) \mapsto <v, w>$.
  \begin{enumerate}
    \ii
    If $F = \RR$
    \begin{gather*}
      <v, w> = 1/2(<v + w, v + w> - <v, v> - <w, w>) \\
      <v, w> = \frac{1}{4}(<v + w, v + w> - <v - w, v - w>) % why is this true?
    \end{gather*} 
    \ii
    If $F = \CC$ 
    \begin{gather*}
      <v, w> = <v + w, v + w> \\
      + i<v + iw, v + iw> \\
      + -1<v - w, v - w> \\
      + -i<v - iw, v - iw>
    \end{gather*} 
  \end{enumerate} 
\end{theorem} 

\begin{theorem}[Cauchy Schwarz Inequality]
  \begin{gather*}
    |<v, w>|^2 \leq ||v||^2||w||^2
  \end{gather*} 
\end{theorem} 

\end{document}

