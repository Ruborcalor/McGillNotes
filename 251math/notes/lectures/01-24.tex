\documentclass[class=scrartcl, crop=false]{standalone}

\usepackage[sexy]{evan}
\usepackage{cole}


\date{2020-01-24}


\begin{document}

\section{Lecture 01-24}

Recommend Colmez. Drawback is that it is in french.

Let $V$ be a finite dimensional vector space. Let $B$ be a basis for $V$. $B = (v_1, \dots, v_n) \in V^n$. $v \in V$ has coordinates  
\begin{gather*}
  x = 
  \begin{pmatrix}
    x_1 \\
    \vdots \\
    x_n
  \end{pmatrix} \in F^n
\end{gather*} if $v = Bx$ 
\\\\
If $T$ is a linear transformation,
\begin{gather*}
  T: V \to V \\
  V \simeq_B F^n_1 \\
  V \simeq_{B'} F^n_2 \\
  F^n_1 \to_{M_{T, B}} F^n_2
\end{gather*} % insert image

\begin{fact}
  If $B$ and $B'$ are different bases for $V$, then the matrices $T_{T, B}$ and $T_{T, B'}$ are \ul{conjugate}. i.e. $\exists P \in \GL_n(F)$ such that $M_{T, B'} = PM_{T, B}P^{-1}$
\end{fact} 

\subsection{Determinant}

\begin{proposition}
  There is a unique function $\det: M_n(F) \to F$ satisfying:
  \begin{enumerate}
    \ii
    $\det$ is \ul{multilinear}. i.e. it is a linear function in each row with all other rows being fixed.
    \ii
    $\det$ is \ul{alternating}, namely, the determinant changes sign after interchanging two rows.
    \begin{gather*}
      \det(M^\sigma) = \sign(\sigma)\det(M), \sigma \in S_n
    \end{gather*} 
  \end{enumerate} 
\end{proposition} 

\subsection{Proof of existence and uniqueness}

\begin{gather*}
  \det(AB) = \det(A)\det(B) \\
  \det(A + B) = ??? \\
  \det(PAP^{-1}) = \det(P)\det(A)\det(P)^{-1} = \det(A)
\end{gather*} 

\begin{definition}
  The determinant of $T: V \to V$ is the determinant of \ul{any} matrix representing $T$.
\end{definition} 

\begin{definition}[Trace]
  Trace $(A) = a_{11} + a_{22} + a_{33} + \cdots + a_{nn}$ where $A = (a_{ij})$. \\
  \begin{gather*}
    \Tr(A + B) = \Tr(A) + \Tr(B) \\
    \Tr(AB) = ??
  \end{gather*} 
\end{definition} 

\begin{lemma}
  \begin{gather*}
    A \cdot B = \sum_{i, j} a_{ij}b_{ij} \\
    \Tr(AB) = A \cdot B^T = \sum_{i, j}a_{ij}b_{ji} \\
    \Tr(BA) = B \cdot A^T = \sum_{i, j}b_{ij}a_{ji} \\
    \Tr(AB) = \Tr(BA) \\
    \Tr(P A P^{-1}) = \Tr(A P^{-1} P) = \Tr(A)
  \end{gather*} 
  So trace is also invariant over conjugation.
\end{lemma} 

\begin{definition}
  The trace of $T: V \to V$ is the trace of \ul{any} matrix representing $T$.
\end{definition} 

\begin{exercise}
  Show that $\End_F(V) = \End_F(V)$ \\
  \begin{enumerate}
    \ii
    First show that $M_n(F) \simeq M_n(F)^*$ where $A \mapsto (x \mapsto \Tr(AX))$.
    \ii
    Then show that $\End_F(V) \simeq \End_F(V)^*$. Solution the mapping $T \mapsto (U \mapsto \Tr(TU))$.
  \end{enumerate} 
\end{exercise} 

If $T: V \to V$ is a linear transformation, study the structure of $T$ acting on $V$ (nullspace, eigenspaces, eigenvalues, characteristic polynomial, minimal polynomial.)

\begin{gather*}
  F[T] = \{a_0 + a_1T + a_nT^n + \cdot\} \in \End_F(V) \subseteq \End_F(V)
\end{gather*} 
$F[T]$ is a sub F-algebra of $\End_F(V)$.
\begin{remark}
  If $\dim(V) > 1$, then $F[T] \neq \End_F(V)$.
  \\\\
  $F[T]$ is a quotient ring of $F[x]$, the ring of polynomials. Ther eis a natural ring homomoprhism
  \begin{gather*}
    \varphi_T: F[x] \to F[T] \subseteq \End(V) \\
    p(x) \mapsto P(T)
  \end{gather*} 
  But $F[X]$ is infinite dimensional. So this means that there is a nontrivial kernal because $F[T]$ is not infinite dimensional.
\end{remark} 

\begin{definition}[Defining Ideal]
  The kernel of $\varphi_T$ is called the \ul{defining ideal} of $T$.
  \\\\
  $I_T = \ker(\varphi_T)$.
  \\\\
  $I_T$ is generated by a unique polynomial in $F[x]$ which is monic. $I_T = (P_T(X))$. 
  \begin{gather*}
    P_T(X) = X^ + a_{m - 1}x^{m - 1} + \cdots + a_0, \quad a_j \in F
  \end{gather*} 
\end{definition} 

What is $P_T(x)$?
\begin{gather*}
  \varphi_T(p_t) = 0 \\
  \varphi_T(T) = 0
\end{gather*} 
$P_T$ is called the minimal polynomial. $f \in F[x]$. $f(T) = 0$. $p_T(x) | f(x)$.
\end{document}
