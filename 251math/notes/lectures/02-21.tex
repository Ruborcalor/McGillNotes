\documentclass[class=scrartcl, crop=false]{standalone}

\usepackage[sexy]{evan}
\usepackage{cole}


\date{2020-02-21}


\begin{document}

\section{Lecture 02-21}

\begin{theorem}
  If $M$ is a symmetric $n \times n$ matrix with real entries, then $M$ is diagonalizable. 
\end{theorem} 

If $M$ is symmetric, then $M = (a_{ij})_{i, j = 1, \dots, n}, \quad a_{ij} = a_{ji}$.

Language for approaching this result: self adjoint operators on inner product spaces.

\subsection{Duality}
$V$ vector space. $V^*$ is the space of linear functionals $V \to F$.
\\\\
If $T: V_1 \to V_2$ is a linear transformation, then it induces 
\begin{gather*}
  T^*: V_2^* \to V_1^* \\
  T^*(l) = l \circ T \\\\
  V_1 \to_{T_1} V_2 \to_{T_2} V_3 \\
  (T_2 \circ T_1)^* : V_3^* \to V_1^* \\
  (T_2 \circ T_1)^* = T_1^* \circ T_2^* \\\\
  l \in V_3^* \\
  (T_2 \circ T_1)^*(l) = l \circ (T_2 \circ T_1) = (l \circ T_2) \circ T_1 \\
  = [T_2^*(l)] \circ T_1 = T_1^*(T_2^*(l)) = T_1^* \circ T_2^*(l)
\end{gather*} 

\begin{lemma}
  \begin{enumerate}
    \ii[]
    \ii
    If $T: W \to V$ is injective, then $T^*: V^* \to W^*$ is surjective.
    \ii
    If $T: V \to W$ is surjective, then $T^*$ is injective.
  \end{enumerate} 
  \leavevmode
  \begin{proof}
    \begin{enumerate}
      \ii[]
      \ii
      If $T$ is injective, then it realises co inclusion of $W$ into $V$ and 
      \begin{gather*}
        T^*(l) = l|_{\Ima(T) = W}
      \end{gather*} 
      Surjectivity of $T^*$ means that given $l_0: \Ima(T) \to F$, $\exists$ an extension $l: V \to F$ such that $l|_W = l_0$. After choosing a complementary $W'$ such that $W \oplus W' = V$, we let $l(w + w') = l_0(w)$.
      \ii
      If $T: V \to W$ is surjective, then $\ker(T^*) = \{l: W \to F \ \text{such that} \ l \circ T = 0\}$ 
      \begin{align*}
        l \circ T = 0 & \Leftrightarrow l \circ T(v) = 0 \quad & \forall v \in V \\
                      & \Leftrightarrow l(T(v)) = 0 \quad & \forall v \in V \\
        & \Leftrightarrow l(w) = 0 \quad    & \forall w \in \Ima(T) \\
        & \Leftrightarrow l(w) = 0 \quad    & \forall w \in W
      \end{align*} 
      So $\ker(T^*) = 0 \Rightarrow T^*$ is injective.
    \end{enumerate} 
    \leavevmode \\
    If $W$ is a subspace of $V$, then $W^*$ is a quotient of $V^*$. If $W$ is a quotient of $V$, then $W^*$ is a subspace of $V^*$.
    \begin{gather*}
      W^* = \{l: V \to F \ \text{such that} \ l|_{\ker(V \to W) = 0}\}
    \end{gather*} 
  \end{proof} 
\end{lemma} 

GIven a $W \subseteq V$, there is a canonical subspace of $V^*$ attached to $W$, 
\begin{gather*}
  W^{\perp} = \ker(V^* \to W^*) \\
  W^{\perp} = \{l: V \to F \ \text{such that} \ l(W) = 0\}
\end{gather*} 
The assignment $W \mapsto W^{\perp}$ sets up an inclusion reversion bijection between subspaces of $V$ and subspaces of $V^*$.
\begin{gather*}
  W \Leftrightarrow W^{\perp} \\
  0 \Leftrightarrow V^* \\
  V \Leftrightarrow 0
\end{gather*} 

Claim: $\dim(W) + \dim(W^{\perp}) = \dim(V) = \dim(V^*)$.

Caveat: $W \oplus W^{\perp}$ does not make sense.
\begin{proof}
  \begin{gather*}
    i: V \hookrightarrow V \\
    i^* : V^* \to W^* \\
    W^{\perp} = \ker(i^*: V^* \to W^*)
  \end{gather*} 
\end{proof} 

\subsection{Rank-nullity Theorem}
\begin{gather*}
  \dim(W^\perp) + \dim(W^*) = \dim(V^*) \\
  \dim(W^\perp) + \dim(W) = \dim(V)
\end{gather*} 
If $W \subseteq V^*$, then $W^\perp \subseteq V$. $W^\perp = \{v \in V \ \text{such that} \ l(v) = 0, \quad \forall l \in W\}$.

\end{document}
