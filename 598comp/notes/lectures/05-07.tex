\documentclass[../598comp.tex]{subfiles}

\graphicspath{ {./lectures/images/}{./images/} }

\date{05-07}

\begin{document}

\section{05-07}

\begin{lemma}[Pumping Lemma]
  \begin{gather*}
    L = \{a^nb^n \ | \ n \geq 0\}
  \end{gather*}
  Not possible to recognize this language with finite automaton because it would
  require unbounded memory. Non regular languages could recognize this.
  \\\\
  Suppose we have putative (generally considered or reputed to be) DFA that recognizes $L$. Then we can pick a word
  with a block of $a$ with length greater that the number of states in the
  machine. This means that the machine will have to repeat a state. This rests
  on the pigeon hole principle.
  \\\\
  The idea is that now you can exploit this loop as many times as you'd like by
  inserting the string that brings you about the loop.
  \\\\
  Now for the lemma. Let $L$ be a regular language. Then
  \begin{gather*}
    \exists p > 0 \ \text{such that} \ \forall w \in \Sigma^* \ | \ w \in L
    \wedge |w| \geq p \\
    \exists x, y, z \in \Sigma^* \ \text{such that} \ w = xyz, \ |xy| \leq p, \
    |y| > 0 \\
    \forall i \in \NN, \ xy^iz \in L
  \end{gather*}
  An intuition for this is that $y$ is the word that takes you through the loop,
  so you can repeat it as many times as you'd like.
  \begin{proof}
    Given a DFA for $L$ choose $p$ to be strictly greater than the number of
    states in the DFA. If $|w| \geq p \wedge w \in L$ then as the automaton
    changes state it must hit the same state twice while reading the first $p$
    letters, when $p$ is greater than the number of states in the machine.
  \end{proof}
\end{lemma}

\begin{definition}[Finite Language]
  A \ul{finite language} is one containing a finite number of words.
\end{definition}

\begin{fact}
  Every finite language is regular. The $p$ that you choose is longer than any
  word in the language. So then a finite language would have no words of length
  greater than $p$.
\end{fact}

\begin{fact}
  $L$ regular $\Rightarrow$ $L$ can be pumped. Contrapositive is that $L$ cannot
  be pumped $\Rightarrow L$ not regular.
  \\\\
  Note that it is \ul{not} true that $L$ can be pumped $\Rightarrow L$ regular.
\end{fact}

\begin{lemma}[Pumping Lemma Contrapositive]
  \begin{gather*}
    L \subseteq \Sigma^* \ s.t. \ \forall p > 0 \\
    \exists w \in L \ with \ |w| \geq p \ s.t. \\
    \forall x, y, z \in \Sigma^* \ with \ w = xyz, \ |xy| \leq p \wedge |y| > 0
    \\
    \exists i \in \NN \ s.t. \ xy^iz \notin L \\
    \Rightarrow L \ \text{is not regular}
  \end{gather*}
  Use games to deconstruct this statement. You and the devil. You play the
  $\exists$ quantifiers, and the devil plays the $\forall$ quantifiers. You must
  come up with a strategy to \ul{win every game}.
  \\\\
  The obvious first move is represented by a \ul{symbolic} $p$. Your first move
  is \ul{explicit}. The devil's move in step 3 must be analyzed by an exhaustive
  case analysis. Your last move must specify a response for \ul{all} cases.
\end{lemma}

\begin{example}
  $L = \{a^nb^n \ | \ n \geq 0\}$
  \begin{enumerate}
  \item 
    Demon choses $p > 0$
  \item 
    You chose $w = a^pb^p$
  \item 
    The devil is constrained by $|xy| \leq p$ to choose $y$ to consist
    exlucisvely of a's. So $y = a^k$ for some $0 < k \leq p$.
  \item 
    I choose $i = 2$. Didn't quite catch the rest
  \end{enumerate}
  Thus $L$ is not regular.
\end{example}

\begin{example}
  $L = \{a^q \ | q \ \text{a prime number}\}$
  \\\\
  Demon picks $p > 0$. I pick $a^n$ where $n > p$, $n$ is a prime. Demon has to
  pick $y = a^k$ where $0 < k \leq p$. I pick $i > 1$, deferring the exact
  choice. New string $xy^iz$ is $a^{n + (i - 1)k}$. Choose $i = n + 1$. Then $a^{n + nk} =
  a^{n(1 + k)}$ which is not a prime number so $L$ is not regular.
\end{example}

\begin{example}
  $L = \{a^nb^m \ | \ n \neq m\}$  % review
  \\\\
  Wants you have a stock of languages that you know are not regular, you don't
  always have to do pumping. This example is hard to do \ul{directly} with the
  pumping lemma.
  \\\\
  $\ol{L}$ (L complement) is a big mess. But $\ol{L} \cap a^*b^* = \{a^nb^n \ |
  \ n \geq 0\}$. This is not a regular language to $L$ is not regular.
\end{example}

\begin{example}
  $L = \{a^ib^i \ | i > j\}$.
\end{example}

\begin{example}
  $L = \{x + y = z \ | xyz \in \{0, 1\}^* \wedge \text{the equation is valid} \}$
  \\\\
  Demon picks $p$. I pick
  \begin{gather*} % review
    \underbrace{111 \cdots 1}_p
  \end{gather*}
\end{example}

\begin{definition}
  If $S \subseteq \NN$, define $unary(S) = \{1^n \ | n \in S\}$. $binary(S) = \{w \in \{0, 1\}^* \ | \
  w \text{read as a binary number is in} S\}\}$
  \\\\
  If $binary(S)$ is regular does that mean $unary(S)$ is regular? No. Consider
  $S = \{2^n \ | \ n \geq 1\}$. Then binary is regular because $100^*$ is
  clearly regular. $unary(S) = \{1^{2^p}\}$ is not regular because you can pump
  to a non power of 2.
\end{definition}

\subsection{Kleene Algebras}

\begin{definition}[Semi-ring]
  A set with 2 operations. $S$: the carrier of the semi-ring. $+: S \times S \to
  S$. $\times: S \times S \to S$. $0: S, \ 1:S$. $(S, +, 0)$ forms a commutative
  monoid. $(S, x, 1)$ forms a monoid. $\times$ distributes over $+$. 0
  annihilates with $x$ i.e. $a \times 0 = 0 \times a = 0$.
  \\\\
  Semi-ring is similar to a ring, but doesn't require that each element has an
  additive inverse. i.e. if $(S, +, 0)$ forms a group then it produces a ring
  instead of a semi-ring.
  \begin{example}
    $(\NN, 0, 1, +, \times)$. This forms a semi ring, because we don't have the
    negative integers for the additive inverses. $(\ZZ, 0, 1, +, \times)$ would
    form a ring.
    \\\\
    $\ZZ + i\ZZ = \{a + ib \ | \ a, b \in \ZZ, \ i^2 = -1\}$ is the ring of
    Gaussian integers.
  \end{example}
  \begin{example}
    $n \times n$ matrices over $\NN$. Multiply by matrix multiplication and add componentwise.
  \end{example}
  \begin{example}
    An \ul{idempotent semiring} $J$ is a semi ring such that $\forall x \in J, x^2 =
    x$. $(T, F, \vee, \wedge)$.
    \\\\
    idempotent is an element which is unchanged in value when multiplied or
    otherwise operated on by itself.
  \end{example}
  \begin{example}
    If we have any semiring, the set of $n \times n$ matrices with entries in this
    semiring form a semi-ring.
  \end{example}

\end{definition}
\begin{definition}[Kleene Algebras]
  $K = (S, +, \cdot, 0, 1, *)$.
  \begin{enumerate}
  \item 
    $(S, +, 0)$: commutative monoid
  \item 
    $(S, \cdot, 1)$: monoid
  \item 
    $(S, +, \cdot, 0, 1)$ forms an idempotent semiring.
  \item 
    \begin{gather*}
      1 + aa^* = a^* \\
      a + a^*a = a^*
    \end{gather*}
  \item 
    We introduce a partial order $a \leq b \coloneqq a + b = b$ (check that this
    is really a partial order). 2 rules.
    \begin{gather*}
      b + ac \leq c \Rightarrow a*b \leq c \\
      b + ca \leq c \Rightarrow ba^* \leq c
    \end{gather*}
  \end{enumerate}
\end{definition}

\begin{example}
  Let $\Sigma$ be a finite alphabet $S = \text{regular languages} \subseteq
  \Sigma^*$. $+$ is union. $\cdot$ concatenation. $*$ is kleene star.
\end{example}

\begin{example}
  $S$ any set and $R$ the collection of \ul{binary relations} on $S$. A binary
  relation $r \subseteq S \times S$. $+$ is union. $\cdot$ is relational
  composition. $x r y$ means $(x, y) \in r$. $x (r \cdot s) y = \exists z s.t. x
  r z \wedge z s y$. $0 \coloneqq \varnothing$. $1 \coloneqq \{(s, s) \ | \ s
  \in S\}$. $r^*$ is the reflexive, transitive closure of $r$.
  \\\\
  graphs are a nice way of picturing relations. $r^*$ is reflexive, transitive
  closure of $r$. transitive closure let's you take the paths of the graph. And
  everything is related to itself. Directed graph because not necessarily
  symmetric. $x r^* y$ if $\exists n \in \NN, z_1, \cdots, z_n s.t. x r z_1
  \wedge z_1 r z_2 \cdots z_n r y$ . i.e. there exists a path from x to y.
  \\\\
  Solving for x. $ax + b = x$. $a^*b$ is the smallest solution. $aa*b + b = (aa*
  + 1)b = a^*b$. Smallest solution means that if $x$ is another solution, then
  $a^*b \leq x$.
\end{example}

\begin{example}
  If $K$ is any kleene algebra, $M_n(K)$ is $n \times n$ matrices with entries
  in $K$. Do operations as you would expect about matrices. What the heck is
  star though?
  0 is 0. 1 is 1 along diagonal and 0 everywhere else.
  \begin{gather*}
    \begin{bmatrix}
      a & b \\
      c & d
    \end{bmatrix}^*
    \coloneqq
    \begin{bmatrix}
      (a + bd^*c)^* & (a + bd^*c)^*bd^* \\
      (d + ca^*b)^*ca^* & (d + ca^*b)^*
    \end{bmatrix}
  \end{gather*}
  Now we can solve matrix vector equations.
\end{example}

Dexter Kozen has developed and extended the theory tremendously including
probabilistic extensions.

\end{document}