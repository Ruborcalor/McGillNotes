\documentclass[class=scrartcl, crop=false]{standalone}

\usepackage[sexy]{evan}


\begin{document}

\section{Normal Subgroups and Factor Groups}

\begin{definition}
  A subgroup $H \subset G$ is \ul{normal} if $gH = Hg$ for all $g \in G$.
  \begin{example}
    \begin{enumerate}
      \ii
      Every subgroup of H is normal if $G$ is abelian.
      \ii
      If $[G:H] = 2$, then $H$ is normal. This is because $gH \cup H = G = H \cup Hg$.
      \ii
      Let $H \subset D_h$ be a subgroup of rotations. Then $H$ is normal (because $[D_h:H] = 2 $). However, let $R = \langle r \rangle $ where $r$ is reflection, then $R$ is not normal in $D_n$.
      \ii
      $\{e\} \subset G$ and $G \subset G$ are normal.
    \end{enumerate}
  \end{example}
\end{definition}
\begin{theorem}
  Let $N \subset G$ be a subgroup. TFAE
  \begin{enumerate}
    \ii 
    $N$ is normal in $G$.
    \ii
    $gNg^{-1} \subset N$ for all $g \in G$.
    \ii
    $gNg^{-1} = N$ for all $g \in G$.
  \end{enumerate}
  \begin{note}
    For $S \subset G$ and $x,y \in G$, $xSy = \{xsy: s \in S\}$
  \end{note}
  \begin{proof}
    \begin{enumerate}
      \ii[]
      \ii[] $(1 \Rightarrow 2)$
      We must show that $gng^{-1} \in N$ for all $n \in N$.
      \begin{gather*}
        gN = Ng \Rightarrow \exists n' \in N \ \text{such that} \ gn = n'g \\
        \text{Hence: } \quad (gn)g^{-1} = (n'g)g^{-1} = n' \in N
      \end{gather*}
      \ii[] $(2 \Rightarrow 3)$
      Suffices to show that $N \subset gNg^{-1}$.
      \begin{gather*}
        g^{-1}ng \in g^{-1}N(g^{-1})^{-1} \subset N \Rightarrow g^{-1}ng = n' \ \text{for some} \ n' \in N \\
        \ \text{So} \ n = gn'g^{-1}
      \end{gather*}
      \ii[] $(3 \Rightarrow 1)$ 
      Right multiply by $g$. $gNg^{-1} = N$ gives $gN = Ng$.
    \end{enumerate}
  \end{proof}
\end{theorem}

\subsection{Factor Group or Quotient Group}
\begin{definition}
  Let $N \subset G$ be a normal subgroup of $G$. The left cosets of $N$ in $G$ form a group whose operation is $(aN)(bN) = (abN)$. 
  This is the \ul{quotient group} of $G$ and $N$, denoted by $G / N$.
\end{definition}
\begin{theorem}
  $G / N$ is really a group!
  \begin{proof}
    \begin{enumerate}
      
      \ii[]
      \ii
    To show: Operation is well defined. If $aN = a'N$ and $bN = b'N$, then $abN$ = $a'b'N$.
    \\\\
    We know that $a' = an_1$ and $b' = bn_2$ where $n_1,n_2 \in N$. Hence $a'b' = (an_1)(bn_2)$. Because $Nb = bN$, we hvae that $n_1b = bn_3$ for some $n_3 \in N$. Therefore $a'b' = a(n_1b)n_2 = a(bn_3)n_2 = abn_3n_2$.
    \\\\
    Thus $a'b'N = abN$ since $(ab)^{-1}(a'b') = b^{-1}a^{-1}abn_3n_2 = n_3n_2 \in N$.
    \ii
    To show: Associativity.
    \begin{gather*}
      aN(bNcN) = aN(bcN) = a(bc)N = abcN \\
      (aNbN)cN = (abN)cN = (ab)cN = abcN
    \end{gather*}
    \end{enumerate}
    \ii
    To show: Identity. $eNxN = exN = xN = xeN = xNeN$
    \ii
    To show: Inverses. $(xN)(x^{-1}N)$ = $xx^{-1}N = eN = x^{-1}xN = x^{-1}NxN$
  \end{proof}
\end{theorem}
\begin{recall}
  If $G$ is finite, $|G / N| = [G:N] = |G| / |N|$
\end{recall}
\begin{example}
  $\ZZ_n$ is just notation for $\frac{\ZZ}{n\ZZ}$
   \[
     \text{Quotient Group} \ \ZZ / 4\ZZ: \quad
       \setlength{\extrarowheight}{3pt}% local setting
       \begin{array}{l|*{5}{l}}
       \circ      & 0 + 4\ZZ & 1 + 4\ZZ & 2 + 4\ZZ & 3 + 4\ZZ \\
       \hline
       0 + 4\ZZ           & 0 + 4\ZZ & 1 + 4\ZZ & 2 + 4\ZZ & 3 + 4\ZZ \\
       1 + 4\ZZ           & 1 + 4\ZZ & 2 + 4\ZZ & 3 + 4\ZZ & 4 + 4\ZZ \\
       2 + 4\ZZ           & 2 + 4\ZZ & 3 + 4\ZZ & 4 + 4\ZZ & 5 + 4\ZZ \\
       3 + 4\ZZ           & 3 + 4\ZZ & 4 + 4\ZZ & 5 + 4\ZZ & 6 + 4\ZZ \\
       \end{array}
   \]
\end{example}
\begin{example}
  $H \subset D_n$ be subgroup of rotations. $D_n / H \cong \ZZ_2$ since $[D_n:H] = 2$.
\end{example}
\begin{example}
  $S_n / A_n \cong \ZZ_2$
\end{example}
\begin{example}
  $N = \{\pm 1\}$ is normal in $Q$. It's cosets are:
  \begin{gather*}
    1N = \{\pm 1\} = N 1 \\
    j N = \{\pm j\} = N j \\
    k N = \{\pm k\} = N k \\
    i N = \{\pm i\} = N i \\
  \end{gather*}
  What is $Q / N$? Note: $|Q / N| = [Q:N] = 4$.
  \[
      Q / N: \quad
      \setlength{\extrarowheight}{3pt}% local setting
      \begin{array}{l|*{5}{l}}
      \circ      & 1N & iN & jN & kN \\
      \hline
      1N           & 1N & iN & jN & kN \\
      iN           & iN & 1N & kN & jN \\
      jN           & jN & kN & 1N & iN \\
      kN           & kN & jN & iN & 1N \\
      \end{array}
  \]
\end{example}
\begin{example}
  \[
      (\ZZ_4, +): \quad
      \setlength{\extrarowheight}{3pt}% local setting
      \begin{array}{l|*{5}{l}}
        \circ      & (0,0) & (1,0) & (0,1) & (1,1) \\
      \hline
        (0,0)           & (0,0) & (1,0) & (0,1) & (1,1) \\
        (1,0)           & (1,0) & (0,0) & (1,1) & (0,1) \\
        (0,1)           & (0,1) & (1,1) & (0,0) & (1,0) \\
        (1,1)           & (1,1) & (0,1) & (1,0) & (0,0) \\
      \end{array}
  \]
    
\end{example}
\end{document}
