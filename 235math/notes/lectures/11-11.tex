\documentclass[class=scrartcl, crop=false]{standalone}

\usepackage[sexy]{evan}
\usepackage{cole}

\date{2019-11-11}


\begin{document}

\section{11-11}

\begin{proposition}
  Let $\varphi: R \to S$ be a ring homomorphism.
  \begin{enumerate}
    \ii
    $R$ is commutative implies that $\varphi(R)$ is a commutative ring
    \ii
    If $R$ and $S$ have unity $1_R$ and $1_S$ and $\varphi$ is surjective, then $\varphi(1_R) = 1_S$
    \ii
    If $R$ is a field, then $\varphi(R) = \{0\}$  or $\varphi(R)$ is a field.
    \begin{proof}[Proof of 3]
      We know that $\varphi(R)$ is a commutative subring by (1). Let $a \in \varphi(R)$.
      \\\\
      $\varphi(1_R)$ is the multiplicative identity for $\varphi(R)$ so $a = \varphi(\hat{a})$ for some $\hat{a} \in R$.
      \[
        a \cdot \varphi(1_R) = \varphi(\hat{a})\varphi(1_R) = \varphi(\hat{a}1_R) = \varphi(\hat{a}) = a
      \] 
      Similarly, $\varphi(1_R)a = a$.
      \\\\
      If $\varphi(x) \neq 0_S$, then $x \neq 0_R$. So $\exists x^{-1} \in R$ such that $xx^{-1} = 1_R$.
      \[
        \varphi(x) \cdot \varphi(x^{-1}) = \varphi(xx^{-1}) = \varphi(1_R)
      \]
      So $\varphi(x^{-1})$ equals $(\varphi(x))^{-1}$.
      \\\\
      Either $\varphi(R) = \{0\}$, or it doesn't.
      \\\\
      If $\varphi(1_R) \neq 0_S$, we are done.
      \\\\
      If $\varphi(1_R) = 0_S$, then $\varphi(R) = \{0\}$ because for each $a \in \varphi(R)$, $a = \varphi(\hat{a})$ for some $\hat{a} \in R$. Therefore:
      \[
        a = \varphi(\hat{a}) = \varphi(\hat{a}1_R) = \varphi(\hat{a})\varphi(1_R) = a \cdot 0_S = 0
      \]

      \begin{recall}
        Being a field is a stronger property than being an integral domain. When every element has a multiplicative inverse, it must be an integral domain.
      \end{recall} 
      
    \end{proof} 
  \end{enumerate} 
\end{proposition} 
\subsection{Ideals}
\begin{definition}
  An \ul{ideal} $I$ in ring $R$ is a subring $I \subset R$ such that if $x \in I$ and $r \in R$, then $xr \in I$ and $rx \in I$.
  \begin{example}
    $\{0\} \subseteq R$ and $R \subseteq R$ are ideals.
  \end{example} 
  \begin{example}
    If $a \in R$ is a commutative ring, then $\langle a \rangle = \{ar : r \in R\}$ is an ideal.
    \\\\
    $\langle a \rangle $ is a \ul{principal ideal}.
    \begin{proof} \leavevmode \\
      Prooving that $\langle a \rangle $ is a subring:
      \\\\
      $\langle a \rangle $ is non empty because $0 = 0a \in \langle a \rangle $.
      \\\\
      $r_1a, r_2a \in \langle a \rangle \Rightarrow r_1a \cdot r_2a = (r_1\cdot r_2)a \in \langle a \rangle $.
      \\\\
      $(ar_1)(ar_2) = a(r_1ar_2) = ar_3 \in \langle a \rangle $.
      \\\\
      % $x \in \langle a \rangle \Rightarrow rx \in \langle a \rangle $ because $x = as$ for some $s \in R$.
      % \\\\
      Prooving that $\langle a \rangle $ is an ideal:
      \\\\
      \[
        x \in \langle a \rangle \Rightarrow rx \in \langle a \rangle 
      \] because $x = as$ for some $s \in R$. Therefore $rx = r(as) = a(rs) \in \langle a \rangle $.
    \end{proof} 
  \end{example} 
\end{definition} 
\begin{theorem}
  Every ideal in $\ZZ$ is $\langle n \rangle $ for some $n$.
\end{theorem} 
\begin{proposition}
  The kernel of a ring homomorphism $\varphi: R \to S$ is an ideal of $R$.
  \begin{proof}
    $K = \ker(\varphi)$ is an additive subgroup.
    \\\\
    We must check that $k \in K \Rightarrow rk \in K \ \text{and} \ kr \in K$ for all $r \in R$.
    \\\\
    $rk \in K$ because $\varphi(rk) = \varphi(r)\varphi(k) = \varphi(r)0 = 0$ 
    \\\\
    $kr \in K$ because $\varphi(kr) = \varphi(k)\varphi(r) = 0\varphi(r) = 0$ 

  \end{proof} 
\end{proposition} 
\begin{theorem}
  Let $I$ be an ideal of $R$. The factor group $R / I$ is a ring with multiplication! 
  \\\\
  $(a + I)(b + I) = (ab + I)$.
  \begin{proof}
    Check that it is well defined. i.e. that if $a + I = a' + I$ and $b + I = b' + I$, then we need $(a + I)(b + I) = (a' + I)(b' + I)$.
    \\\\
    Let $a' = a + \alpha$ where $\alpha \in I$, and let $b' = b + \beta$ where $\beta \in I$. Then:
    \[
      a'b' = (a + \alpha)(b + \beta) = ab + a\beta + \alpha b + \alpha \beta
    \]
    $ab + a\beta + \alpha b + \alpha\beta \in ab + I$ because $a\beta + \alpha b + \alpha \beta \in I$. Therefore $a'b' + I = ab + I$
  \end{proof} 
\end{theorem} 

\begin{theorem}[1st Isomorphism Theorem for Rings]
  Let $\varphi: R \to S$ be a homomorphism.
  \\\\
  Let $I = \ker(\varphi)$.
  \\\\
  Let $\phi: R \to R / I$.
  \\\\
  Then there exists $\nu: R / I \to \varphi(R)$ such that $\varphi = \nu \circ \phi$.
  
\end{theorem} 

\end{document}
