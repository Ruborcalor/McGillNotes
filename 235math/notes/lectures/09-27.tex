\documentclass[class=scrartcl, crop=false]{standalone}

\usepackage[sexy]{/home/gautierk/.config/evan}
\usepackage{nicematrix}
\NiceMatrixOptions{transparent}

\title{Notes 09-27}

\begin{document}


\section{Cyclic Subgroups}


\begin{note}[Generator Group Notation]
  \hfill \newline
  Let $g \in (G, \circ)$. Notation: $\langle g\rangle = \{g^n: n \in \mathbb{Z}\}$ \\
  Let $g \in (G, +)$. Notation: $\langle g\rangle  = \{ng: n \in \mathbb{Z}\}$
\end{note}

\begin{example}[Generator Groups]
  \begin{align*}
    5 & \in \ZZ. & \quad\langle 5 \rangle = \{ \dots, -10, -5, 0, 5, \dots \} \\
    2 & \in \ZZ. & \quad\langle 2 \rangle = \{\text{Even integers.}\} \\
    5 & \in \ZZ_{10}. & \quad\langle 5 \rangle = \{0, 5\} \\
    6 & \in \ZZ_{10}. & \quad\langle 6 \rangle = \{6, 2, 8, 4, 0\} \\ \\
  \end{align*}
  Note: $\ZZ_{10} = \langle 1 \rangle = \langle 3 \rangle = \langle 7 \rangle = \langle 9 \rangle $
\end{example}

\begin{theorem}
  Let G be a group. Let $x \in G$, then $\langle x \rangle$ is a subgroup of G. Also, $\langle x \rangle$ is the smallest subgroup containing x.
\end{theorem}

\begin{definition}
  $\langle x \rangle$ is the cyclic subgroup generated by $x$. If $G = \langle x \rangle$, then $G$ is a cyclic group and $x$ is a \ul{generator} of G.
\end{definition}

\begin{definition}
  Detecting wheter or not a subset is a subgroup.
  \begin{enumerate}
    \ii
    The identity Element is in the subgroup.
    \ii
    Inverse of each element is inside.
    \ii
    If two elements are inside, their product is inside as well.
  \end{enumerate}
\end{definition}

\section{Chapter 5 09-27}

\begin{definition}
  A \ul{permutation} of set X is a bijection $f:X \to X$.
\end{definition}

\begin{example}
  $x = \{1, 2, 3, 4\} \to \{3, 1, 4, 3\}$
\end{example}

I'm not fast enough to write this
\begin{gather*}
  \begin{bmatrix}
    1 & 2 & 3 & 4 \\
    3 & 1 & 4 & 3
  \end{bmatrix} \\
  \text{General notation:}
  \begin{bmatrix}
    1 & 2 & \dots & n \\
    f(1) & f(2) & f(\dots) & f(n)
  \end{bmatrix}
\end{gather*}

\begin{definition}
  The symmetric group of degree n (on n objects) is group $S_N$ consisting of all permutations of $X = \{1, 2, \dots n\}$
\end{definition}

\begin{theorem}
  $S_n$ is a group whose binary operation is composition of functions.
\end{theorem}

\begin{proof}
  \begin{enumerate}
    \ii[]
    \ii
    Composition of functions is associative.
    \ii
    Inverses exist because inverses of bijections are bijections. $f^{-1}$ is inverse of $f$.
  \end{enumerate}
\end{proof}

\begin{example}
  \[
    \begin{bmatrix}
      1 & 2 & 3 & 4 \\
      4 & 3 & 1 & 2
    \end{bmatrix} 
    \begin{bmatrix}
      1 & 2 & 3 & 4 \\
      2 & 1 & 4 & 3
    \end{bmatrix} =
    \begin{bmatrix}
      1 & 2 & 3 & 4 \\
      3 & 4 & 2 & 1
    \end{bmatrix}
  \]
\end{example}

\begin{note}
  $S_n$ has $n!$ elements.
\end{note}
\begin{example}
  Consider the following function.
  \[
    \begin{bmatrix}
      1 & 2 & 3 & 4 & 5 & 6 \\
      1 & 3 & 4 & 5 & 2 & 6
    \end{bmatrix}
  \]
\end{example}

\begin{definition}
  A \ul{cycle} is a permutation with property that there is a subset $\{a_1, a_2, \dots, a_m\} \subset \{1, 2, \dots, n\}$ such that $f(a_i) = a_{i + 1}$ for $1 \leq i < m$, and $f(a_m) = a_1$, and $f(x) = x$ when $x \notin \{a_1, \dots, a_m\}$.
  % $x_1 \to x_2 \to x_3 \to \dots \to x_1$ for some subset $\{x_i\} \subset X$ and $y \to y$ for all $y \notin \{x_i\}$
\end{definition}

\begin{example}
  Consider the following function.
  \[
    \begin{bmatrix}
      1 & 2 & 3 & 4 & 5 & 6 & 7 \\
      4 & 2 & 3 & 7 & 5 & 6 & 1
    \end{bmatrix}
  \]
  
  $1 \to 4 \to 7 \to 1$. \quad $(1, 4, 7)$ are being cycled.

  $(2, 3, 5, 6)$ are fixed.
\end{example}

\begin{note}
  Use notation $(a_1, a_2, \dots, a_m)$ for the cycle. All other elements are fixed.
\end{note}

\begin{example}
  $(3 \ 7 \ 5 \ 1) \in S_7$ contains the same information as:
  \[
    \begin{bmatrix}
      1 & 2 & 3 & 4 & 5 & 6 & 7 \\
      3 & 2 & 7 & 4 & 1 & 6 & 5
    \end{bmatrix}
  \]
  But the former is easier to understand.
\end{example}

\begin{note}
  $(a_1, a_2, \dots, a_m)$ and $(b_1, b_2, \dots, b_l)$ are \ul{disjoint} if $a_i \neq b_j$ for $i, j$.
\end{note}

\begin{example}
  $(3 \ 7 \ 5 \ 1)$ is disjoint from $(64)$, but note that there are multiple ways of representing the same cycle. \newline
  For example. $(3 \ 7 \ 5 \ 1) = (5 \ 1 \ 3 \ 7) = (7 \ 5 \ 1 \ 3)$
\end{example}

\begin{theorem}
  Disjoint cycles commute.
  $$ (a_1 \ \dots \ a_m)(b_1 \ \dots \ b_l) = (b_1 \ \dots \ b_l)(a_1 \ \dots \ a_m) $$
  if $c \notin \{a_1 \ \dots \ a_m, b_1 \ \dots \ b_l\}$
\end{theorem}

\begin{theorem}
  Every permutation is a product of disjoint cycles.
\end{theorem}

\begin{example}
  \[
    \begin{bmatrix}
      1 & 2 & 3 & 4 & 5 & 6 & 7 & 8 & 9 \\
      3 & 6 & 5 & 9 & 8 & 2 & 4 & 1 & 7 
    \end{bmatrix}
    =
    (3 \ 5 \ 8)(2 \ 6)(7 \ 4 \ 9)
    \in S_9
  \]
  More Practice:
  \[
    \begin{bmatrix}
      1 & 2 & 3 & 4 & 5 & 6 & 7 & 8 & 9 & 10 \\
      7 & 8 & 2 & 1 & 5 & 3 & 6 & 10 & 9 & 4
    \end{bmatrix}
    =
    (1 \ 7 \ 6 \ 3 \ 2 \ 8 \ 10 \ 4)(5)(9)
    \in S_{10}
  \]
  Practice in the other direction:
  \[
    ((1 \ 3 \ 5)(2 \ 7 \ 6 \ 4))((1\ 2)(3 \ 4)(5 \ 6 \ 7)) = (1 \ 7)(2 \ 3)(4 \ 5)(6)
  \]
  The $(6)$ at the end is unnecessary because it is an identity element.
\end{example}

He just drew a pictorial circle on the board. I am just going to watch and absorb.

\begin{theorem}
  Every permutation is a product of transpositions because:
  \begin{theorem}
    Every n-cycle is a product of $(n - 1)$ transpositions.
  \end{theorem}
  \begin{proof}
    $(a_1 \ a_2 \ \dots \ a_m) = (a_1 \ a_m)(a_1 \ a_{m - 1}) \dots (a_1 \ a_3)(a_1 \ a_2)$
  \end{proof}
\end{theorem}

\end{document}
