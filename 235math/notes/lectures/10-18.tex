\documentclass[class=scrartcl, crop=false]{standalone}

\usepackage[sexy]{evan}
\usepackage{nicematrix}
\NiceMatrixOptions{transparent}


\begin{document}

\section{Isomorphisms Continued}

\begin{theorem}
  If $G$ is cyclic and $|G| = n$, then $G \equiv \ZZ_n$.
  \begin{proof}
    Consider $\phi:\ZZ_n \to G$ given by $\phi(i) = g^i$, then $\phi$ is a bijection.

    Injective:
    $\phi(i) = \phi(j) \Rightarrow g^i = g^j \Rightarrow g^{i - j} = g^0 \Rightarrow i - j \equiv_n = 0 \Rightarrow i = j$

    Surjective: 
    Let $G = \langle g \rangle.$

    $\{g^0, g^1, \dots, g^{n - 1}\} = G$

    $\{0, 1, \dots, n - 1\} = \ZZ_n$
  \end{proof}
\end{theorem}

\begin{theorem}
  Cor 9.9.

  If $|G| = p$ and $p$ is prime, then $G \equiv\sim \ZZ_p$

   \begin{proof}
     We showed that $G = \langle g \rangle $ for any $g \neq e$.

     My understanding: if prime order, it must be cyclic.
  \end{proof}
\end{theorem}

\begin{theorem}
  Isomorphism is an equivalence relation on a set of groups.

  Reflexive: $G \equiv\sim G$ because $1_G:G \to G$ is isomorphism. 
  \[1_G(ab) = ab = 1_G(a) \cdot 1_G(b)\]

  Symmetrical: $G \equiv\sim K \Rightarrow K \equiv\sim G$ because $\phi:G \to K$ isomorphism then $\phi^{-1}:K \to G$ is isomorphism.

  Transitive: $f: G \to K$ and $h:K \to J$ are isomorphisms then $h \circ f: G \to J$ is ismorphism.
\end{theorem}

\begin{theorem}[Cayley's Theorem]
  Every group is isomorphic to a permutation group.
  \begin{recall}
    A permutation group is a subgroup of $S_n$
  \end{recall}
  \begin{proof}
    G is isomorphic to a subgroup of the group of bijections of the set G. You could think of this as $S_G$.

    For  $g \in G$, let $\lambda_g:G \to G$ be permutation "left multiply by g" 
    i.e. $\lambda_g(x) = gx$ for all $x \in G$.

    Let $\overline{G} = \{\lambda_g:g \in G\}$

    Claim:  $G \cong \overline{G}$ with $\phi(g) = \lambda_g$

    Injectivity: if $\phi(x) = \phi(y)$ then $\lambda_x$ and $\lambda_y$ are some bijection of $ G$.
    \[x = xe = \lambda_x(e) = \lambda_y(e) = ye = y\]

    Surjectivity (immediate). $\overline{G} = \{\lambda_g:g \in G\} = \{\phi(g):g \in G\} = \phi(G)$

    Homomorphism: 
    \begin{gather*}
      \phi(xy) = \lambda_{xy} \\
      \phi(x)\phi(y) = \lambda_x \lambda_y \\
      \lambda_{xy}(z) = (xy)z \ \text{for all} \ z \in G \\
      \lambda_x(\lambda_y(z)) = \lambda_x(yz) = x(yz) \\
      (xy)z = x(yz) \ \checkmark
    \end{gather*}
  \end{proof}
\end{theorem}

\begin{example}
  \begin{gather*}
    G = \{\pm 1, \pm i\} \\
    G \cong \overline{G} \subset S_G \cong S_4 \\
    1 \to \lambda_1 = 
    \begin{bmatrix}
      1 & -1 & i & -i \\
      1 & -1 & i & -i
    \end{bmatrix}  = () 
    \\
    -1 \to \lambda_{-1} = 
        \begin{bmatrix}
      1 & -1 & i & -i \\
      -1 &  1 & -i &  i
    \end{bmatrix} = (1 \ -1) (i \ -i)
    \\
    i \to \lambda_{i}
        \begin{bmatrix}
      1 & -1 & i & -i \\
      i &  -i & -1 &  1
    \end{bmatrix}  = (1 \ i \ -1 \ -i)
    \\
    -i \to \lambda_{-i} = 
        \begin{bmatrix}
      1 & -1 & i & -i \\
      -i &   i &  1 &  -1
    \end{bmatrix}  = (1 \ -i \ -1 \ i)
  \end{gather*}
\end{example}

\begin{example}
  \[Q_8 \cong \overline{Q_8} \subset S_8\]
\end{example}

\begin{example}
  \begin{gather*}
    \ZZ_6 \subset \to  S_{\ZZ_6} = S_{\{0, 1, 2, 3, 4, 5\}} \\
    2 \to_{\phi} \lambda_2 \qquad \lambda_2: \ZZ_6 \to\ZZ_6 \qquad\lambda_2(x) = 2 + x \\
    \lambda_2 = (0 \ 2 \ 4)(1 \ 3 \ 5) \\
    \lambda_3 = (0 \ 3)(1 \ 4)(2 \ 5) \\
    \lambda_5 = (0 \ 5 \ 4 \ 3 \ 2 \ 1)
  \end{gather*}
\end{example}

\end{document}
