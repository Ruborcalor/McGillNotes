\documentclass[class=scrartcl, crop=false]{standalone}

\usepackage[sexy]{evan}
\usepackage{nicematrix}
\NiceMatrixOptions{transparent}

\usetikzlibrary{calc}
\usepackage{listofitems}
\newcommand\cycle[2][\,]{%
  \readlist\thecycle{#2}%
  (\foreachitem\i\in\thecycle{\ifnum\icnt=1\else#1\fi\i})%
}



\begin{document}

\section{Cosets}

\begin{definition}
  Let $G$ be a group. Let $H$ be a subgroup of $G$.  $g \in G$ 

  $gH = \{gh: h \in H\}$ (left coset)

  $Hg = \{hg: h \in H\}$ (right coset)
\end{definition}

\begin{theorem}
  Left (or right) cosets of $H$ in $G$ \ul{partition} $G$.
\end{theorem}

\begin{theorem}[Lagrange-theorem]
  Let $G$ be a finite group. Let $H$ be a subgroup of $G$.
  \[
    [G:H] = \frac{|G|}{|H|}
  \]
  \begin{note}
    $[G:H]$ is the number of cosets of $H$ in $G$.
  \end{note}
\end{theorem}

\begin{remark}
  Right and left cosets are not necessarily equal.
  \begin{example}
    Let $H = \{id, (1 \ 2)\}$ be a subgroup of $S_3$.
    \begin{gather*}
      (1 \ 2 \ 3)H = \{(1 \ 2 \ 3), (1 \ 2 \ 3)(1 \ 2)\} = \{(1 \ 2 \ 3), (1 \ 3)\} \\
      H(1 \ 2 \ 3) = \{(1 \ 2 \ 3), (1 \ 2)(1 \ 2 \ 3)\} = \{(1 \ 2 \ 3), (2 \ 3)\}
    \end{gather*}
  \end{example}
\end{remark}

\begin{exercise}
  \begin{enumerate}[label=(\alph*)]
    \ii[]
    \ii
    What is the index of $\langle 6 \rangle $ in $\ZZ_{24}$.

    By lagrange:
    \begin{gather*}
      [\ZZ_{24}, \langle 6 \rangle ] = \frac{|\ZZ_{24}|}{|\langle 6 \rangle |} = \frac{24}{4} = 6 \\
      \langle 6 \rangle = \{0, 6, 12, 18\} \qquad |\langle 6 \rangle | = 4
    \end{gather*}
    \ii
    Let $\sigma = \cycle{1, 2, 5, 4}\cycle{2, 3}$ in $S_5$. What is the index of $\langle \sigma \rangle $ in $S_5$?
    \begin{gather*}
      \sigma = \cycle{1, 2, 5, 4}\cycle{2, 3} = \cycle{2, 3, 5, 4, 1} \ \text{because sigma is not disjoint} \ \\
      |\sigma| = 5 \qquad |\langle \sigma \rangle | = 5 \\
      [S_5:\langle \sigma \rangle ] = \frac{5!}{5} = 4! = 24
    \end{gather*}
  \end{enumerate}
\end{exercise}

\begin{exercise}
  Find the left cosets of $H = \{id, \mu\}$ in $D_4$. ($D_4$ is the symmetries of a square.)
  \begin{recall}
    $\mu = \cycle{1, 2}\cycle{3, 4}$
  \end{recall}
  $D_4 = \{id, \cycle{1, 2, 3, 4}, \cycle{1, 3}\cycle{2,4}, \cycle{1,2}\cycle{3,4}, \cycle{1,4,3,2}, \cycle{1,4}\cycle{2,3}, \cycle{1,3},\cycle{2,4}\}$

  By lagrange: $[D_4,H] = \frac{|D_4|}{|H|} = \frac{8}{2} = 4$
\end{exercise}

\begin{example}
  Let $H = \{id, \cycle{1,2}\cycle{3,4}\}$
  \begin{gather*}
    \cycle{1,3}H = \{\cycle{1,3},\cycle{1,3}\cycle{1,2}\cycle{3,4}\} = \{\cycle{1,3},\cycle{1,2,3,4}\} \\
    \cycle{2,4}H = \{\cycle{2,2},\cycle{2,4}\cycle{1,2}\cycle{3,4}\} = \{\cycle{2,4},\cycle{1,4,3,2}\} \\
    \cycle{1,3}\cycle{2,4}H = \{\cycle{1,3}\cycle{2,4},\cycle{1,3}\cycle{2,4}\cycle{1,2}\cycle{3,4}\} = \{\cycle{1,3}\cycle{2,4},\cycle{1,4}\cycle{2,3}\} \\
  \end{gather*}
  Conceptually it makes sense that the size of a subgroup must divide the size of the group so that the cosets of the subgroup can partition the group into subgroups of equal sizes.
\end{example}

\begin{recall}
  $S_n$ represents all possible bijections between $\NN_n$ and $\NN_n$
\end{recall}

\begin{exercise}
  Let $G$ be a group and $H$ a subgroup of $G$ such that $[G:H] = 2$ and $a,b \in G \setminus H$. Show that $ab \in H$.

  We know that $a^{-1} \notin H$. Therefore $a^{-1}H \neq H$.

  \begin{recall}
    $g_1H = g_2H \Leftrightarrow g_1 \in g_2H$. This implies that $g_1 \notin g_2H \Leftrightarrow g_1H \neq g_2H$.
  \end{recall}

  Similarly $b \notin H$ implies that $bH \neq H$.

  There are only two cosets, so $a^{-1}H = bH$. Therefore 
  \[
    a^{-1}h = bh'
  \]
  for some $h, h' \in H$. Reordering we get that $ab = h(h')^{-1} \in H$
\end{exercise}

\begin{exercise}
  Is it possible to have a group $G$ of order 6 such that all of its elements have order 1 or 2? NO

  Proof by contradiction.
  $G = \{e, g_1, g_2, g_3, g_4, g_5\}$ such that $|g_i| = 2$ for all $i \in \{1, 2, 3, 4, 5\}$ 

  Claim: With this construction, $G$ must be abelian.
  \begin{gather*}
    a,b \in G \\
    ab = id \cdot ab = (ba)^2ab \ \text{because} \ ba \in G \Rightarrow (ba)^2 = e \\
    \text{Expanding: } ab = babaab = bab^2 = ba
  \end{gather*}

  Consider the subgroup $H = \{1, g_1, g_2, g_1g_2\}$. This is a subgroup because G is abelian and all its elements have order 2.

  By lagrange: $[G:H] = \frac{|G|}{4}$. This is a contradiction because $|G| = 6$ and 4 does not divide 6.

  Basically, with the assumption that G is of order 6 with all elements being of order 1 or 2, we can build a subgroup of order 4 which doesn't make sense because 4 doesn't divide 6.
\end{exercise}

\begin{exercise}
  Let $H,K$ be subgroups of $G$. Show that for $x,y \in G$, either $xH \cap yK = \varnothing$, or $xH \cap yK$ is a coset of $H \cap K$. (Recall that $H \cap K$ is a subgroup).

  Pick $x, y \in G$. If $xH \cap yK = \varnothing$, we are done with this case.

  Assume that $xH \cap yK \neq \varnothing$. Pick $g \in xH \cap yK$. Then $g \in xH \Rightarrow xH = gH$ and $g \in yK \Rightarrow yK = gK$.

  Therefore $xH \cap yK = gH \cap gK$.

  Claim:  $gH \cap hK = g(H \cap K)$
   \begin{proof}
     ($\subseteq$ ). Pick $z \in gH \cap gK$.
     \begin{gather*}
       \Rightarrow z = gh = gk \ \text{for some } \ h\in H, k \in K \\
       \Rightarrow h = k \ \text{so} \ h \in H\cap K \\
       \text{So } z = gh \quad h \in H \cap K \\
       \text{Then } z \in g(H \cap K) \\\\
       (\supseteq)
       \text{Let } z \in g(H \cap K) \\
       \Rightarrow z = gl \ \text{for some} \ l \in H \cap K \\
       l \in H \Rightarrow z \in gH \\
       l \in K \Rightarrow z \in gK \\
       z \in gH \cap gK
     \end{gather*}
     Therefore $gH \cap hK = g(H \cap K)$
  \end{proof}
\end{exercise}


\end{document}
