\documentclass[class=scrartcl, crop=false]{standalone}

\usepackage[sexy]{/home/gautierk/.config/evan}
\usepackage{nicematrix}
\NiceMatrixOptions{transparent}


\begin{document}

\section{10-11}

\begin{lemma}
  Conjugacy is an equivalence relation.

  $x \sim y$ if $x = gyg^{-1}$ for some $g \in G$
\end{lemma}

\begin{theorem}
  Any two k-cycles in $S_n$ are conugate. Moreover, any conjugate of a k-cycle is a k-cycle.

  \begin{proof}
    $\alpha = (a_1 a_2 \dots a_k) \quad \beta = (b_1 b_2 \dots b_k)$

    Let $\sigma$ be a bijection with $\sigma(b_i) = a_i$ for $1 \leq i \leq k$ 

    Then $\sigma \beta \sigma^{-1} = \alpha$. Claiming that $\beta$ and $\alpha$ are conjugate to one another.

    If $x \notin \{a_1 \ \dots \ a_k\}$, then $\sigma \alpha \sigma^{-1}(x) = x$

    \begin{example}
      $\sigma \beta \sigma^{-1} (a_i) = \sigma \beta b_i = \sigma b_{i + 1} = a_{i + 1} = \alpha (a)$
    \end{example}
  \end{proof}
\end{theorem}

\subsection{$A_4$ is the group of rigid motions of a tetrahedron}

\begin{note}
  $A_4$ is the subgroup of even elements in $S_4$.
\end{note}

\begin{gather*}
  \text{identity: } [e] = \{()\} \\
  [(12)(34)] = \{(12)(34), (13)(24), (14)(23)\} \\
  \text{Clockwise rotations about a face: } [(123)] = \{(123), (134), (142), (243)\} \\
  \text{Counter clockwise rotations about a face: } [(132)] = \{(132), (143), (124), (234)\}
\end{gather*}

\begin{note}
  $A_4$ has no 6 element subgroup even though 6 divides $|A_4|$
\end{note}

\section{Isomorphisms}

\begin{definition}
  $(G, \cdot) \ \text{and} \ (H, \circ)$ are isomorphic if there exists a bijection $\phi:G \to H$ such that $\phi(a \cdot b) = \phi(a) \circ \phi(b)$ for all $a, b \in G$.

  This would make $\phi$ an isomorphism.

  $G "equal sign with sim on top" H$
\end{definition}

\begin{note}
  Let $H \subset G$ be a subgroup. It is possible for $x \not\sim y$ in H while $x \sim y$ in G. This is a distinguishing between $\sim_H$ and $\sim_G$.


  \begin{example}
    $H \subset S_{17}$

    $H = \langle (1 \ 2 \dots 17) \rangle $

    H is cyclic and abelian.

    It has 17 conjugacy classes.

    All nontrivial elements of H are conjugate in $S_{17}$.
  \end{example}
\end{note}

\begin{note}
  In an abelian group, two elements are conjugate iff they are equal.

  If $H$ is abelian, then $(x \sim y) \Leftrightarrow x = aya^{-1} = y$
\end{note}
  


\end{document}
