\documentclass[class=scrartcl, crop=false]{standalone}

\usepackage[sexy]{evan}


\begin{document}

\section{Tutorial 2019-11-01. Normal Subgroups and Quotient Groups}

\subsection{Normal Subgroups}

$N \subseteq G$ is \ul{normal} if $gN = Ng, \ \ \forall g \in G$. Equivalently $gNg^{-1} \subseteq N$ and $gNg^{-1} = N$.
\[
  gNg^{-1} = \{gng^{-1}: n \in N\}
\]
Prooving that $gNg^{-1}$ is a subgroup.
\\\\
Identity: $e \in N \Rightarrow$ 
\[
  geg^{-1} = gg^{-1} = e \in gNg^{-1}
\]
\\\\
Inverses: $gng^{-1} \in gNg^{-1}$. $(gng^{-1})^{-1} = gn^{-1}g^{-1}$. $n \in N$ so $gn^{-1}g^{-1} \in gNg^{-1}$
\\\\
Closure:
$gn_1g^{-1}, gn_2g^{-1} \in gNg^{-1}$
\[
  gn_1g^{-1}gn_2g^{-1} = gn_1n_2g^{-1}
\]
$n_1n_2 \in N$ so $gn_1n_2g^{-1} \in gNg^{-1} \ \checkmark$

\subsection{Quotient Groups}

Let $N \subseteq G$ be a normal subgroup. Then $G / N$ is a group with the operation $(aN)(bN) = (ab)N$

\begin{example}
  Let $G = \ZZ$. Let $N = 24 \ZZ = \{0, 24, 48, \dots\}$. $\ZZ / 24 \ZZ \cong \ZZ_{24}$
\end{example}
\begin{example}
  $D_8 = \{id, r, \dots, r^7, s, s r, \dots, s r^7\}$
  \\
  $N = \langle r^4 \rangle = \{id, r^4\}$ is a normal subgroup.
  \\
  $|D_8 / N| = 16 / 2 = 8$. Finding all the cosets of $N$ :
  \begin{gather*}
    id \cdot N ,\quad sN \\
    r \cdot N ,\quad s rN \\
    r^2 \cdot N ,\quad s r^2N \\
    r^3 \cdot N ,\quad s r^3N
  \end{gather*}

  \begin{exercise}
    Let $G$ be a cyclic group where  $G = \langle a \rangle $. Show that $G / N$ is cyclic.
    \\\\
    Claim: $G / N = \langle aN \rangle $.
    \\\\
    Let $bN \in G / N$. $b \in G$, so $b = a^k$ for some $k$. $bN = a^kN = (aN)^k \Rightarrow g / N = \langle aN \rangle $.
  \end{exercise}

  
\end{example}
\begin{remark}
  Let $G$ be a group, and $H, K \subseteq G$ be subgroups of $G$ such that $H \subseteq K \subseteq G$. Then $H$ being normal in $K$ and $K$ being normal in G does NOT imply that $H$ is normal in G.
  \begin{example}
    Consider the following:
    \begin{gather*}
      D_4 = \{id, r, r^2, r^3, \mu_1, \mu_2, \mu_3, \mu_4\} \\
      K = \{id, \mu_1, \mu_3, r^2\} \\
      H = \{id, \mu_1\}
    \end{gather*}
    Show that $K$ is normal in $D_4$, and that $H$ is normal in $K$, but that $H$ is not normal in $D_4$.
    \begin{note}
      Tips for determining wether or not $H$ is normal in $G$:
      \begin{enumerate}
        \ii
        If $G$ abelian, then all of its subgroups must be normal.
        \ii
        If $G$ is simple, then it has no normal non-trivial proper subgroups.
        \ii
        If $[G:H] = 2$, then $H$ is normal (we proved this in a previous assignment).
        \ii
        If all else fails, compute
      \end{enumerate}
      $[D_4:K] = 2$ so $K$ is normal in $ D_4$. $[K:H] = 4 / 2 = 2$ so $H$ is normal in $K$.
      \\\\
      Now to show that $H$ is not normal in $D_4$ with a counter example.
      \begin{align*}
        \mu_1 &= (24) \\
        H &= \{(), (24)\} \\
        r &= (1234) \in D_4 \\
        rH &= \{(1234, (12)(34)\} \\
        Hr &= \{(1234), (14)(23)\}
      \end{align*}
      Therefore $rH \neq Hr \Rightarrow H$ is not normal in $D_4$.
    \end{note}
  \end{example}
\end{remark}

\begin{exercise*}
  Let $G$ be a group, and $N \subseteq G$ be a normal subgroup. Let $gN \in G / N$.
  \begin{enumerate}[label=(\alph*)]
    \ii
    Show that $|gN| = n$ in $G / N$ where  $n$ is the smallest natural number such that  $g^n \in N$.
    \\
    Observe that $(gN)^n = g^nN = eN \Leftrightarrow g^n \in N$. Therefore the order of $(gN)$ is the smallest of $n \in \NN$ such that $g^n \in N$.
    \ii
    Give an example where $|gN|$ in $G / N$ is \ul{strictly} smaller that $|g|$ in $G$.
    \\
    Let $G = \ZZ_4 = \{0, 1, 2, 3\}$ \\
    Let $N = \langle 2 \rangle = \{0, 2\}$ \\
    Elements of $G / N$:
    \begin{gather*}
      0 + N = \{0, 2\} = 2 + N \\
      1 + N = \{1, 3\} = 3 + N
    \end{gather*}
  \end{enumerate}
\end{exercise*}

\begin{gather*}
  \ZZ / 3 \ZZ = \\
  0 + 3 \ZZ = \{0, 3\} \\
  1 + 3 \ZZ = \{1, 4\} \\
  2 + 3 \ZZ = \{2, 5\}
\end{gather*}

\end{document}
