\documentclass[class=scrartcl, crop=false]{standalone}

\usepackage[sexy]{/home/gautierk/.config/evan}
\usepackage{nicematrix}
\NiceMatrixOptions{transparent}

\begin{document}

\title{Notes 2019-09-20}
\author{Cole Killian}


\subsection{Proposition 3.21}

Proposition 3.21: Let G be a group, let $a, b \in G$.

Then the equations $ax = b$ and $xa = b$ have unique solutions.

\subsubsection{Proof of existence:} Let $x = a^{-1}b$. This is a solution for the first equation. $a * a^{-1}b = b = b \ \checkmark$ 

Let $x = ba^{-1}$. This is a solution for the second equation.

\subsubsection{Proof of uniqueness:} To do this we will show that two solutions are always the same. Suppose c and d are solutions to $ax = b$. Therefore $a * c = b \ \text{and} \ a * d = b$.

Therefore $a * c = a * d \Rightarrow a^{-1} * a * c = a^{-1} * a * d \Rightarrow c = d \ \checkmark$. This is the proof for the first equation; the same steps can be used for the second equation.

\subsection{Proposition 3.22}

Let G be a group. $(ba = ca) \Rightarrow (b = c)$.  $(ab = ac) \Rightarrow b = c$. The idea behind this is \underline{left right} cancellation.

$ba = ca \Rightarrow baa^{-1} = caa^{-1} \Rightarrow b = c$

\subsection{Notation}

$g^n = g \circ g \circ g \circ \dots \circ g$ where the number of g equals n - 1

$g^0 = e$

$g^{-1} = g^{-1} \circ g^{-1} \circ \dots \circ g^{-1}$ where the number of g equals n - 1

From this: $g^m \circ g^n = g^{m + n}$ and  $(g^m)^n = g^{m * n}$

Careful:  $(ab)^m \neq a^m b^n$. Not necessarily commutative so you can't pass them through one another. Review a conceptually understanding of this.

For commutative  $(G, +)$

 $-g$ is notation for inverse of g.

\subsection{3.3 a \underline{subgroup} H of group G is a subset of G}

st $(H, \circ)$ is itself a group.

\subsubsection{ex. of the above.}

$_3\mathbb{Z} = \{\dots, -9, -6, -3, 0, 3, 6, \dots\}$

Perform checks. every element has an inverse. Review the group requirements. Every element has inverse. There is a unique identity element. Associative. 


ex. The trivial subgroup  $\{e\} \subset G$


ex.  $(\mathbb{C}^*, \circ)$. Let  $H = \{1, -1, i, -i\}$

ex.  $SL_2(R) \subset GL_2(R)$.

Subgroup of $2\times 2$ real invertible matricies but this time determ inent must equal 1.  This works because determ inent of inverse of matrix is multiplicative inverse of determ inent; which in this case is also 1.


ex.  $SL_2(Z) \subset SL_2(R) \subset GL_2(R)$.

\subsection{Proposition 3.30: Criterian for subgroup}

A subset $H \subset G$ of a group $(G, \circ)$ is a subgroup iff: 

(1) $e \in H$ 

(2) $h_1, h_2 \in H \Rightarrow h_1h_2, \in H$

3)  $h \in H \Rightarrow h^{-1} \in H$ 

\subsubsection{Proof that being a group gives these requirements}

(1) Let $e'$ be identity element of H. Then $e' = e'e = ee' =  e \Rightarrow e = e'$

(2) Holds because H is a group. We like to say: H is "closed under multiplication".

(3) Since H is a  group, h must have an inverse.

\subsubsection{Proof that these requirements means it must be a subgroup}

Conditions (1), (2), (3), and associativity $\Rightarrow$ H is a group using operation of G. Must be associative because G is associative so any subset of G is also associative. 

\subsection{Proposition 3.31}

H is a subgroup $\Leftrightarrow$ $H \neq \varnothing$

Easy to understand that  $g, h \in H \Rightarrow gh^{-1} \in H$

A little harder to see that $gh^{-1} \in H $

$H \neq \varnothing \Rightarrow \exists x \in H $






\end{document}
