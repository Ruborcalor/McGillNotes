\documentclass[class=scrartcl, crop=false]{standalone}

\usepackage[sexy]{/home/gautierk/.config/evan}
\usepackage{/home/gautierk/.config/Latex/cole}

\date{2019-11-27}


\begin{document}

\section{Lecture 11-27}

\begin{example}
  \begin{gather*}
    \ZZ_3[x] / \langle x^3 + 2x + 1 \rangle 
  \end{gather*} 
  $x^3 + 2x + 1$ is irreducible because $\deg \leq 3$ and no zeros. Hence we obtain a field.
  \\\\
  Elements s: $\{ax^2 + bx + c + I : a, b, c \in \ZZ_3 \}$ where $I = \langle x^3 + 2x + 1 \rangle $. There are $3^3$ elements because $\ZZ_3$ has 3 elements and there are three coefficients in each polynomial.
  \\\\
  Then:
  \begin{gather*}
    0 + I = (x^3 + 2x + 1 + I) = (x^3 + I) + (2x + 1 + I)
  \end{gather*} 
  So $x^3 + I = (x + 2 + I)$.

\end{example} 

\begin{example}
  Example of addition
  \begin{gather*}
    (x^2 + 2x + I) + (2x + 1 + I) \\
    = x^2 + 4x + 1 + I \\
    = x^2 + x + 1 + I
  \end{gather*} 
\end{example} 
\begin{example}
  Example of multiplication
  \begin{gather*}
    (2x + 1 + I)(x^2 + 2x + I) \\
    = 2x^3 + x^2 + x^2 + 2x + I \\
    = 2x^3 + 2x^2 + 2x + I \\
    = 2(x + 2) + 2x^2 + 2x + I \\
    = 2x^2 + x + 1
  \end{gather*} 
\end{example} 

\begin{fact}
  The group of units of a finite field is cyclic!
\end{fact} 

\begin{exercise}
  Find the inverse of $(x^2 + 1 + I)$.
  \\\\
  It's inverse is $(x^2 + 1 + I)^25$.
  \\\\
  Another way to do it would be to take $(ax^2 + bx + c + I)(x^2 + 1 + I) = 1 + I$ and solve the system of linear equations over $\ZZ_3$ just like you would in linear algebra.
\end{exercise} 

\subsection{Classification of symmetries over $E^2$ plane}

\begin{enumerate}
  \ii
  e \ul{identity}
  \ii
  $\theta_p$ \ul{$\theta$ notation}. Counterclockwise rotation about point $p \in E^2$.
  \ii
  \ul{translation} $\vec{u} \to \vec{u} + \vec{v}$
  \ii
  \ul{reflection} over some line $l$
  \ii
  \ul{glide reflection} a reflection and translation over the same line
\end{enumerate} 

\begin{definition}
  A \ul{Freeze Group} $G$ is an infinite subgroup $G$ of isometries($E^2$ ) that is \ul{actually a subgroup of Isom(Strip)} which is \ul{discrete} in the sense that finitely many elements $g \in G$ have distance $(p, g(p)) < 1$.
  \\\\
  Classification: 7 types of freeze groups.


\end{definition} 


\end{document}
