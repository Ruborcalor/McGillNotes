\documentclass[class=scrartcl, crop=false]{standalone}

\usepackage[sexy]{evan}
\usepackage{nicematrix}
\NiceMatrixOptions{transparent}

\title{Math 235 Notes --- 10-07}

\begin{document}

\section{Lecture 10-07}


\begin{theorem}
  Proposition 6.9. Let $H \subset G$ and  $g \in G$.

  There is a bijection $\phi :H\to gH$ defined by $\phi(h) = gh$ 
  \begin{proof}
    This is injective because $(gh_1 = gh_2) \Rightarrow h_1 = h_2$ 

    This is surjective because $gH = \{gh: h \in H\}$ 

    $\{\phi(h): h \in H\} = \phi(H)$
  \end{proof}
\end{theorem}

\begin{theorem}
  Lagrange's Theorem

  Let $G$ be a finite group and $H \subset G$ a subgroup.

  Then $\frac{|G|}{|H|} = \left[G:H\right]$


  \begin{proof}
    $G = g_1H \cup g_2H \dots g_hH$ by theorem 6.4

    Each of $|g_iH| = |H|$ by proposition 6.9

    $|G| = n|H| = \left[G:H\right]|H| \Rightarrow \frac{|G|}{|H|} = \left[G:H\right]$ 
  \end{proof}
\end{theorem}

\begin{theorem}
  Corollary 6.11.

  Let $G$ be finite and $g \in G$. Then $|g|$ divides $|G|$.

  \begin{proof}
    $|g| = |\langle g \rangle|$ which represents a subgroup of G which divides  $|G|$ by Lagrange's Theorem.
  \end{proof}
\end{theorem}

\begin{theorem}
  Corollary. If $g \in G$ is finite, then $g^{|G|} = e$. Intuitively this makes sense because the order of $g$ divides the order of $G$.

  $g^{|G|} = g^{|g| \cdot \left[G:\langle g \rangle \right]} = g^{|g|^{\left[G:\langle g \rangle \right]}} = e^{\left[G:\langle g \rangle \right]} = e$
\end{theorem}

\begin{example}
  For $\sigma \in S_n, \sigma^{n!} = e$. But this is very inefficient.
\end{example}

\begin{theorem}
  Corollary. If $|G| = p$ with p prime, then $G = \langle g \rangle $ for each $g \in G - \{e\}$.
  \begin{proof}
    $1 \neq |g|$ Nd $|g|$ divides $|G| = p (by 6.11)$. Therefore $|\langle g \rangle | = p$ so $\langle g \rangle = G$.
  \end{proof}
\end{theorem}

\begin{theorem}
  Corollary. If $K \subset H \subset G$ is a finite group, then $\left[G:K\right] = \left[G:H\right]\left[H:K\right]$

   \begin{proof}
     $\left[G:K\right] = \frac{|G|}{|K|} = \frac{|G|}{|H|} \cdot \frac{|H|}{|K|} = \left[G:H\right]\left[H:K\right]$.
  \end{proof}
\end{theorem}

\begin{definition}
  Euler $\phi$ function. $\phi:\NN\to\NN$

  $|U_n| = \phi(n)$
   \begin{example}
     $\phi(1) = 1$ 

     $\phi(9) = |\{1, 2, 4, 7, 8\}| = 5$ 

     $\phi(8) = |\{1, 3, 5, 7\}| = 4$
  \end{example}
\end{definition}

\begin{theorem}
  6.18. Euler's Theorem

  Let $a, n \in \ZZ$ with $n > 0$ and $\gcd(a, n) = 1$, then $a^{\phi(n)} = 1$ (mod n)

  \begin{proof}
    Regard $a \in U_n$.

    $a^{\phi(n)} = a^{|U_n|} = 1$ (mod n)
    
    i.e. $g^{|G|} = e$

    At the time of righting this it makes perfect sense, but we will see how it goes when I revisit it haha.
  \end{proof}
\end{theorem}

\begin{theorem}
  6.19. Fermat Little Theorem.

  Let $p$ be prime and p does not divide a. (If p divided a, then a wouldn't be in the group of units $U_p$.)

  Then $a^{p - 1} = 1$ (mod p)

  \begin{proof}
    with $p = n$ prime, $a^{p - 1} = a^{\phi(p)} \equiv_p 1$. This works because when p is prime, $\phi(n) = p - 1$.

    \ul{moreover}, for any a, $a^p \equiv_p a$. If $p | a$ this is $0 \equiv 0$.
  \end{proof}
\end{theorem}

\begin{definition}
  \ul{Conjugacy}: Let $x, y \in G$. x is \ul{conjugate} to y if there exists $g \in G$ such that $x = gyg^{-1}$. We use the notation $x \sim y$.

  \ul{Lem}: Conjugacy is an equivalence relation. Transitivity, Reflexivity, Symmetry.
  \begin{proof}
    Reflexivity: $x \sim x$ because $x = exe^{-1}$.

    Symmetry: $(x \sim y) \Rightarrow (x = gyg^{-1}) \Rightarrow y = g^{-1}xg \Rightarrow y = gg^{-1}xg^{-1}g^{-1}$

    Transitivity: 
  \end{proof}
\end{definition}

\end{document}
