\documentclass[class=scrartcl, crop=false]{standalone}

\usepackage[sexy]{evan}
\usepackage{cole}

\date{2019-11-13}


\begin{document}

\section{Lecture 11-13}

\begin{definition}
  Let $I$ be an ideal of $R$. Then $\phi: R \to R / I$ is a \ul{canonical homomorphism} associated to $I$.
  \begin{gather*}
    \phi(r) = r + I \\
    \phi(xy) = xh + I \\
    \phi(x) \phi(y) = (x + I)(y + I)
  \end{gather*} 
\end{definition} 

\subsection{Maximal and Prime Ideals}

\begin{definition}
  
An ideal $M \subseteq R$ is \ul{maximal} if the only ideal larger than $M$ is $R$ itself.
\\\\
i.e. There does not exist ideal $I$ with $M \subsetneq I \subsetneq R$.
\end{definition} 

\begin{definition}
  An ideal $P \subsetneq R$ where $R$ is commutative is \ul{prime} if for all $a, b \in R$, $ab \in P \Rightarrow [a \in P \ \text{or} \ b \in P]$. 
\end{definition} 
\begin{example}
  A proper ideal $n\ZZ \subset \ZZ$ is maximal $\Leftrightarrow$ $n\ZZ \subset \ZZ$ is prime $\Leftrightarrow$ $n$ is a prime number. Reasoning:
  \\\\
  $n\ZZ \subsetneq m\ZZ \subsetneq \ZZ$ if and only if $m | n$ but $m \neq 1$ and $m \neq n$ i.e. $ n$ is not prime.
  \\\\
  $(ab \in n\ZZ) \Leftrightarrow n | (ab)$ but if $n$ is prime then $n | (ab) \Leftrightarrow n | a$ or $n | b$. Hence $a \in n\ZZ$ or $b \in n\ZZ$.
  \\\\
  If $n$ is not prime, then $n = xy$ where $1 < x, y < n$ and $xy \in n\ZZ$ but $x \notin n \ZZ$ and $y \notin n\NN$. This would mean that $n\ZZ$ is not prime.
\end{example} 

\begin{example}
  In $\ZZ[x]$, the ideal $\langle x \rangle $ is prime but not maximal.
  \\\\
  \textbf{Maximal Proof}:
  \\\\
  $\langle x \rangle $ is not maximal because $\langle x \rangle \subsetneq \langle x, 2 \rangle \subsetneq \ZZ[x]$
  \\\\
  $\langle x, 2 \rangle $ consists of all polynomials of the form $f \cdot x + g \cdot 2$ (where $f, g \in \ZZ[x]$ ).  i.e. all polynomials whose consant term is even.
  \\\\
  $\langle x \rangle $ consists of all polynomials of the form $f \cdot x$. i.e. all polynomials whose constant term is zero.
  \\\\
  \textbf{Prime Proof}:
  \\\\
  $\langle x \rangle $ is prime because $f \cdot g \in \langle x \rangle \Rightarrow (f \in \langle x \rangle \ \text{or} \ g \in \langle x \rangle $ because if both $f$ and $g$ have non zero constant term than $f \cdot g$ has a non zero constant term.
\end{example} 

\begin{theorem}
  Let $R$ be a commutative ring with $1$. Let $I \subsetneq R$ be a proper ideal. Then:
  \\\\
  $I$ is maximal $\Leftrightarrow$ $R / I$ is a field.
  \\\\
  $I$ is prime $\Leftrightarrow$ $R / I$ is an integral domain.
\end{theorem} 

\begin{example}
  Let $R = \RR[x]$, and $I = \langle x^2 + 1 \rangle $. Then $R / I \cong \CC$. Note the following for gaining an intuition:
  \\\\
  \begin{gather*}
    (x + I)(x + I) = (x^2 + I) \\
    (x^2 + I) + (1 + I) = (x^2 + 1 + I) = 0 + I \Rightarrow (x^2 + I) = (-1 + I)
    \\\\
    i \leftrightarrow x + I \\
    1 \leftrightarrow 1 + I
  \end{gather*} 

  Going through an a demonstration:

  \begin{gather*}
    7x^3 - 3x^2 + x + 9 + I \leftrightarrow ? \in \CC \\
    (7x^3 + I) + (-3x^2 + I) + (x + I) + (9 + I) \\
    (7x + I)(x^2 + I) + (-3 + I)(x^2 + I) + (x + I) + (9 + I) \\
    (7x + I)(-1 + I) + (-3 + I)(-1 + I) + (x + I) + (9 + I) \\
    (-7x + I) + (3 + I) + (x + I) + (9 + I) \\
    (-6x + I) + (12 + I) 
  \end{gather*} 

\end{example} 

\end{document}
