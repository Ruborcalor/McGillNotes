\documentclass[class=scrartcl, crop=false]{standalone}

\usepackage[sexy]{evan}
\usepackage{cole}

\date{2019-12-02}


\begin{document}

\section{Lecture 12-02}

\begin{theorem}[Finding a field from an integral domain]
  Let $D$ be an integral domain. $\exists$ a field $\FF_D$ and an injective homomorphism $\phi: D \to \FF_D$ such that every $f \in F_D$ is equal to $\phi(d_1) \cdot \phi(d_2)^{-1}$ for some $d_1, d_2 \in D$.
  \\\\
  $\FF_D$ is called the \ul{field of fractions} equal associated to $D$.
\end{theorem} 

\begin{example}
  $\QQ[x]$ is not a field, but $\QQ(x) = \{p / q : p, q \in \QQ[x], \ q \neq 0 \}$ is a field.
\end{example} 
\begin{example}
  If $D$ is a field, then $D = \FF_D$.
\end{example} 

\subsection{Construction of $\FF_D$ }

We represent $a / b$ as $(a, b)$.
\begin{gather*}
  S = \{(a, b) \in D : b \neq 0\}
\end{gather*} 

Define $\sim $ on $S$ such that $(a_1, b_1) \sim (a_2, b_2)$ if and only if $a_1b_2 = a_2b_1$. i.e. 
\begin{gather*}
  \frac{a_1}{b_1} = \frac{a_2}{b_2} \quad \ \text{if and only if} \ \quad a_1b_2 = a_2b_1
\end{gather*} 

\begin{proof}[Proof that $\sim$ is an equiv relation]
  \begin{enumerate}
    \ii[]
    \ii
    Reflexive
    \\\\
    $a_1b_1 = a_1b_1$
    \\\\
    $(a_1, b_1) \sim (a_1, b_1)$ 
    \ii
    Symmetric
    \\\\
    If $(a_1, b_1) \sim (a_2, b_2)$, then $(a_2, b_2) \sim (a_1, b_1)$ 
    \ii
    Transitive
    \\\\
    If $(a_1, b_1) \sim (a_2, b_2)$ and $(a_2, b_2) \sim (a_3, b_3)$, then $(a_1, b_1) \sim (a_3, b_3)$.
  \end{enumerate} 
\end{proof} 

Claim: $+, \cdot$ are well-defined on $S / \sim$. Proof as an exercise.
\\\\
Notation:
\begin{gather*}
  (a_1, b_1) \cdot (a_2, b_2) = (a_1a_2, b_1b_2) \\
  \frac{a_1}{b_1} \cdot \frac{a_2}{b_2} = \frac{a_1b_2}{a_2b_2}
  \\\\
  (a_1, b_1) + (a_2, b_2) = (a_1b_2 + a_2b_1, b_1b_2) \\
  \frac{a_1}{b_1} + \frac{a_2}{b_2} = \frac{a_1b_2 + a_2b_1}{b_1b_2}
\end{gather*} 

Claim: $(S / \sim, \cdot, + , 0 / 1, 1 / 1) \equiv \FF_D$ is a field.
\\\\
with inverses
\begin{gather*}
  (\frac{a}{b})^{-1} = \frac{b}{a} \quad \ \text{if} \ \quad a \neq 0
  \\
  -(\frac{a}{b}) = (-\frac{a}{b})
\end{gather*} 

\begin{proof}
  Check associativity, distributivity, commutativity. (Exercise)
  \begin{gather*}
    \frac{0}{1} + \frac{a}{b} = \frac{0b + a 1}{b 1} = \frac{a}{b} \ \checkmark
    \\\\
    \frac{1}{1} \cdot \frac{a}{b} = \frac{1a}{1b} = \frac{a}{b} \ \checkmark
    \\\\
    \frac{a}{b} \neq 0 \quad (a \neq 0) \\
    \frac{a}{b} \cdot \frac{b}{a} = \frac{ab}{ab} \sim \frac{1}{1} \\
    \Rightarrow \frac{b}{a} = (\frac{a}{b})^{-1}
    \\\\
    \frac{a}{b} + \frac{-a}{b} = \frac{ab + (-a)b}{b^2} = \frac{0}{b^2} \sim \frac{0}{1} \\
    \Rightarrow -(\frac{a}{b}) = \frac{-a}{b}
  \end{gather*} 
\end{proof} 


\subsection{Factorization}

Let $R$ be a commutative ring with unity.

\begin{definition}[Definition of divides, unit, and associate]
  $a$ \ul{divides} $b$ if $\exists c \in R$ such that $a \cdot c = b$.
  \\\\
  A \ul{unit} $u$ is an element with an inverse.
  \\\\
  $a$ and $b$ are \ul{associates} if $a = b \cdot u$ for a unit $u$.
\end{definition} 

\begin{example}
  \begin{gather*}
    4 = 2 \cdot 2 = (-2) \cdot (-2) \\
    2 = \underbrace{(-1)}_{\text{unit}} \cdot (-2)
  \end{gather*} 
\end{example} 

\begin{note}
  Being associates is an equivalence relation.
  \begin{gather*}
    a = b \cdot u \\
    \Rightarrow b = a \cdot u^{-1}
  \end{gather*} 
\end{note} 

\begin{example}
  In $\ZZ$, associates are $\pm n$
\end{example} 

\begin{definition}[Irreducible. Prime] \leavevmode \\\\
  Suppose $D$ is an integral domain.
  \\\\
  $p \in D$ non-zero, non-unit is \ul{irreducible} if $p = ab \Rightarrow a$ is a unit \ul{or} $b$ is a unit.
  \\\\
  $p$ is \ul{prime} if $p \ | \ a \cdot b \Rightarrow p \ | \ a$  \ul{or} $p \ | \ b$.
  
\end{definition} 


\subsection{Summary}
Key take away: Really just trying to do what the rationals did to the integers, but to a general integral domain. This is useful because fields are very easy to work with, while integral domains are not very easy to work with.

What's special about the prime numbers? You can uniquely factor everything into a product of prime numbers. Everything you do with $\ZZ$ crucially lies on this fact. The question is whether or not you could do this with all general rings. The answer is not always, but many times you can.


\end{document}
