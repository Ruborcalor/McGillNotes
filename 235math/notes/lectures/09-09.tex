\documentclass[class=scrartcl, crop=false]{standalone}
\usepackage[sexy]{evan}
\usepackage{nicematrix}
\NiceMatrixOptions{transparent}


\begin{document}

\title{Notes 2019-09-09}

\section{Lecture 2019-09-09}

\begin{theorem}
  

Suppose $f: A \to B \ \text{and} \  g: B \to C$ are surjective then  $g \circ f : A \to C$ is surjective

Proof:
$c \in C $

since $g: B \to C$ there exists $b \in B $ s.t. $g(b) = c$ 
since $f:A \to B $ is surjective, exists $a \in A$ s.t. $f(a) = b$ 

thus $(g\circ f) (a) = g(f(a)) = g(b) = c$

\end{theorem}

\begin{definition}
  
A function $g:b \to A$ is inverse to function $f: A \to B$ if:
\begin{align*}
f \circ g = 1_B \\
g \circ f = 1_A
\end{align*} 
\begin{center}
  
\begin{gather*}
A \to^f B \to^g C\\
A \to^f B \to^g A\\
g \circ f = 1_A\\
\\
B \to^g A \to^f B\\
f \circ g = 1_B\\
\end{gather*}
\end{center}

\end{definition}

\begin{note}
  
They say f and g are inverible, use notation $f^-1$ for inverse of f

\end{note}

\begin{theorem}
Let $f: A \to B $ be a map:
f is invertible if and only if f is a bijection
\end{theorem}

\begin{align*}
P\Leftrightarrow Q \\
P \Leftarrow Q \\
P \Rightarrow Q \\
\end{align*} 

Proof that f is invertible means f is a bijection: 
\begin{align*}
  \ \text{let} \ g = f^{-1}
  \ \text{f is surjective since for all} \ b \in B \\
  \ \text{we have } \ f(g(b)) = f\circ g (b) = 1_B(b) = b \\\\
  \ \text{f is injective since if} \ f(a_1) = f(a_2) \Rightarrow g(f(a_1)) = g(f(a_2))
\end{align*} 

injective: if $f(a_1) = f(a_2)$, then $a_1 = a_2$

Proof that f is a bijection means f is invertible
define $f^{-1}: B \to A$ thus:


for each $b \in B$, there exists $a \in A \ \text{s.t.} \ f(a) = b$ and $a$ is unique with this property (by injectivity)

define $f^{-1}(b) = a$ then $f\circ f^{-1} (b) = f(f^{-1}(b)) = f(a) = b$ 
$f^{-1}\circ f(a) = f^{-1}(f(a)) = f^{-1}(b)$

so $f\circ f^{-1} = 1_B$


\begin{definition}[Equivalence Relation]
 Equivalence Relation on a set $X$ is a relation $R \subset X \times X$
\begin{align*}
  R \ \text{is reflexive} \ (x, x) \in R \ \text{for all} \ x \in X \\
  \ \text{is symmetric} \ (x, y) \in R \to (y, x) \in R \\
  \ \text{is transitive} \ (x, y) \in R \ \text{and} \ (y, z) \in R \to (x, z) \in R
\end{align*} 
 
\end{definition}

\begin{note}
Usually denote equiv relations by $x \sim y$ instead of $(x, y) \in R$ 
\begin{align*}
\ \text{or} \ x = y \\
x \equiv y \\
\end{align*} 
 
\end{note}

\begin{definition}
  
A partition of $X$ is a collection of disjoint nonempty subsets of $X$ whose union is $X$
\end{definition}

\begin{example}
 $\{X_k: k \in K \}$
 $x_i \cap x_j = \varnothing \ \text{for} \ i \neq j$

$\{1, 2, 3, 4, 5, 6, 7, 8, 9, 0 \} = \{1, 4, 5\} \cup \{6\} \cup \{9\} \cap \{2, 3, 7, 9, 0 \}$

$X = X_1 \cup X_2 \cup X_3 \cup X_4 $

\end{example}

\subsection{Creating a partition}

let $x$ be a set with equivalence relation $\sim$ 
for $y \in X$, let $[y] = \{x \in X : x \sim y \}$

$[y]$ is the equivalence class represented by y

\begin{theorem}
    Theorem 1.25: The equivalence classes of an equivalence relation ($\sim$ ) form a partition of X.
\end{theorem}

Proof:
\begin{gather*}
1. \ \text{each equiv class is nonempty since } \ y \in [y] \\
2. \ \text{equiv classes are either disjoint or equal since if } \ y \in [a] \ \text{and} \ y \in [b] \ \\
\text{then } \ [a] \subset [b] \ \text{since } \ c \in [a] \Rightarrow c \sim a \Rightarrow^{transitivity} c \sim y \Rightarrow^{transitivity} c \sim b \Rightarrow c \in [b] \\
  \ \text{similarly} \ [b] \subset [a] \\
3. X = \cup_{x \in X} [x]
\end{gather*} 
Conversely, given a partition of $X$ you can define an equivalence relation by declaring $x \sim y \Rightarrow x, y $ lie in the same part of the partition

\begin{note}
An equivalence relation is a disguised version of a partition
\end{note}

\begin{definition}
Definition: congruence modulo n equivalence relation on $Z$ 
\end{definition}

$a \equiv_n b$ if  $n$ divides $(b - a)$ 
i.e. $b - a =  mn$ for some $m \in Z$ 
do NOT use $(a \equiv b(mod n)$ 

EX.  $\equiv_2$ partition 
\begin{align*}
  \{, -4, -2, 0, 2, \cdot \} \\
  \{, -3, -1, 1, 3, \cdot \}
\end{align*} 

Proof: $\equiv_n $ is equiv relation
\begin{align*}
  1. a \equiv_n a \ \text{since} \  n|(a - a)  \\
2. (a \equiv_n b) \Rightarrow (b \equiv_n a) \ \text{since} \ n | (b - a) \ \text{then} \ n|(a - b) \\
3. a \equiv_n b \ \text{and} \ b \equiv_n c \ \text{then} \ a \equiv_n c
\end{align*} 
$n | (b - a) \ \text{and} \ n | (c - b) \ \text{so} \ n | (b - a) + (c - b)$


\end{document}
