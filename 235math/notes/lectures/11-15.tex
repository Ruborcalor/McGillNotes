\documentclass[class=scrartcl, crop=false]{standalone}

\usepackage[sexy]{evan}
\usepackage{cole}

\date{2019-11-15}


\begin{document}

\section{Lecture 11-15}

\begin{example}
  \begin{gather*}
    \ZZ_3[x] / I \ \text{where} \ I = \langle x^2 + 2 \rangle \\
    (x^2 + I) + (2 + I) = (x^2 + 2 + I) = (0 + I) \\
    (2 + I) + (1 + I) = (0 + I) \\
    \ \text{so} \ (x^2 + I) = -(2 + I) = (1 + I) \\
    \underbrace{((2x + 1) + I)}_{\text{Non zero}}\underbrace{((x + 1) + I)}_{\text{Non zero}} = (2x^2 + 2x + x + 1 + I) = (2x^2 + 1 + I) \\
    = (x^2 + I) + (x^2 + I) + (1 + I) = (1 + I) + (1 + I) + (1 + I) = (0 + I)
  \end{gather*} 
  Hence $\ZZ_3[x] / I$ is not an integral domain.
\end{example} 

\begin{theorem}[The Chinese Remainder Theorem]
  Let $n_1, n_2, n_3, \dots, n_k$ be positive integers with $\gcd(n_i, n_j) = 1$ for $i \neq j$.
  \\\\
  Then for any $a_1, a_2, a_3, \dots, a_k$, the following has a solution:
  \begin{gather*}
    x \equiv_{n_1} a_1 \\
    x \equiv_{n_2} a_2 \\
    \vdots \\
    x \equiv_{n_k} a_k \\
  \end{gather*} 
  Moreover, for any two solutions $x$ and $x'$, $x \equiv x' \mod(n_1 n_2 \cdots n_k)$.
  \begin{example}
    Generally, $\ZZ_p \times \ZZ_q \cong \ZZ_{pq}$ if $\gcd(p, q) = 1$. \\\\
    For example, $\ZZ_7 \times \ZZ_8 \cong \ZZ_{56}$.
  \end{example} 
  \begin{proof}
    Consider the homomorphism $\varphi: \ZZ \to \ZZ_7 \times \ZZ_8$ defined by
    \begin{gather*}
      \varphi(a) = \left([a]_7, [a]_8\right)
      \\\\
      \varphi(a + b) = ([a + b]_7, [a + b]_8) = ([a]_7 + [b]_7, [a]_8 + [b]_8) \\
      = ([a]_7, [a]_8) + ([b]_7, [b]_8) = \varphi(a) + \varphi(b)
      \\\\
      \varphi: \ZZ \to G \qquad \varphi(n) = g^n \ \text{where} \ g = (1, 1)
    \end{gather*} 
    $\varphi$ is surjective by the chinese remainder theorem. Indeed fo any $a_1, a_2$, there exists $x$ such that $x \equiv_7 a_1$ and $x \equiv_8 a_2$ so $\varphi(x) = (a_1, a_2)$.
    \begin{note}
      $[a]_7$ means $a \mod(7)$.
    \end{note} \noindent
    \\
    What is $\ker(\varphi)$?
    \\\\
    $\ker(\varphi) = 7\ZZ \cap 8\ZZ = 56 \ZZ$ by the first isomorphism theorem. 
    Because we know that $\varphi(\ZZ) = \ZZ_7 \times \ZZ_8$, and that by the first isomorphism theorem, $\varphi(\ZZ) \cong \ZZ / \ker(\varphi)$. And $\varphi(\ZZ) = \ZZ_7 \times \ZZ_8 \cong \ZZ_{56} = \ZZ / \ZZ_{56}$.
  \end{proof} 
\end{theorem} 

\begin{lemma}[16.41]
  Let $m$ and $n$ be positive integers with $\gcd(m, n) = 1$. Then for all $a, b \in \ZZ$,
  \begin{gather*}
    x \equiv_m a \\
    x \equiv_n b
  \end{gather*} has a solution.
  \\\\
  Moreover, the solution is unique $\mod (mn)$. i.e. if $x_1$ and $x_2$ are solutions, then $x_{1} \equiv_{mn} x_2$.
  \begin{example}
    \begin{gather*}
      x \equiv_7 6 \\
      x \equiv_8 4
    \end{gather*} has solution 20. The full set of solutions is $20 + 56 \ZZ$.
  \end{example} 
  \begin{proof}
    We know that $x \equiv_m a$ has solutions of the form $\{a + mp: p \in \ZZ\}$. We must find solutions such that 
    \begin{gather*}
      a + mp \equiv_n b \Rightarrow mp \equiv_n b - a
    \end{gather*} 
    But $\gcd(m, n) = 1$ implies that there exists $s, t$ such that $1 = sm + tn$. i.e. $s$ is the multiplicative inverse of $m$ in $\ZZ_n$. Hence 
    \begin{gather*}
      smp \equiv_n s(b - a) \\
      \Rightarrow p \equiv_n s(b - a)
    \end{gather*} 
    Therefore we have found $x$ which satisfies $x \equiv_m a$ and $x \equiv_n b$.
  \end{proof} \leavevmode \\\\
  Suppose $x_1$ and $x_2$ are both solutions. Then: 
  \begin{gather*}
    x_1 - x_2 \equiv_m 0 \\
    x_1 - x_2 \equiv_n 0
  \end{gather*} 
  Hence $m \, | \, (x_1 - x_2)$ and $n \, | \, (x_1 - x_2)$ so $mn \, | \, (x_1 - x_2)$.
\end{lemma} 


\end{document}
