\documentclass[class=scrartcl, crop=false]{standalone}

\usepackage[sexy]{evan}
\usepackage{nicematrix}
\NiceMatrixOptions{transparent}

\title{Math 235 Tutorial --- 09-30}

\begin{document}

\section{Lecture 09-30}

\subsection{Review Complex Numbers}

Recall $\CC = \{a + bi: a, b \in \RR\}$

Come equipped with:
\begin{enumerate}
  \ii
  Addition: $(a_1 + b_1i) + (a_2 + b_2i) = (a_1 + a_2) + (b_1 + b_2)i$
  \ii
  Multiplication:
\end{enumerate}

Complex numbers are associative and commutative under these operations.

There exists a complex conjugate.
\begin{example}
  Complex conjugate of $(a + bi) = (a - bi)$
\end{example}
\begin{note}
  $(a + bi)(a - bi) = a^2 + b^2$
\end{note}

$\CC^*$ is a multiplicative group of complex numbers. All imaginary numbers have an inverse.

Rectangular and Polar coordinates.

Rectangular: x axis is real component, y axis is imaginary component.

Polar: radius is $\sqrt{a^2 + b^2}$

$\sqrt{a^2 + b^2}\cos{\theta} + \sqrt{a^2 + b^2}\sin{\theta}i = \sqrt{a^2 + b^2}cis\theta = re^{i\theta}$

We have that $(r_1cis\theta_1)(r_2cis\theta_2) = r_1r_2cis(\theta_1 + \theta_2)$.

r is the "scale factor"

$cis\theta$ is the "rotation"

When $r = 1$, we get the subgroup of unit length com plex numbers.

$cis\theta = 1 =$ identity.
$(cis\theta)^{-1} = cis(\theta)$
$cis\theta_1(cis\theta_2) = cis(\theta_1 + \theta_2)$

The nth roots of unity are the solutions to $x^n = 1$ in $\CC^*$.

They form a cyclic subgroup of order n.
Form verticies of a polygon with n verticies.

Recall. Given a geometric object X, its group of isometries or symmetries Isom$(X)$ is a group with multiplication as composition of functions.

Recall. An isometry $f: X \to X$ is a distance preserving function.

dist $(p, q)$ = dist $(f(p), f(q) \forall p, q \in X$.

\begin{definition}
  Dinedral group - $D_n$ is the group of isometries of regular n-gon.
  \begin{example}
    $|D_n| = 2n$. It has n reflections and n rotations (counting the identity).

    Consider $n = 3$. This gives a regular triangle. There are three reflections and three rotations. In this example, reflections fix 1 vertex and midpoing of opposite edge. Rotations fix the center. Group of isometries is not abelian because if you do the same thing in different orders you get different results.
  \end{example}
\end{definition}

\ul{Lemma}. The set of rotations forms a subgroup.

\ul{Remark}. Each reflection is its own inverse.
\ul{Remark}. Product of two reflections is a rotation.Generally it is by twice the angle between the axes of reflection.

When n is even, reflections occur in two ways.
\begin{enumerate}
  \ii
  Fix two vertices
  \ii
  Fix two midpopints of opposite edges.
\end{enumerate}

Interpretation for $n = 1$ and $n = 2$.
\begin{enumerate}
  \ii
  $n = 2$ represented by isometry of "bigon" (looks like a lemon). 
  \ii
  $n = 1$ represented by isometry of a water droplet. $reflection, identity$.
\end{enumerate}
These two groups are the only abelian dinedral groups.

Let X be a rigid object. Its group of rigit motions consists of isometries that can be "physically realised". They are all rotations.

\begin{example}
  Let X be a "brick". Dimensions: $2 \times 3 \times 5$
  \begin{enumerate}
    \ii
    Isom X consists of 16 elements. We have 3 $\pi$ rotations, the identity, and reflections.
    \ii
    Rigid motions of X consists of 8 elements.
  \end{enumerate}
\end{example}


\end{document}
