\documentclass[class=scrartcl, crop=false]{standalone}

\usepackage[sexy]{/home/gautierk/.config/evan}
\usepackage{/home/gautierk/.config/Latex/cole}


\begin{document}

\section{Simple Groups}

\begin{definition}
  A group $G$ with no normal subgroups except $G$ and $\{1_G\} = \{e\}$ is called \ul{simple}.
\end{definition}

\begin{example}
  \begin{enumerate}
    \ii[]
    \ii
    $\ZZ_p$ with $p$ prime. The only subgroups are $G$ and $\{1_G\}$.
    \ii
    $A_n \quad \forall n \geq 5$.
  \end{enumerate}
\end{example}

In some sense, simple groups are like the primes. Every group can be built from simple groups.

\section{Homomorphisms}

\begin{definition}
  A \ul{homomorphism} from group $(G, \cdot)$ to $(H, \circ)$ is a map $\phi:G \to H$ such that is preserves multiplication. i.e. $\phi(g_1 \cdot g_2) = \phi(g_1)\circ\phi(g_2)$ for all $g_1, g_2 \in G$.
  \\\\
  The range $\phi(G) \subset H$ is called the \ul{homomorphic image} of $G$.
  \begin{remark}
    $\phi(G)$ is a subgroup of $H$.
  \end{remark}
  \begin{note}
    All isomorphisms are homomorphisms with the additional property that $\phi$ is a bijection.
  \end{note}
\end{definition}

\begin{example}
  Let $g \in G$. There is a homomorphism $\phi:\ZZ \to G$ defined by $\phi(n) = g^n$.
  \\\\
  Check: (review how binary operations apply below)
  \begin{gather*}
    \phi(a + b) = g^{a + b} = g^ag^b = \phi(a)\phi(b) \\
    \phi(\ZZ) = \langle g \rangle \subset G
  \end{gather*}
\end{example}
\begin{example}
  \begin{gather*}
    \det: \GL_n(\RR) \to \RR^* \\
    \det(AB) = \det(A) \cdot \det(B)
  \end{gather*}
\end{example}
\begin{example}
  Let $G = $ the isometries of a tetrahedron.
  \\\\
  $\phi:G \to \{\pm 1\}$. $\phi(g) = \pm 1$ if $g$ preserves orientation. $\phi(g) = -1$ if $g$ reverses orientation.
\end{example}
\begin{theorem}
  Let $\phi:G_1 \to G_2$ be a homomorphism.
  \begin{enumerate}
    \ii
    If $e_1$ is the identity element of $G_1$, the $\phi(e_1)$ is the identity element of $G_2$.
    \ii
    $\phi(g^{-1}) = [\phi(g)]^{-1}$ 
    \ii
    $H_1 \subset G_1$ is a subgroup $\Rightarrow \phi(H_1) \subset G_2$ is a subgroup
    \ii
    $H_2 \subset G_2$ is a subgroup $\Rightarrow \phi^{-1}(H_2)\subset G_1$ is a subgroup
    \ii
    $H_2 \subset G_2$ is a normal subgroup $\Rightarrow \phi^{-1}(H_2)\subset G_1$ is a normal subgroup
  \end{enumerate}
  \begin{note}
    Normal groups can be used to build factor and quotient groups.
  \end{note}
  \begin{proof}
    Of the above statements.
    \begin{enumerate}
      \ii
      $\phi(e_1) = \phi(e_1e_1) = \phi(e_1)\phi(e_1)$. Therefore $e_2 = \phi(e_1)$.
      \ii
      $e_2 = \phi(e_1) = \phi(g\cdot g^{-1}) = \phi(g) \cdot \phi(g^{-1})$. Therefore $\phi(g)$ and $\phi(g^{-1})$ are inverse to one another.
      \ii
      Identity: $e_1 \in H_1 \Rightarrow e_2 = \phi(e_1) \in \phi(H_1)$, so image of $\phi$ contains identity element.
      \\\\
      Inverses: $g_2 \in \phi(H_1) \Rightarrow g_2 = \phi(g_1)$ for some $g_1 \in H_1 \Rightarrow g_1^{-1} \in H_1 \Rightarrow \phi(g^{-1}) = [\phi(g_1)]^{-1} = g_2^{-1} \in \phi(H_1)$. Therefore  image contains inverses.
      \\\\
      Closure: Let $g_2, g_2' \in \phi(H_1)$. Therefore $\exists g_1, g_1' \in H_1$ such that $g_2 = \phi(g_1)$ and $g_2' = \phi(g_1')$. Therefore:
      \[
        g_1g_1' \in H_1 \Rightarrow \phi(g_1g_1') \in \phi(H_1) \Rightarrow g_2g_2'=\phi(g_1)\phi(g_1') \in \phi(H_1)
      \]
      \ii
      Identity: $e_1 \in \phi^{-1}(H_2)$ because $\phi(e_1) = e_2 \in H_2$.
      \\\\
      Inverses: $g_1 \in \phi^{-1}(H_2) \Rightarrow g_1^{-1} \in \phi^{-1}(H_2)$ because $\phi(g_1^{-1}) = [\phi(g_1)]^{-1} \in H_2$.
      \\\\
      Closure: $g_1,g_1' \in \phi^{-1}(H_2) \Rightarrow g_1g_1^{-1} \in \phi^{-1}(H_2)$ because $\phi(g_1g_1') = \phi(g_1)\phi(g_1') \in H_2$
      \ii
      Show that for all $g_1 \in G_1$, $g_1\phi^{-1}(H_2)g_1^{-1} \subset \phi^{-1}(H_2)$ 
      \\\\
      Let $k \in \phi^{-1}(H_2)$. Then $\phi(g_1kg_1^{-1}) = \phi(g_1)\phi(k)\phi(g_1^{-1}) = \phi(g_1)\phi(k)[\phi(g_1)]^{-1} \in H_2$. Since we construct with $H_2 \subset G_2$ is normal. Remember that $\phi(k) \in H_2$.
    \end{enumerate}
  \end{proof}
\end{theorem}
\end{document}
