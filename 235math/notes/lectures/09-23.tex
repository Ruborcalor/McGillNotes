\documentclass[class=scrartcl, crop=false]{standalone}

\usepackage[sexy]{evan}
\usepackage{nicematrix}
\NiceMatrixOptions{transparent}

\begin{document}

\title{Notes 2019-09-23}
\author{Cole Killian}

\section{Cyclic Groups}


\subsection{Cyclic Subgroup}

Let $g \in (G, \circ)$. Notation: $<g> = \{g^n: n \in \mathbb{Z}\}$

Let $g \in (G, +)$. Notation: $<g> = \{ng: n \in \mathbb{Z}\}$

\subsection{Examples}
$5 \in \mathbb{Z}$. $<5> = \{\dots, -10, -5, 0, 5, \dots\}$

$2 \in \mathbb{Z}$. $<2> = \{\text{even integers}\}$

$5 \in \mathbb{Z}_{10}. <5> = \{0, 5\}$

$6 \in \mathbb{Z}_{10}. <6> = \{6, 2, 8, 4, 0\}$

$2 \in \mathbb{Z}_{10}. <2> = \{2, 4, 6, 8, 0\}$

$3 \in \mathbb{Z}_{10}. <3> = \{3, 6, 9, 2, 5, 8, 1, 4, 7, 0\}$ 

Note: $<1> = <3> = <7> = <9> = \mathbb{Z}_{10}$. These capture the whole group.

 \fbox{\begin{minipage}{\linewidth}
Theorem 4.3 - Let G be a group. Let $x \in G$, then $<x>$ is a subgroup of G. Another way of thinking about it: $<x>$ is the smallest subgroup containing x.
\end{minipage}}



Definition / Notation: $<x>$ is the cyclic subgroup generated by x. If $G = <x>$, then G is a cyclic group and x is a \underline{generator} of G.

Detecting whether or not a subset is a subgroup.

Criteria

(0) Identity element.

(1) Inverse of each element is inside.

(2) Two elements inside, their product is inside.


\subsection{Proof}

(0) $x^0 \in <x>$ so $e \in <x>$.

(1) If $g \in <x>$ then $g = x^m$ for some $m \in \mathbb{Z}$. $g^{-1} = x^{-m}$ because $x^{-m} * x^{m} = x^0 = e$. Therefore $g^{-1} \in <x>$

(2) Let  $g, k \in <x>$, then $g = x^m \ \text{and} \ k = x^n$ for some $m, n \in \mathbb{Z}$so $g \circ h = x^m \circ x^n = x^{m + n} \in <x>$.

Note: Finite groups are really complicated.

The \underline{order} of x in G equals the smallest $n > 0$ such that $x^n = e$. If  $x^n \neq e$ for all $n > 0$ we declare x in G to have infinite order.

Definition / Notation: $|x|$ represents the order of x.

\subsection{Examples}

In $\mathbb{Z}_{10}$ :       $|5| = 2$, $|3| = 10$, $|0|$ = 1

3 in $\mathbb{Z}$ has infinite order. All x in Z have infinite order except the identity element.

$2 \in \mathbb{R}^*$.    $<2> = \{2^n: n \in \mathbb{Z}\} = \{\dots, \frac{1}{8}, \frac{1}{4}, \frac{1}{2}, 1, 2, \dots \}$. Infinite order.

\fbox{\begin{minipage}{\linewidth}
    Theorem 4.9 - Every cyclic group is abelian (commutative).
\end{minipage}}

\subsection{Proof}

Suppose G = $<x>$. For each $g, k \in G$ there exist $m, n \in \mathbb{Z}$ such that $g = x^m \ \text{and} \ k = x^n$

$g \circ k = x^m * x^n = x^{m + n} = x^{n + m} = x^n + x^m = k \circ g$

\subsection{Practice}

$\mathbb{Q}_8$. Quaternians. I'm not sure what the "8" is for.

$<i> = \{1, i, -1, -i\}$

$<-i> = \{1, -i, -1, i\}$ 

$<1> = \{1\}$

$<-1> = \{-1, 1\}$

$<j> = \{1, j, -1, -j\}$

Note to self: Groups are not necessarily commutative, but cyclic groups are always commutative. Review: Abelian. 

\subsection{The \underline{group of units modulo n}}

$U_n = \{m: 1 \leq m < n, \text{gcd}(m, n) = 1\}$ 

Binary operation: Multiply elements of $U_n$ by computing remainder of xy modulo n.

\subsection{Examples}

$U_{10} = \{1, 3, 7, 9\}$

Cayley Table: Can't make the table fast enough. Notes: each element appears once per row.

Changin to $U_{15}$ :

$U_{15} = \{1, 2, 4, 7, 8, 11, 13, 14\}$

$U_8 = \{1, 3, 5, 7\}$

$<1> = \{1\}$

$<3> = \{1, 3\}$

$<5> = \{1, 5\}$

$<7> = \{1, 7\}$

 $U_8$ is not cyclic. It is commutative because the cayley table is symmetric across $y = -x.$ 

 Remember: $U_n$ is abelian because xy mod n equals yx mod n ((because multiplication in integers is commutative)).

 $U_3 = \{1, 2\}$. Is it cyclic. Yes because $<2>$ generates it. $<2> = \{1, 2\}$

 $U_4 = \{1, 3\} = <3>$

 $U_5 = \{1, 2, 3, 4\} = <2> = \{1, 2, 4, 3\} = \{2^0, 2^1, 2^2, 2^3\}$







\end{document}
