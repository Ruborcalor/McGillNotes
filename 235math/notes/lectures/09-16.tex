\documentclass[class=scrartcl, crop=false]{standalone}

\usepackage[sexy]{/home/gautierk/.config/evan}
\usepackage{nicematrix}
\NiceMatrixOptions{transparent}

\begin{document}

\title{Notes 2019-09-16}
\author{Cole Killian}

\begin{theorem}
  
  Theorem There are infinitely many primes.
\end{theorem}

\begin{proof}[Proof: Argument by contradiction]

Suppose finetly many primes - $P_1, P_2, \dots, P_n$ 

let $p = p_1p_2p_3\dots p_n + 1$

 $p > p_n $ which means that p is not prime

 but every composite number has prime factor so $p = p_k r$ for some $k$ 

 impossible!

 $p_k r = p_k(p_1 \dots p_{k+1} \dots p_n) + 1$

 which would require that  $p_k | 1$ but this is impossible
\end{proof}

\begin{theorem}
  
 Theorem Fundametnal theorem of arithmetic

 let $n \in \mathbb{Z}$ with $n > 1$ 
 Then $n = p_1p_2\dots p_k$ is a product of primes

 This product is unique in a certain sense that:

 if $n = q_1q_2\dots q_l$, then $k = l$ and sequences are actually the same after reording them

 ex. $2 * 2 * 3 * 3 * 3 * 5 * 5$ 

 $5 * 2 * 3 * 2 * 5 * 3 * 3$

\end{theorem}
 \subsection*{Why is this true?}

 two things going on: \underline{exist} and \underline{unique}

 \subsection*{proof of existence:}

 Show by (strong) induction that for  $n \geq 2$,
$S_n = $ "n is a product of primes"

(base case) n = 2

2 is a product of primes. $2 = 2 \ \checkmark$

( (strong) induction): Either n+1 is prime, or  $n+1 = ab \ \text{where} \ 2 \leq a, b, \leq n$

by (strong) induction, $a = p_1p_2\dots p_k, b = q_1q_2\dots q_l$ where a and b are a product of primes. Therefore n+1 is a product of primes.


Proof of uniqueness. Note, new discussion, dosen't realte to previous proof

\subsection{Review proof of uniqueness}

suppose $p_1 \dots p_k$ = n = $q_1 \dots q_l$

assume  $p_1 \leq p_2 \leq \dots \leq p_k$ and $q_1 \leq q_2 \leq \dots \leq q_l$ 

assume $p_1 \leq q_1$

then $p_1 | n $ so $p_1 | q_k$ for some k

so $p_1 = q_k$ thus $p_1 \leq q_1 \leq q_k$

so  $p_1 = q_1$

now $(p_2 \dots p_k) = (q_2 \dots q_l)$ by induction $k = l$ and the sequence are the same. $n / p$ has a unique prime factorization and so

\subsection{Definition and example of Groups}

a \underline{binary operation} on a set G is a function $f: G \times G \to G$

math world is built out of binary operation: multiplication, subtraction, addition...

denote $f(a, b) $ by $a \circ b$ or $a \cdot b$ or  $ab$

Def: a \underline{group} $(G, \circ)$ is a set G with a binary operation $(a, b) \to a \cdot b \in G$

such that 

(1) the operation is associative. i.e. $(a \cdot b) \cdot c = a \cdot (b \cdot c)$

\subsection*{Review: associative, communative...} 

(2) there exists an \underline{identity element} $e \in G$ s.t. $e \cdot x = x = x \cdot e$ for all $x \in G$

(3) Each element  $x \in G$ has an \underline{inverse} $y \in G$ s.t. $x \cdot y = e$

$x^{-1}$ Often denotes inverse

We are blessed with a group theorist :)

\subsection*{example}
ex. $(\mathbb{Z}, +)$ is a group

(1) $(a + b) + c = a + (b + c)$

(2) $e = 0, a + 0 = a = 0 + a$

(3) inverse of $x$ denoted by  $-x$ 

\subsection*{idea}
$(G, \circ)$ is \underline{commutative} or \underline{abelian} if $a \circ b = b \circ a$ for all $a, b \in G$

\subsection*{examples of commutative groups}

ex. $(\mathbb{Z}, \cdot)$, $\cdot = $ "times"/multiplication is NOT a group

(1) yes associative $(a * b) * c = a * (b * c)$

(2) has identity element $e = 1$ 

(3) BUT inverses don't always exist. $2^{-1} = ?$. No integer inverse of 2

On the other hand: $(\mathbb{Q}*, \cdot)$ is a commutative group. Note: $\mathbb{Q}* = \mathbb{Q} - \{0\}$

identity (better word for  $e$ ) is 1

\subsubsection*{ex. $(\mathbb{Q}, +)$ is a commutative group.}
inverse of $\frac{2}{3} \ \text{is} \  -\frac{2}{3}$

\subsection*{definition: $(G, \circ)$ is a finite group if $G$ is a finite set.}
otherwise we call $G$ an infinite group.

What is more important when talking about a group. $G$ or $\circ$? The $\circ$, everything is built into the $\circ$. i.e. $G \times G \to^{f} G$ and  $(a, b) \to a \circ b$.

$|G|$ represents the number of elements in G

Let us now get familiar with Finite cyclic group $\mathbb{Z}_n$

Let  $\mathbb{Z}_n = \{0, 1, 2, \dots, n - 1\}$

Define binary operation  $a + b = c$ where $a + b \equiv_n c$ (called addition modulo n)

Turns out that this is a commutative group.  $(\mathbb{Z}_n, +)$ is a commutative group.

ex. in $\mathbb{Z}_n$ 

$2 + 2 = 4$, $3 + 3 = 1$, $4 + 1 = 0$, $4 + 4 = 3$

Requirements: 

(1) associative $\ \checkmark$

(2) 0 is the identity element

(3) Inverse exists. i.e. inverse of 3 = 2, inverse of 4 = 1, inverse of 1 = 4

\subsection*{Starting discussions on wednesday with Cayley table}

I'm not gonna be able to type this lmao

Grid like a multiplication table, but more general. "The Cayley table of a group". Summary of a binary operation.



\end{document}
